% \renewcommand{\thesubsection}{\Alph{section}}
\newpage
\titleformat{\chapter}[block]{\huge{}\bfseries\centering}{}{0.5em}{}

\chapter*{ХАВСРАЛТ}
\addcontentsline{toc}{chapter}{Хавсралт}

\section*{Хавсралт А. Эх кодын холбоос}

Төслийн бүх эх код, моделийн архитектур, техникийн индикаторууд болон бусад хэрэгжүүлэлтийг дараах GitHub repository-д байршуулсан:

\begin{center}
\Large
\textbf{\url{https://github.com/Asura-lab/Forex-Signal-App}}
\end{center}

\vspace{1cm}

\noindent\textbf{Repository-д байгаа гол бүрэлдэхүүн хэсгүүд:}

\begin{itemize}
    \item \textbf{backend/} - Flask REST API сервер, ML моделиуд, preprocessing код
    \item \textbf{mobile\_app/} - React Native + Expo мобайл апп
    \item \textbf{models/} - Сургагдсан моделиуд (XGBoost, LightGBM, Random Forest)
    \item \textbf{data/} - EUR/USD түүхэн өгөгдөл
    \item \textbf{docs/} - Техникийн баримт бичиг
\end{itemize}

\vspace{0.5cm}

\noindent\textbf{Техникийн хэрэгжүүлэлтүүд:}

\begin{itemize}
    \item ForexHybridNet моделийн архитектур (CNN + BiLSTM + Attention)
    \item RSI, MACD, Bollinger Bands техникийн индикаторууд
    \item Focal Loss функц (class imbalance шийдвэрлэх)
    \item Ensemble систем (XGBoost, LightGBM, Random Forest)
    \item JWT authentication систем
    \item MongoDB өгөгдлийн санд холбогдох
\end{itemize}

\section*{Хавсралт Б. Hyperparameter тохиргоо}

Энэхүү судалгаанд ашигласан V10 Ensemble систем нь XGBoost, LightGBM, CatBoost алгоритмуудын 7 өөр хувилбараас бүрдэнэ.

\subsection*{1. Өгөгдөл бэлтгэх ба Labeling тохиргоо}
\begin{table}[H]
\centering
\begin{tabular}{|l|c|l|}
\hline
\textbf{Параметр} & \textbf{Утга} & \textbf{Тайлбар} \\
\hline
Timeframe & 1 min & Ашигласан цагийн хуваарь \\
Forward Window & 60 & Таамаглах ирээдүйн хугацаа (бар) \\
Min Pips & 15 & Шаардлагатай хамгийн бага үнийн өөрчлөлт \\
Risk/Reward Ratio & 1.5 & Ашиг ба алдагдлын харьцаа (TP/SL) \\
Ensemble Threshold & 85\% & Итгэлцлийн босго утга \\
\hline
\end{tabular}
\caption{Labeling Parameters}
\end{table}

\subsection*{2. XGBoost Models Configuration}
\begin{table}[H]
\centering
\begin{tabular}{|l|c|c|c|}
\hline
\textbf{Parameter} & \textbf{XGB1 (Pri)} & \textbf{XGB2 (Deep)} & \textbf{XGB3 (Cons)} \\
\hline
n\_estimators & 600 & 400 & 800 \\
max\_depth & 6 & 8 & 4 \\
learning\_rate & 0.03 & 0.05 & 0.02 \\
subsample & 0.8 & 0.7 & 0.85 \\
colsample\_bytree & 0.8 & 0.7 & 0.85 \\
reg\_alpha & 0.1 & 0.05 & 0.2 \\
reg\_lambda & 1.0 & 0.5 & 2.0 \\
min\_child\_weight & 3 & - & 5 \\
gamma & - & 0.1 & - \\
\hline
\end{tabular}
\caption{XGBoost Hyperparameters}
\end{table}

\subsection*{3. LightGBM Models Configuration}
\begin{table}[H]
\centering
\begin{tabular}{|l|c|c|}
\hline
\textbf{Parameter} & \textbf{LGB1 (Primary)} & \textbf{LGB2 (Leaves)} \\
\hline
n\_estimators & 600 & 500 \\
max\_depth & 6 & 8 \\
learning\_rate & 0.03 & 0.04 \\
num\_leaves & 31 & 63 \\
min\_child\_samples & 30 & 20 \\
subsample & 0.8 & 0.75 \\
colsample\_bytree & 0.8 & 0.75 \\
reg\_alpha & 0.1 & 0.05 \\
reg\_lambda & 1.0 & 0.5 \\
\hline
\end{tabular}
\caption{LightGBM Hyperparameters}
\end{table}

\subsection*{4. CatBoost Models Configuration}
\begin{table}[H]
\centering
\begin{tabular}{|l|c|c|}
\hline
\textbf{Parameter} & \textbf{CAT1 (Primary)} & \textbf{CAT2 (Deeper)} \\
\hline
iterations & 600 & 500 \\
depth & 6 & 8 \\
learning\_rate & 0.03 & 0.04 \\
l2\_leaf\_reg & 3.0 & 2.0 \\
random\_strength & 0.5 & 0.3 \\
bagging\_temperature & 0.5 & 0.3 \\
\hline
\end{tabular}
\caption{CatBoost Hyperparameters}
\end{table}
