\chapter{СУДАЛГААНЫ ҮР ДҮН}

\section{Моделийн сургалтын үр дүн}

\subsection{Сургалтын процесс}

Модель нь 100 epoch-ийн турш сургагдах боломжтой байсан боловч early stopping-ийн улмаас бодит сургалт нь илүү эрт зогссон. Сургалтын явцад training loss болон validation loss-ийн динамик нь Зураг \ref{fig:training_loss}-д харуулсан.

\begin{figure}[H]
\centering
% Placeholder for training curve
\fbox{\parbox{0.8\textwidth}{\centering\vspace{2cm}Training Loss vs Validation Loss График\\(Notebook-ийн үр дүнгээс оруулах)\vspace{2cm}}}
\caption{Сургалтын loss-ийн өөрчлөлт}
\label{fig:training_loss}
\end{figure}

\subsection{Learning Rate Schedule}

Cosine Annealing with Warm Restarts scheduler ашигласан бөгөөд SWA (Stochastic Weight Averaging) 40-р epoch-оос эхэлсэн. Энэ нь моделийн generalization чадварыг сайжруулахад тусалсан.

\section{Үнэлгээний хэмжүүрүүд}

\subsection{Regression Metrics}

Үнийн таамаглалын (pip prediction) чанарыг дараах хэмжүүрүүдээр үнэлсэн:

\begin{table}[H]
\centering
\caption{Regression үнэлгээний үр дүн}
\begin{tabular}{|l|c|c|c|}
\hline
\textbf{Хэмжүүр} & \textbf{Validation} & \textbf{Test} & \textbf{OOS} \\
\hline
MAE (pips) & - & - & - \\
\hline
RMSE (pips) & - & - & - \\
\hline
$R^2$ Score & - & - & - \\
\hline
\end{tabular}
\label{tab:regression_metrics}
\end{table}

\textbf{Тайлбар:}
\begin{itemize}
    \item \textbf{MAE (Mean Absolute Error):} Таамагласан болон бодит утгын дундаж абсолют зөрүү
    \item \textbf{RMSE (Root Mean Squared Error):} Квадрат дундаж алдааны язгуур
    \item \textbf{$R^2$ Score:} Тодорхойлох коэффициент (1-д ойр байх тусам сайн)
\end{itemize}

\subsection{Direction Classification Metrics}

Үнийн чиглэлийн таамаглалын (өсөх/буурах) чанар:

\begin{table}[H]
\centering
\caption{Direction classification үнэлгээний үр дүн}
\begin{tabular}{|l|c|c|c|}
\hline
\textbf{Хэмжүүр} & \textbf{Validation} & \textbf{Test} & \textbf{OOS} \\
\hline
Accuracy & - & - & - \\
\hline
Precision & - & - & - \\
\hline
Recall & - & - & - \\
\hline
F1 Score & - & - & - \\
\hline
AUC-ROC & - & - & - \\
\hline
Brier Score & - & - & - \\
\hline
\end{tabular}
\label{tab:direction_metrics}
\end{table}

\textbf{Тайлбар:}
\begin{itemize}
    \item \textbf{Accuracy:} Зөв таамаглалын хувь
    \item \textbf{Precision:} Өсөх гэж таамагласны хэдэн хувь нь зөв байсан
    \item \textbf{Recall:} Бодит өсөлтийн хэдэн хувийг зөв таамагласан
    \item \textbf{AUC-ROC:} ROC муруйн доорх талбай (1-д ойр байх тусам сайн)
    \item \textbf{Brier Score:} Магадлалын калибрацийн хэмжүүр (0-д ойр байх тусам сайн)
\end{itemize}

\subsection{Big Move Classification Metrics}

Том хөдөлгөөний (±30 pips) таамаглалын чанар:

\begin{table}[H]
\centering
\caption{Big Move (±30 pips) classification үнэлгээний үр дүн}
\begin{tabular}{|l|c|c|c|c|}
\hline
\textbf{Зорилт} & \textbf{Precision} & \textbf{Recall} & \textbf{F1} & \textbf{AUC} \\
\hline
Up30 (+30 pips) & - & - & - & - \\
\hline
Down30 (-30 pips) & - & - & - & - \\
\hline
\end{tabular}
\label{tab:bigmove_metrics}
\end{table}

\section{Итгэлтэй таамаглалын шинжилгээ}

Моделийн итгэлийн түвшингийн дагуу гүйцэтгэлийг шинжилсэн. Өндөр итгэлтэй таамаглалууд илүү нарийвчлалтай байдаг:

\begin{table}[H]
\centering
\caption{Итгэлийн түвшингийн дагуу hit rate}
\begin{tabular}{|c|c|c|c|c|}
\hline
\textbf{Босго} & \textbf{Өсөх hit rate} & \textbf{Өсөх trades} & \textbf{Буурах hit rate} & \textbf{Буурах trades} \\
\hline
60\% & - & - & - & - \\
\hline
70\% & - & - & - & - \\
\hline
80\% & - & - & - & - \\
\hline
\end{tabular}
\label{tab:confidence_metrics}
\end{table}

\section{Калибрацийн шинжилгээ}

Моделийн магадлалын калибраци нь таамагласан магадлал болон бодит давтамж хоёр хэр нийцэж байгааг харуулдаг.

\begin{figure}[H]
\centering
% Placeholder for calibration curve
\fbox{\parbox{0.6\textwidth}{\centering\vspace{3cm}Calibration Curve\\(Notebook-ийн үр дүнгээс оруулах)\vspace{3cm}}}
\caption{Калибрацийн муруй (OOS)}
\label{fig:calibration}
\end{figure}

Хэрэв модель сайн калибрацитай бол калибрацийн муруй нь диагональ шулуунд ойрхон байна. Expected Calibration Error (ECE) нь калибрацийн чанарыг тоон утгаар илэрхийлнэ.

\section{Confusion Matrix}

\begin{figure}[H]
\centering
% Placeholder for confusion matrices
\fbox{\parbox{0.45\textwidth}{\centering\vspace{3cm}Up30 Confusion Matrix\vspace{3cm}}}
\hfill
\fbox{\parbox{0.45\textwidth}{\centering\vspace{3cm}Down30 Confusion Matrix\vspace{3cm}}}
\caption{Confusion matrix (a) Up30, (b) Down30}
\label{fig:confusion}
\end{figure}

\section{Backtesting үр дүн}

\subsection{Арилжааны стратеги}

Моделийн таамаглал дээр суурилсан энгийн threshold стратеги турших:

\begin{itemize}
    \item \textbf{Худалдан авах (BUY):} prob\_up $\geq$ 0.65
    \item \textbf{Зарах (SELL):} prob\_up $\leq$ 0.35
    \item \textbf{Хүлээх (HOLD):} Бусад тохиолдолд
\end{itemize}

Нэмэлтээр ``Big Move Filter'' ашиглан зөвхөн том хөдөлгөөн хүлээгдэж байгаа үед арилжаа хийх:
\begin{itemize}
    \item max(prob\_up30, prob\_down30) $\geq$ 0.25
\end{itemize}

\subsection{Стратегийн харьцуулалт}

\begin{table}[H]
\centering
\caption{Backtesting стратегийн харьцуулалт (OOS)}
\begin{tabular}{|l|c|c|c|c|c|}
\hline
\textbf{Стратеги} & \textbf{Trades} & \textbf{Hit Rate} & \textbf{Total Pips} & \textbf{Avg Pips} & \textbf{Max DD} \\
\hline
Conservative (60/40) & - & - & - & - & - \\
\hline
Moderate (65/35) & - & - & - & - & - \\
\hline
Aggressive (70/30) & - & - & - & - & - \\
\hline
Moderate + Filter & - & - & - & - & - \\
\hline
Aggressive + Filter & - & - & - & - & - \\
\hline
\end{tabular}
\label{tab:backtest}
\end{table}

\textbf{Тайлбар:}
\begin{itemize}
    \item \textbf{Trades:} Нийт арилжааны тоо
    \item \textbf{Hit Rate:} Ашигтай арилжааны хувь
    \item \textbf{Total Pips:} Нийт олсон/алдсан pip
    \item \textbf{Avg Pips:} Арилжаа бүрийн дундаж pip
    \item \textbf{Max DD:} Хамгийн их drawdown
\end{itemize}

\subsection{Cumulative PnL}

\begin{figure}[H]
\centering
% Placeholder for PnL curve
\fbox{\parbox{0.8\textwidth}{\centering\vspace{3cm}Cumulative PnL График\\(Notebook-ийн үр дүнгээс оруулах)\vspace{3cm}}}
\caption{Хуримтлагдсан ашиг/алдагдал (OOS)}
\label{fig:pnl}
\end{figure}

\section{Үнийн таамаглалын харьцуулалт}

\begin{figure}[H]
\centering
% Placeholder for price prediction
\fbox{\parbox{0.9\textwidth}{\centering\vspace{3cm}Actual vs Predicted Price График\\(Notebook-ийн үр дүнгээс оруулах)\vspace{3cm}}}
\caption{Бодит үнэ vs Таамагласан үнэ (OOS)}
\label{fig:price_pred}
\end{figure}

\section{Мобайл аппликейшн}

\subsection{Аппликейшний интерфейс}

React Native ашиглан хөгжүүлсэн мобайл апп нь дараах үндсэн дэлгэцүүдтэй:

\begin{enumerate}
    \item \textbf{Нүүр хуудас:} Одоогийн таамаглал, магадлал, санал болгох арилжааны чиглэл
    \item \textbf{Түүх:} Өмнөх таамаглалууд, тэдгээрийн үр дүн
    \item \textbf{Тохиргоо:} Push notification, threshold тохиргоо
\end{enumerate}

\begin{figure}[H]
\centering
% Placeholder for app screenshots
\fbox{\parbox{0.3\textwidth}{\centering\vspace{4cm}Home Screen\vspace{4cm}}}
\hfill
\fbox{\parbox{0.3\textwidth}{\centering\vspace{4cm}History Screen\vspace{4cm}}}
\hfill
\fbox{\parbox{0.3\textwidth}{\centering\vspace{4cm}Settings Screen\vspace{4cm}}}
\caption{Мобайл аппликейшний дэлгэцүүд}
\label{fig:app_screens}
\end{figure}

\section{Үр дүнгийн хэлэлцүүлэг}

\subsection{Давуу талууд}

\begin{enumerate}
    \item \textbf{Multi-task Learning:} Regression болон classification-ийг нэгтгэснээр илүү баялаг representation сурсан
    \item \textbf{Гибрид архитектур:} CNN локал загвар, LSTM урт хамаарал, Attention чухал цэгүүдийг илрүүлсэн
    \item \textbf{Орчин үеийн техникүүд:} Focal Loss, SWA зэрэг нь generalization сайжруулсан
    \item \textbf{Итгэлтэй таамаглал:} Өндөр итгэлтэй таамаглал илүү нарийвчлалтай
\end{enumerate}

\subsection{Хязгаарлалтууд}

\begin{enumerate}
    \item \textbf{Зах зээлийн өөрчлөлт:} Моделийн гүйцэтгэл зах зээлийн нөхцөл өөрчлөгдөхөд буурч болно
    \item \textbf{Гэнэтийн үйл явдал:} Геополитик үйл явдал, эдийн засгийн гэнэтийн мэдээнд модель бэлтгэлгүй
    \item \textbf{Slippage, Commission:} Backtesting-д арилжааны зардал тооцоогүй
    \item \textbf{Ганц валютын хос:} Зөвхөн EUR/USD дээр турших
\end{enumerate}

\subsection{Сайжруулах боломжууд}

\begin{enumerate}
    \item \textbf{Sentiment analysis:} Мэдээний sentiment-ийг шинж чанар болгон нэмэх
    \item \textbf{Ensemble:} Олон моделийн хослол ашиглах
    \item \textbf{Reinforcement Learning:} Position sizing, risk management-ийг RL-ээр сургах
    \item \textbf{Олон валютын хос:} Бусад валютын хосуудад өргөтгөх
\end{enumerate}
