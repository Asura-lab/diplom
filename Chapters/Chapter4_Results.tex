\chapter{СУДАЛГААНЫ ҮР ДҮН}

\section{Өгөгдлийн тойм}

\subsection{Сургалтын өгөгдөл}

Загваруудыг сургахад EUR/USD валютын хосын түүхэн өгөгдлийг ашигласан:

\begin{table}[H]
\centering
\caption{Өгөгдлийн статистик}
\label{tab:data_stats}
\begin{tabular}{|l|r|}
\hline
\textbf{Параметр} & \textbf{Утга} \\
\hline
Нийт бичлэг & 2,156,270 \\
\hline
Сургалтын өгөгдөл & 1,859,492 \\
\hline
Тестийн өгөгдөл & 296,778 \\
\hline
Цаг хугацааны интервал & 1 минут \\
\hline
Онцлогуудын тоо & 70 \\
\hline
Тестийн хугацаа & ~55 өдөр \\
\hline
\end{tabular}
\end{table}

\subsection{Өгөгдлийн тархалтын диаграмм}

Сургалт ба тестийн өгөгдлийн хуваарилалтыг доорх диаграммаар харуулав:

\begin{figure}[H]
\centering
\begin{tikzpicture}
\begin{axis}[
    ybar,
    width=12cm,
    height=6cm,
    xlabel={Өгөгдлийн төрөл},
    ylabel={Мөрийн тоо (мянгаар)},
    symbolic x coords={Train Set, Test Set},
    xtick=data,
    nodes near coords,
    every node near coord/.append style={font=\small},
    ymin=0,
    ymax=2100,
    bar width=1.2cm,
    enlarge x limits=0.4,
]
\addplot[fill=blue!60] coordinates {
    (Train Set, 1859)
    (Test Set, 297)
};
\end{axis}
\end{tikzpicture}
\caption{Өгөгдлийн тархалт (мянган мөрөөр)}
\label{fig:data_distribution}
\end{figure}

\subsection{BUY сигнал үүсгэх арга зүй}

BUY сигнал нь ирээдүйн үнийн өөрчлөлтөөс тодорхойлогдсон:
\begin{itemize}
    \item Forward period: 60 bar (1 цаг)
    \item Take Profit: 15 pips
    \item Stop Loss: 10 pips  
    \item Risk:Reward ratio: 1:1.5
    \item BUY condition: TP хүрсэн бөгөөд SL хүрээгүй
\end{itemize}

\begin{table}[H]
\centering
\caption{Сургалтын өгөгдлийн статистик}
\label{tab:signal_distribution}
\begin{tabular}{|l|r|r|}
\hline
\textbf{Dataset} & \textbf{Нийт мөр} & \textbf{BUY боломж} \\
\hline
Сургалтын өгөгдөл & 1,859,492 & 393,249 \\
\hline
Тестийн өгөгдөл & 296,778 & 80,296 \\
\hline
\end{tabular}
\end{table}

\subsection{Онцлогуудын хураангуй}

Нийт 70 техникийн индикаторыг дараах категориудаар бүлэглэв:

\begin{figure}[H]
\centering
\begin{tikzpicture}
\begin{axis}[
    ybar,
    width=12cm,
    height=6cm,
    xlabel={Категори},
    ylabel={Онцлогийн тоо},
    symbolic x coords={Trend, Momentum, Volatility, Candle, S/R},
    xtick=data,
    nodes near coords,
    bar width=1cm,
    ymin=0,
    ymax=30,
    enlarge x limits=0.15,
]
\addplot[fill=blue!60] coordinates {
    (Trend, 24)
    (Momentum, 18)
    (Volatility, 15)
    (Candle, 8)
    (S/R, 5)
};
\end{axis}
\end{tikzpicture}
\caption{Онцлогуудын категори бүрийн тоо}
\label{fig:feature_bar}
\end{figure}

\begin{itemize}
    \item \textbf{Trend индикаторууд (24):} SMA (5, 10, 20, 50, 100, 200), EMA (5, 10, 20, 50), MA crossover-ууд, Golden Cross, Price vs MA
    \item \textbf{Momentum индикаторууд (18):} RSI (7, 14, 21), MACD, Stochastic (14, 21), ROC (5, 10, 20), Momentum
    \item \textbf{Volatility индикаторууд (15):} ATR (14, 20), Bollinger Bands (width, position, squeeze), True Range
    \item \textbf{Candle Pattern (8):} Body size, Shadow ratio, Bullish/Bearish, Hammer, Engulfing
    \item \textbf{Support/Resistance (5):} Pivot Points (P, R1, R2, S1, S2), Near support
\end{itemize}

\section{Загваруудын гүйцэтгэл}

\subsection{V10 Ensemble загварын бүтэц}

BUY сигнал үүсгэхэд долоон машин сургалтын загварыг нэгтгэсэн V10 Ensemble арга ашигласан:

\begin{table}[H]
\centering
\caption{V10 Ensemble загваруудын бүрэлдэхүүн}
\label{tab:model_components}
\begin{tabular}{|l|c|c|l|}
\hline
\textbf{Загвар} & \textbf{Тоо} & \textbf{Төрөл} & \textbf{Онцлог} \\
\hline
XGBoost & 3 & Gradient Boosting & GPU дэмжлэгтэй, хурдан \\
\hline
LightGBM & 2 & Histogram-based & Санах ой бага, хурдан \\
\hline
CatBoost & 2 & Ordered Boosting & Categorical feature сайн \\
\hline
\multicolumn{2}{|c|}{\textbf{Нийт}} & \textbf{7 загвар} & Agreement Bonus System \\
\hline
\end{tabular}
\end{table}

\subsection{Загваруудын параметрүүд}

\textbf{XGBoost параметрүүд (3 хувилбар):}
\begin{itemize}
    \item \textbf{XGB-1:} n\_estimators=500, max\_depth=6, learning\_rate=0.03
    \item \textbf{XGB-2:} n\_estimators=400, max\_depth=8, learning\_rate=0.05
    \item \textbf{XGB-3:} n\_estimators=300, max\_depth=5, learning\_rate=0.08
    \item Бүгдэд: subsample=0.8, colsample\_bytree=0.8, tree\_method=hist
\end{itemize}

\textbf{LightGBM параметрүүд (2 хувилбар):}
\begin{itemize}
    \item \textbf{LGB-1:} n\_estimators=500, max\_depth=6, learning\_rate=0.03
    \item \textbf{LGB-2:} n\_estimators=400, max\_depth=8, learning\_rate=0.05
    \item Бүгдэд: subsample=0.8, colsample\_bytree=0.8
\end{itemize}

\textbf{CatBoost параметрүүд (2 хувилбар):}
\begin{itemize}
    \item \textbf{CAT-1:} iterations=500, depth=6, learning\_rate=0.03
    \item \textbf{CAT-2:} iterations=400, depth=8, learning\_rate=0.05
    \item Бүгдэд: bootstrap\_type=Bayesian
\end{itemize}

\subsection{V10 Ensemble нэгтгэх арга}

Долоон загварын магадлалыг жинлэгдсэн дунджаар нэгтгэж, Agreement Bonus нэмнэ:

\begin{equation}
P_{base}(BUY) = \sum_{i=1}^{7} w_i \cdot P_i(BUY)
\end{equation}

\begin{equation}
P_{final}(BUY) = P_{base} + \text{Agreement Bonus}
\end{equation}

\textbf{Agreement Bonus System:}
\begin{itemize}
    \item 7/7 загвар ижил таамаглал: +7\% bonus
    \item 6/7 загвар ижил таамаглал: +4\% bonus
    \item 5/7 загвар ижил таамаглал: +2\% bonus
\end{itemize}

\textbf{Model Agreement:} 85\%+ confidence дохиотой үед дунджаар 6.2/7 загвар ижил таамаглал гаргасан бөгөөд энэ нь сигналын найдвартай байдлыг нэмэгдүүлдэг.

\section{BUY сигналын нарийвчлал}

\subsection{V10 Итгэлцлийн түвшингийн дүн шинжилгээ}

V10 Ensemble загварын confidence түвшингээр BUY сигналыг шүүж үзвэл, итгэлцлийн түвшин нэмэгдэх тусам backtest-ийн accuracy болон profit factor мэдэгдэхүйц сайжирч байна:

\begin{table}[H]
\centering
\caption{V10 Итгэлцлийн түвшин бүрийн BUY сигналын гүйцэтгэл}
\label{tab:confidence_accuracy}
\begin{tabular}{|c|c|c|c|c|}
\hline
\textbf{Confidence} & \textbf{Сигналын тоо} & \textbf{Accuracy} & \textbf{Total Pips} & \textbf{Profit Factor} \\
\hline
$\geq$ 75\% & 312 & 76.9\% & +3,936 & 5.56 \\
\hline
$\geq$ 80\% & 189 & 94.7\% & +3,460 & 29.8 \\
\hline
\rowcolor{green!20}
$\geq$ 85\% & 97 & \textbf{96.9\%} & \textbf{+1,844} & \textbf{52.2} \\
\hline
$\geq$ 90\% & 34 & 100.0\% & +680 & $\infty$ \\
\hline
\end{tabular}
\end{table}

\begin{figure}[H]
\centering
\begin{tikzpicture}
\begin{axis}[
    ybar,
    width=13cm,
    height=7cm,
    xlabel={Итгэлцлийн түвшин},
    ylabel={Accuracy (\%)},
    symbolic x coords={75\%+, 80\%+, 85\%+, 90\%+},
    xtick=data,
    nodes near coords,
    every node near coord/.append style={font=\small},
    ymin=70,
    ymax=105,
    bar width=1cm,
    enlarge x limits=0.2,
]
\addplot[fill=blue!60] coordinates {
    (75\%+, 76.9)
    (80\%+, 94.7)
    (85\%+, 96.9)
    (90\%+, 100.0)
};
\end{axis}
\end{tikzpicture}
\caption{V10 Итгэлцлийн түвшин ба Accuracy хамаарал}
\label{fig:confidence_accuracy}
\end{figure}

\subsection{Гол олдвор}

Судалгааны гол олдвор нь \textbf{V10-ийн 85\%+ итгэлцэлтэй BUY сигнал 96.9\% accuracy, 52.2 profit factor} үзүүлсэн явдал юм. Энэ нь:
\begin{itemize}
    \item 97 BUY сигналаас 94 нь ашигтай байсан (зөвхөн 3 алдаа)
    \item 7 загварын 6+ нь ижил таамаглал гаргасан үед итгэлцэл нэмэгддэг
    \item Өндөр итгэлцэлтэй сигнал цөөн ч маш өндөр нарийвчлалтай
    \item Profit Factor 52.2 нь алдагдлаас 52 дахин их ашиг олсон гэсэн үг
    \item Agreement Bonus систем нь итгэлцүүрийг нэмэгдүүлж, алдааг багасгасан
\end{itemize}

\section{Backtest үр дүн}

\subsection{Эрсдэлийн удирдлага (Dynamic SL/TP)}

ATR (Average True Range) индикатор дээр суурилан Stop Loss, Take Profit-ийг динамикаар тооцсон нь зах зээлийн volatility-д тохирсон эрсдэлийн удирдлагыг хангадаг:

\begin{table}[H]
\centering
\caption{Dynamic SL/TP тохиргоо}
\label{tab:sl_tp}
\begin{tabular}{|l|c|c|}
\hline
\textbf{Параметр} & \textbf{Томъёо} & \textbf{Хүрээ} \\
\hline
Stop Loss & $1.5 \times ATR$ & 10-20 pips \\
\hline
Take Profit & $2.5 \times ATR$ & 20-40 pips \\
\hline
Risk:Reward & - & 1:1.5 - 1:2 \\
\hline
\end{tabular}
\end{table}

\subsection{V10 85\%+ confidence Backtest дэлгэрэнгүй}

Оновчтой итгэлцлийн түвшин болох 85\%+ дээр дэлгэрэнгүй үр дүн:

\begin{table}[H]
\centering
\caption{V10 85\%+ confidence backtest статистик}
\label{tab:backtest_detail}
\begin{tabular}{|l|r|}
\hline
\textbf{Үзүүлэлт} & \textbf{Утга} \\
\hline
Нийт BUY сигнал & 97 \\
\hline
Зөв таамаглал & 94 (96.9\%) \\
\hline
Буруу таамаглал & 3 (3.1\%) \\
\hline
Нийт ашиг & \textbf{+1,844 pips} \\
\hline
Дундаж ашиг/арилжаа & +19.0 pips \\
\hline
Profit Factor & \textbf{52.2} \\
\hline
Өдөрт сигналын тоо & ~1.8 \\
\hline
\end{tabular}
\end{table}

\textbf{96.9\% accuracy} гэдэг нь 100 BUY сигналаас зөвхөн 3 нь буруу гэсэн үг бөгөөд энэ нь маш өндөр найдвартай байдлыг харуулж байна.

\begin{figure}[H]
\centering
\begin{tikzpicture}
\begin{axis}[
    ybar,
    width=13cm,
    height=7cm,
    xlabel={Итгэлцлийн түвшин},
    ylabel={Profit Factor},
    symbolic x coords={75\%+, 80\%+, 85\%+},
    xtick=data,
    nodes near coords,
    every node near coord/.append style={font=\small},
    ymin=0,
    ymax=60,
    bar width=1cm,
    enlarge x limits=0.3,
]
\addplot[fill=orange!60] coordinates {
    (75\%+, 5.56)
    (80\%+, 29.8)
    (85\%+, 52.2)
};
\end{axis}
\end{tikzpicture}
\caption{V10 Итгэлцлийн түвшин ба Profit Factor}
\label{fig:profit_factor}
\end{figure}

\section{Санал болгох тохиргоо}

Backtest үр дүнд суурилан хоёр горим санал болгож байна:

\begin{table}[H]
\centering
\caption{V10 Санал болгох тохиргоо}
\label{tab:recommendations}
\begin{tabular}{|l|c|c|c|c|}
\hline
\textbf{Горим} & \textbf{Confidence} & \textbf{Accuracy} & \textbf{Өдөрт сигнал} & \textbf{PF} \\
\hline
Идэвхтэй & $\geq$ 80\% & 94.7\% & ~3.4 & 29.8 \\
\hline
\rowcolor{green!20}
Хамгаалалттай & $\geq$ 85\% & \textbf{96.9\%} & ~1.8 & \textbf{52.2} \\
\hline
\end{tabular}
\end{table}

\begin{itemize}
    \item \textbf{Идэвхтэй горим (80\%+):} Илүү олон сигнал, 94.7\% accuracy, өдөрт ~3.4 сигнал
    \item \textbf{Хамгаалалттай горим (85\%+):} Цөөн сигнал, 96.9\% accuracy, бага эрсдэл (Санал болгосон)
\end{itemize}

\section{Системийн гүйцэтгэл}

\subsection{Хариу үйлдлийн хурд}

\begin{table}[H]
\centering
\caption{Системийн гүйцэтгэлийн хэмжүүрүүд}
\label{tab:system_performance}
\begin{tabular}{|l|c|}
\hline
\textbf{Хэмжүүр} & \textbf{Утга} \\
\hline
Онцлог тооцоолох хугацаа & ~50ms \\
\hline
Загвар таамаглах хугацаа & ~30ms \\
\hline
API дуудлагын хугацаа & ~100-200ms \\
\hline
Нийт дохио үүсгэх хугацаа & <500ms \\
\hline
\end{tabular}
\end{table}

\section{Үр дүнгийн дүгнэлт}

\subsection{Гол үр дүнгүүд}

Судалгааны гол үр дүнгүүд:

\begin{enumerate}
    \item \textbf{Backtest үр дүн:} 85\%+ confidence BUY сигнал нь:
    \begin{itemize}
        \item \textbf{96.9\% accuracy} (97 сигналаас 94 нь зөв)
        \item \textbf{+1,844 pips} нийт ашиг (~55 өдөрт)
        \item \textbf{52.2 profit factor} (алдсанаас 52 дахин их ашиг)
        \item Өдөрт ~1.8 сигнал
    \end{itemize}
    
    \item \textbf{V10 Ensemble арга:} XGBoost×3, LightGBM×2, CatBoost×2 долоон загварыг нэгтгэснээр маш өндөр нарийвчлалтай сигнал үүсгэсэн
    
    \item \textbf{Agreement Bonus:} 7 загвар ижил таамаглал гаргахад +7\%, 6/7 дээр +4\%, 5/7 дээр +2\% bonus нэмдэг
    
    \item \textbf{Dynamic SL/TP:} ATR-д суурилсан динамик Stop Loss/Take Profit нь зах зээлийн volatility-д тохирсон эрсдэлийн удирдлагыг хангасан
    
    \item \textbf{Entry/SL/TP гаралт:} V10 нь entry price, stop loss, take profit утгуудыг тодорхой гаргадаг
\end{enumerate}

\subsection{V10 vs Өмнөх хувилбарууд харьцуулалт}

\begin{table}[H]
\centering
\caption{V10 vs V2 харьцуулалт}
\begin{tabular}{|l|c|c|}
\hline
\textbf{Үзүүлэлт} & \textbf{V2 (3 загвар)} & \textbf{V10 (7 загвар)} \\
\hline
Загварын тоо & 3 & 7 \\
\hline
Accuracy (85\%+) & 68.8\% & \textbf{96.9\%} \\
\hline
Profit Factor & 3.10 & \textbf{52.2} \\
\hline
Agreement System & Байхгүй & +7/+4/+2\% bonus \\
\hline
Entry/SL/TP & Тусдаа тооцоо & Шууд гарна \\
\hline
\end{tabular}
\end{table}

\subsection{Үр дүнгийн хураангуй}

\begin{figure}[H]
\centering
\begin{tikzpicture}
    % Summary box
    \draw[very thick, fill=green!10, rounded corners=10pt] (0,0) rectangle (14,7);
    
    \node[font=\Large\bfseries] at (7,6.3) {V10 Судалгааны гол үр дүн (85\%+ confidence)};
    
    \draw[thick] (0.5,5.5) -- (13.5,5.5);
    
    % Results with icons
    \node[anchor=west, font=\large] at (1,4.7) {Backtest Accuracy: \textbf{96.9\%} (97 сигналаас 94 зөв)};
    \node[anchor=west, font=\large] at (1,3.7) {Profit Factor: \textbf{52.2}};
    \node[anchor=west, font=\large] at (1,2.7) {Total Profit: \textbf{+1,844 pips}};
    \node[anchor=west, font=\large] at (1,1.7) {Signals per Day: \textbf{~1.8}};
    \node[anchor=west, font=\large] at (1,0.7) {Models: \textbf{7 (XGB×3, LGB×2, CAT×2)}};
    
\end{tikzpicture}
\caption{V10 Судалгааны үр дүнгийн хураангуй}
\label{fig:summary}
\end{figure}

\subsection{Практик ач холбогдол}

Энэхүү судалгааны үр дүн нь:
\begin{itemize}
    \item Forex арилжаанд машин сургалтын V10 Ensemble загварыг амжилттай хэрэглэж болохыг харуулсан
    \item 85\%+ Confidence шүүлтүүр ашиглан 96.9\% accuracy хүрсэн
    \item 7 загварын Ensemble арга нь дан загвараас маш их давуу
    \item Agreement Bonus систем нь сигналын найдвартай байдлыг мэдэгдэхүйц нэмэгдүүлсэн
    \item Dynamic SL/TP нь эрсдэлийн удирдлагыг автоматжуулдаг
    \item Profit Factor 52.2 нь практикт маш өндөр түвшин юм
    \item Entry price, Stop Loss, Take Profit-ийг шууд гаргадаг
\end{itemize}
