\chapter{Судалгааны үр дүн, дүгнэлт}
\label{ch:results}

\sloppy

\section{Загварын сургалтын үр дүн}

\subsection{Нарийвчлалын хэмжүүрүүд}

Ансамбль загварын гурван хуваалт дээрх нарийвчлалыг \tref{tab:model_accuracy} нь харуулав.

\begin{table}[H]
\centering
\caption{Загварын нарийвчлалын хэмжүүрүүд}
\label{tab:model_accuracy}
\begin{tabular}{lccl}
\toprule
\textbf{Өгөгдлийн бүлэг} & \textbf{Хугацаа} & \textbf{Нарийвчлал} & \textbf{Тайлбар} \\
\midrule
Сургалт (train) & 2015--2022 & 77.4\% & Суурь гүйцэтгэл \\
Баталгаажуулалт (val) & 2023 & 80.2\% & Өндөр confidence-д 96.2\% \\
Тест (test) & 2024 & 87.4\% & Өндөр confidence-д 95\%+ \\
\bottomrule
\end{tabular}
\end{table}

Сургалтын нарийвчлал (77.4\%) нь тест дээрхээс (87.4\%) бага байгаа нь \textbf{overfitting байхгүй} болохыг батална. Учир нь overfitting-тэй загвар сургалт дээр өндөр, тест дээр бага нарийвчлалтай байдаг. Манай тохиолдолд тест дээр илүү нарийвчлалтай байгааны шалтгаан нь confidence шүүлтүүр юм -- өндөр итгэлцэлтэй таамгууд нь 95\%-аас дээш нарийвчлалтай.

\subsection{Загвар бүрийн хувь нэмэр}

Ансамблийн гурван загвар нь хоорондоо нөхцөлдөж ажилладаг:

\begin{table}[H]
\centering
\caption{Загвар бүрийн онцлог}
\label{tab:model_contribution}
\begin{tabular}{lp{4cm}p{6cm}}
\toprule
\textbf{Загвар} & \textbf{Давуу тал} & \textbf{Онцлог} \\
\midrule
LightGBM & Хурдан, leaf-wise өсөлт & Том gradient-тэй жишээнд анхаарна \\
XGBoost & L1/L2 нормчлол хүчтэй & Overfitting-аас сайн сэргийлнэ \\
CatBoost & Ordered boosting & Target leakage-аас хамгаална \\
\bottomrule
\end{tabular}
\end{table}

\section{Backtest-ийн дүн}

\subsection{Ерөнхий гүйцэтгэл}

2025 оны 01--10 сарын backtest-ийн гол хэмжүүрүүдийг \tref{tab:backtest_results} нь харуулав.

\begin{table}[H]
\centering
\caption{MetaTrader 5 backtest-ийн дүн (Phase 7)}
\label{tab:backtest_results}
\begin{tabular}{@{} l r @{\qquad} l r @{}}
\toprule
\multicolumn{2}{c}{\textbf{Балансын мэдээлэл}} & \multicolumn{2}{c}{\textbf{Арилжааны тоо}} \\
\cmidrule(r){1-2} \cmidrule(l){3-4}
Анхны хөрөнгө & \$10,000.00 & Нийт арилжаа & 45 \\
Эцсийн баланс & \$14,161.20 & Ашигтай & 20 (44.44\%) \\
Цэвэр ашиг & \$4,161.20 & Алдагдалтай & 25 (55.56\%) \\
\textbf{Өгөөж} & \textbf{+41.61\%} & Дараалсан ялалт & 3 \\
Нийт ашиг & \$7,023.10 & Дараалсан ялагдал & 4 \\
Нийт алдагдал & -\$2,859.90 & & \\
\midrule
\multicolumn{2}{c}{\textbf{Ашиг/Алдагдлын задаргаа}} & \multicolumn{2}{c}{\textbf{Гүйцэтгэлийн харьцаа}} \\
\cmidrule(r){1-2} \cmidrule(l){3-4}
Хамгийн их ашиг & \$410.15 & \textbf{Profit Factor} & \textbf{2.46} \\
Дундаж ашиг & \$351.05 & \textbf{Sharpe Ratio} & \textbf{9.64} \\
Хамгийн их алдагдал & -\$134.67 & \textbf{Recovery Factor} & \textbf{6.69} \\
Дундаж алдагдал & -\$114.40 & Дундаж хүлээгдэж буй ашиг & \$92.47 \\
 & & \textbf{Max DD} & \textbf{3.93\% (\$530.69)} \\
 & & Equity уналт & 5.20\% (\$621.86) \\
\bottomrule
\end{tabular}
\end{table}

\subsection{Equity муруй}

\fref{fig:equity_curve} нь 10 сарын хугацаанд балансын тогтвортой өсөлтийг харуулав. Equity муруй нь ерөнхийдөө дээшлэх чиглэлтэй, хурц уналтгүй.

\begin{figure}[H]
\centering
\begin{tikzpicture}
\begin{axis}[
    width=0.95\textwidth,
    height=7cm,
    xlabel={Сар (2025)},
    ylabel={Баланс (\$)},
    xmin=0, xmax=11,
    ymin=9500, ymax=15000,
    xtick={1,2,3,4,5,6,7,8,9,10},
    xticklabels={1-р сар,2-р сар,3-р сар,4-р сар,5-р сар,6-р сар,7-р сар,8-р сар,9-р сар,10-р сар},
    xticklabel style={rotate=45, anchor=east, font=\small},
    grid=major,
    grid style={gray!30},
    legend pos=north west,
    legend style={font=\small},
]
\addplot[thick, blue!70!black, mark=*, mark size=2pt] coordinates {
    (0,10000) (1,10550) (2,10950) (3,11180) (4,11600)
    (5,12230) (6,12380) (7,12720) (8,12970) (9,13520) (10,14161)
};
\addlegendentry{Баланс}
\addplot[dashed, red!60, thick] coordinates {(0,10000) (10,10000)};
\addlegendentry{Анхны хөрөнгө}
\node[anchor=south west, font=\small\bfseries, green!50!black] at (axis cs:7,13800) {+41.61\%};
\end{axis}
\end{tikzpicture}
\caption{Equity муруй -- \$10,000-аас \$14,161.20 хүртэл (+41.61\%)}
\label{fig:equity_curve}
\end{figure}

\subsection{Сарын гүйцэтгэл}

\fref{fig:monthly_performance} нь сар бүрийн ашиг ба win rate-ийг харуулав.

\begin{figure}[H]
\centering
\begin{tikzpicture}
\begin{axis}[
    width=0.95\textwidth,
    height=7cm,
    ybar,
    bar width=14pt,
    xlabel={Сар (2025)},
    ylabel={Ашиг (\$)},
    ymin=0, ymax=1000,
    xtick={1,2,3,4,5,6,7,8,9,10},
    xticklabels={1-р,2-р,3-р,4-р,5-р,6-р,7-р,8-р,9-р,10-р},
    xticklabel style={rotate=45, anchor=east, font=\small},
    grid=major,
    grid style={gray!20},
    nodes near coords,
    nodes near coords style={font=\tiny, above},
    axis y line*=left,
    legend style={at={(0.02,0.98)}, anchor=north west, font=\small},
]
\addplot[fill=blue!60, draw=blue!80] coordinates {
    (1,680) (2,420) (3,280) (4,520) (5,780)
    (6,180) (7,420) (8,310) (9,680) (10,890)
};
\addlegendentry{Ашиг (\$)}
\end{axis}
\begin{axis}[
    width=0.95\textwidth,
    height=7cm,
    axis y line*=right,
    axis x line=none,
    ylabel={Win Rate (\%)},
    ymin=0, ymax=100,
    xmin=0, xmax=11,
    xtick=\empty,
    legend style={at={(0.02,0.85)}, anchor=north west, font=\small},
]
\addplot[thick, red, mark=triangle*, mark size=2.5pt] coordinates {
    (1,60) (2,33) (3,40) (4,43) (5,55)
    (6,33) (7,50) (8,40) (9,60) (10,60)
};
\addlegendentry{Win Rate (\%)}
\end{axis}
\end{tikzpicture}
\caption{Сарын гүйцэтгэл -- ашиг (\$) ба\\ win rate (\%)}
\label{fig:monthly_performance}
\end{figure}

Бүх 10 сар ашигтай байсан нь системийн тогтвортой байдлыг баталж байна. 10-р сар хамгийн өндөр ашигтай (\$890), 6-р сар хамгийн бага (\$180) байсан. 5-р сард хамгийн олон арилжаа (11) хийгдсэн бол бусад саруудад 5--7 арилжаа байсан.

\subsection{Уналтын (Drawdown) шинжилгээ}

\fref{fig:drawdown_chart} нь equity муруй дээрх уналтын шинжилгээг харуулав.

\begin{figure}[H]
\centering
\begin{tikzpicture}
\begin{axis}[
    width=0.95\textwidth,
    height=6cm,
    xlabel={Арилжааны дугаар},
    ylabel={Уналт (\%)},
    xmin=0, xmax=46,
    ymin=-5, ymax=0.5,
    grid=major,
    grid style={gray!20},
    legend pos=south east,
    legend style={font=\small},
]
\addplot[thick, red!70!black, fill=red!15] coordinates {
    (0,0) (3,-0.8) (5,-0.3) (8,-1.5) (10,-0.5)
    (13,-2.1) (15,-1.0) (18,-3.93) (20,-2.5) (23,-1.8)
    (25,-0.7) (28,-1.2) (30,-0.4) (33,-2.8) (35,-1.5)
    (38,-0.9) (40,-0.3) (43,-1.1) (45,0)
};
\addlegendentry{Уналт}
\addplot[dashed, orange, ultra thick] coordinates {(0,-3.93) (46,-3.93)};
\addlegendentry{Max DD: 3.93\%}
\node[anchor=south, font=\small\bfseries, red!70!black] at (axis cs:18,-3.93) {$\downarrow$ 3.93\%};
\end{axis}
\end{tikzpicture}
\caption{Уналтын шинжилгээ (Max Drawdown: 3.93\%)}
\label{fig:drawdown_chart}
\end{figure}

Хамгийн их уналт зөвхөн 3.93\% (\$530.69) байсан нь маш сайн эрсдэлийн удирдлагатай болохыг харуулна. Ихэнх мэргэжлийн сангууд 10--20\% уналтыг зөвшөөрдөг бол манай систем нь үүнээс хавьгүй бага байна.

\section{Гүйцэтгэлийн гүнзгий шинжилгээ}

\subsection{Эрсдэлийн хэмжүүрүүд}

Backtest-ийн гүйцэтгэлийн гол хэмжүүрүүдийг \tref{tab:risk_metrics} нд нэгтгэн харуулав.

\begin{table}[H]
\centering
\caption{Эрсдэлийн гол хэмжүүрүүд}
\label{tab:risk_metrics}
\begin{tabular}{lrll}
\toprule
\textbf{Хэмжүүр} & \textbf{Утга} & \textbf{Босго} & \textbf{Үнэлгээ} \\
\midrule
Profit Factor & 2.46 & $>$ 2.0 & Маш сайн \\
Sharpe Ratio & 9.64 & $>$ 3.0 & Онцгой \\
Recovery Factor & 6.69 & $>$ 3.0 & Сайн \\
Max Drawdown & 3.93\% & $<$ 10\% & Маш сайн \\
Win Rate & 44.44\% & $>$ 50\% & Хангалтгүй$^{*}$ \\
Нийт өгөөж & +41.61\% & $>$ 0\% & Маш сайн \\
Дундаж ашиг/арилжаа & \$92.47 & $>$ 0 & Сайн \\
\bottomrule
\end{tabular}
\par\smallskip
{\raggedright\scriptsize $^{*}$Win rate 50\%-аас бага боловч дундаж ашиг (\$351) нь дундаж алдагдлаас (\$114) 3 дахин их тул ашигтай.\par}
\end{table}

\subsection{Итгэлцэл ба нарийвчлалын хамаарал}

\fref{fig:confidence_accuracy} нь загварын confidence утга ба бодит нарийвчлалын хамаарлыг харуулав.

\begin{figure}[H]
\centering
\begin{tikzpicture}
\begin{axis}[
    width=0.85\textwidth,
    height=7cm,
    ybar,
    bar width=30pt,
    xlabel={Итгэлцлийн муж (Confidence)},
    ylabel={Нарийвчлал (\%)},
    xmin=0, xmax=5,
    ymin=50, ymax=105,
    xtick={1,2,3,4},
    xticklabels={0.85--0.90, 0.90--0.92, 0.92--0.95, 0.95+},
    xticklabel style={font=\small},
    grid=major,
    grid style={gray!20},
    legend pos=south east,
    legend style={font=\small, legend columns=1, column sep=4pt,
        /tikz/every even column/.append style={column sep=0pt}},
]
\addplot[fill=blue!50, draw=blue!70, nodes near coords, nodes near coords style={font=\small\bfseries, above}] coordinates {
    (1,72) (2,84) (3,91) (4,97)
};
\addlegendentry{Calibrated нарийвчлал}
\addplot[dashed, red, ultra thick, sharp plot, forget plot] coordinates {(0.5,90) (4.5,90)};
\addlegendimage{line legend, dashed, red, ultra thick}
\addlegendentry{Зорилтот: 90\%}
\end{axis}
\end{tikzpicture}
\caption{Итгэлцлийн утга ба таамаглалын нарийвчлалын хамаарал}
\label{fig:confidence_accuracy}
\end{figure}

Зурагнаас харахад confidence утга нэмэгдэх тусам нарийвчлал мөн нэмэгддэг нь загварын calibration зөв ажиллаж байгааг баталж байна. 0.92--0.95 мужид 91\%, 0.95-аас дээш 97\% нарийвчлалтай байна.

\subsection{Техник индикаторын ач холбогдол (Feature Importance)}

\fref{fig:feature_importance} нь загварт хамгийн их нөлөөлсөн техник индикаторуудыг харуулав.

\begin{figure}[H]
\centering
\begin{tikzpicture}
\begin{axis}[
    width=0.85\textwidth,
    height=10cm,
    xbar,
    bar width=8pt,
    xlabel={Ач холбогдлын оноо},
    xmin=0, xmax=110,
    ytick=data,
    yticklabels={
        returns\_1min,
        ma\_5\_5min,
        volatility\_5min,
        close\_1min,
        ma\_20\_1min,
        rsi\_15min,
        ma\_5\_15min,
        volatility\_15min,
        atr\_30min,
        rsi\_30min,
        close\_30min,
        ma\_50\_30min,
        atr\_1H,
        rsi\_1H,
        ma\_20\_1H,
        close\_1H,
        atr\_4H,
        rsi\_4H,
        ma\_50\_4H,
        close\_4H
    },
    yticklabel style={font=\scriptsize\ttfamily},
    grid=major,
    grid style={gray!15},
    nodes near coords,
    nodes near coords style={font=\tiny, anchor=west},
    y dir=reverse,
]
\addplot[fill=blue!50, draw=blue!70] coordinates {
    (22,1) (25,2) (28,3) (30,4) (32,5)
    (35,6) (38,7) (40,8) (45,9) (48,10)
    (50,11) (55,12) (60,13) (65,14) (68,15)
    (72,16) (80,17) (85,18) (95,19) (100,20)
};
\end{axis}
\end{tikzpicture}
\caption{Техник индикаторын ач холбогдол (Top 20)}
\label{fig:feature_importance}
\end{figure}

Хамгийн чухал техник индикаторуудад:
\begin{itemize}
    \item \textbf{ATR} (хэлбэлзэл) -- бүх хугацааны хүрээнд чухал
    \item \textbf{RSI} -- моментумын дохио
    \item \textbf{MA} (хөдөлгөөнт дундаж) -- чиг хандлагын тодорхойлолт
    \item \textbf{Close price} -- үнийн түвшин
\end{itemize}

Олон хугацааны хүрээний индикаторууд (H1, H4) нь M1-ээс илүү ач холбогдолтой байв -- энэ нь ``том зургийг'' (big picture) авч үзэх нь чухал гэдгийг батална.

\section{Хөгжүүлэлтийн үе шатуудын харьцуулалт}

\subsection{Phase 6 ба Phase 7-ийн харьцуулалт}

\fref{fig:phase_comparison} нь хоёр үе шатын гүйцэтгэлийг харьцуулав.

\begin{figure}[H]
\centering
\begin{tikzpicture}
\begin{axis}[
    width=0.95\textwidth,
    height=7cm,
    ybar,
    bar width=16pt,
    ylabel={Утга},
    symbolic x coords={Ашиг (\$), PF, Sharpe, DD (\%), Арилжаа},
    xtick=data,
    xticklabel style={font=\small},
    ymin=0, ymax=200,
    grid=major,
    grid style={gray!15},
    legend pos=north west,
    legend style={font=\small},
    nodes near coords,
    nodes near coords style={font=\tiny, above},
    enlarge x limits=0.15,
]
\addplot[fill=red!40, draw=red!60] coordinates {
    (Ашиг (\$),76) (PF,18) (Sharpe,45) (DD (\%),90) (Арилжаа,121)
};
\addlegendentry{Phase 6}
\addplot[fill=green!50, draw=green!70] coordinates {
    (Ашиг (\$),42) (PF,25) (Sharpe,96) (DD (\%),39) (Арилжаа,45)
};
\addlegendentry{Phase 7}
\end{axis}
\end{tikzpicture}
\caption{Phase 6 ба Phase 7-ийн харьцуулалт}
\label{fig:phase_comparison}
\end{figure}

\begin{table}[H]
\centering
\caption{Phase 6 ба Phase 7-ийн харьцуулалт}
\label{tab:phase_comparison}
\begin{tabular}{lccc}
\toprule
\textbf{Хэмжүүр} & \textbf{Phase 6} & \textbf{Phase 7} & \textbf{Сайжруулалт} \\
\midrule
Нийт дохио & 3,991 & 1,065 & -73\% \\
Нийт арилжаа & 121 & 45 & -63\% \\
Өгөөж & +76.46\% & +41.61\% & Чанар $\uparrow$ \\
Win Rate & 37.19\% & 44.44\% & +19\% \\
Profit Factor & 1.80 & 2.46 & +37\% \\
Max Drawdown & 9.0\% & 3.93\% & -56\% \\
Sharpe Ratio & 4.50 & 9.64 & +114\% \\
\bottomrule
\end{tabular}
\end{table}

Phase 7 нь Phase 6-тэй харьцуулахад эрсдэлийн бүх хэмжүүрээр сайжирсан. Phase 6-д 3,991 дохионоос 121 арилжаа хийж 76.46\% өгөөжтэй байсан ч win rate (37.19\%) бага, drawdown (9\%) өндөр байв. Ялангуяа:
\begin{itemize}
    \item Дохионы чанарыг нэмэгдүүлж (3,991$\to$1,065) арилжааны тоог цөөлсөн ч \textbf{чанар эрс сайжирсан}
    \item Win Rate 37.19\%$\to$44.44\%, Drawdown 56\%-аар буурсан, Sharpe 114\%-аар өссөн
    \item Phase 6-ийн өндөр өгөөж нь олон арилжаатай, өндөр эрсдэлтэй байсан бол Phase 7 нь бага эрсдэлтэй, тогтвортой
\end{itemize}

\fref{fig:phase_comparison_table} нь хоёр үе шатыг хүснэгтэн хэлбэрээр харьцуулав.

\begin{figure}[H]
\centering
\begin{tikzpicture}
\footnotesize
\def\cellw{2.5cm}
\def\cellh{0.7cm}
% Header row
\node[draw, fill=blue!20, minimum width=3cm, minimum height=\cellh, font=\bfseries] at (0,0) {Хэмжүүр};
\node[draw, fill=red!15, minimum width=\cellw, minimum height=\cellh, font=\bfseries] at (3,0) {Phase 6};
\node[draw, fill=green!15, minimum width=\cellw, minimum height=\cellh, font=\bfseries] at (5.7,0) {Phase 7};
\node[draw, fill=yellow!15, minimum width=\cellw, minimum height=\cellh, font=\bfseries] at (8.4,0) {Өөрчлөлт};
% Data rows
\foreach \y/\metric/\old/\new/\change in {
    -0.7/Нийт дохио/3{,}991/1{,}065/{\color{green!50!black}$-$73\%},
    -1.4/Нийт арилжаа/121/45/{\color{green!50!black}$-$63\%},
    -2.1/Өгөөж/+76.46\%/+41.61\%/{\color{blue}Чанар $\uparrow$},
    -2.8/Win Rate/37.19\%/44.44\%/{\color{green!50!black}+19\%},
    -3.5/Profit Factor/1.80/2.46/{\color{green!50!black}+37\%},
    -4.2/Max Drawdown/9.0\%/3.93\%/{\color{green!50!black}$-$56\%},
    -4.9/Sharpe Ratio/4.50/9.64/{\color{green!50!black}+114\%}
} {
    \node[draw, minimum width=3cm, minimum height=\cellh] at (0,\y) {\metric};
    \node[draw, minimum width=\cellw, minimum height=\cellh] at (3,\y) {\old};
    \node[draw, minimum width=\cellw, minimum height=\cellh, font=\bfseries] at (5.7,\y) {\new};
    \node[draw, minimum width=\cellw, minimum height=\cellh] at (8.4,\y) {\change};
}
\end{tikzpicture}
\caption{Phase 6 ба Phase 7 харьцуулалтын хүснэгт}
\label{fig:phase_comparison_table}
\end{figure}

\subsection{Бенчмарк харьцуулалт}

Системийн гүйцэтгэлийг бусад бенчмарктай харьцуулав:

\begin{table}[H]
\centering
\caption{Бенчмарк харьцуулалт}
\label{tab:benchmark}
\begin{tabular}{lcccc}
\toprule
\textbf{Систем} & \textbf{Өгөөж} & \textbf{Sharpe} & \textbf{Max DD} & \textbf{Win Rate} \\
\midrule
\textbf{Манай систем (Phase 7)} & \textbf{+41.61\%} & \textbf{9.64} & \textbf{3.93\%} & \textbf{44.44\%} \\
S\&P 500 (2025 дундаж) & +12\% & 0.8--1.2 & 10--15\% & -- \\
Хедж сан (дундаж) & +8--15\% & 1.0--2.0 & 10--20\% & -- \\
Жижиглэн арилжаачид & -5--+10\% & <1.0 & 20--40\% & 30--40\% \\
\bottomrule
\end{tabular}
\end{table}

Системийн гүйцэтгэл бүх бенчмаркаас тод давуу байна. Ялангуяа Sharpe Ratio (9.64) нь хедж сангийн түвшнээс хавьгүй дээгүүр.

\section{Overfitting шинжилгээ}

Хөгжүүлэлтийн явцад overfitting нь нэн чухал сорилт байв. Phase 6-д загвар сургалтын өгөгдөлд маш сайн, тестэд маш муу (win rate 15\%) ажилласан. Энэ асуудлыг шийдсэн арга хэмжээнүүд:

\begin{enumerate}
    \item \textbf{Индикатор хялбаршуулалт}: 75$\to$48 -- нарийн, чимээ шуугиан бүхий индикаторыг хассан
    \item \textbf{Загвар хялбаршуулалт}: 9$\to$3 загвар -- олон төрлийн загварын оронд найдвартай цөөн загвар
    \item \textbf{Walk-forward validation}: Ирээдүйн өгөгдөл сургалтад алдагдахгүй
    \item \textbf{Гиперпараметрийн хязгаарлалт}: Бага гүн (5--6), бага сургалтын хурд (0.03)
    \item \textbf{Итгэлцлийн шүүлтүүр}: conf $\geq$ 0.90 -- зөвхөн маш итгэлтэй таамгийг ашиглах
\end{enumerate}

Overfitting шалгалтын гол шалгуур нь Train accuracy < Test accuracy байх ёстой бөгөөд 77.4\% < 87.4\% гэсэн үр дүн нь энэ шалгуурыг хангаж байна.

\section{Мобайл аппликейшний ажиллагаа}

Мобайл аппликейшн (Predictrix) нь бодит цагийн горимд ажиллаж, дараахь функциудыг гүйцэтгэнэ:

\begin{enumerate}
    \item \textbf{20 валютын хослолд бодит ханш}: Yahoo Finance API-аас 60 секунд тутам шинэчлэгддэг
    \item \textbf{ML дохио}: GBDT ансамбль (Phase 7) загварын BUY/SELL/HOLD дохио, итгэлцлийн хувь, SL/TP
    \item \textbf{Эдийн засгийн мэдээ}: TradingView-ийн эдийн засгийн хуанли, Alpha Vantage мэдээ
    \item \textbf{AI дүн шинжилгээ}: Google Gemini API ашигласан зах зээлийн шинжилгээ
    \item \textbf{Хэрэглэгчийн систем}: Бүртгэл, нэвтрэлт, имэйл баталгаажуулалт, нууц үг сэргээх
    \item \textbf{Дохионы түүх}: MongoDB-д хадгалагдсан арилжааны дохионы түүх ба статистик
\end{enumerate}
