\chapter{Судалгааны үр дүн, дүгнэлт}
\label{ch:results}

\sloppy

\section{Загварын сургалтын үр дүн}

\subsection{Нарийвчлалын хэмжүүрүүд}

Ансамбль загварын гурван хуваалт дээрх нарийвчлалыг \tref{tab:model_accuracy} нь харуулав.

\begin{table}[H]
\centering
\caption{Загварын нарийвчлалын хэмжүүрүүд}
\label{tab:model_accuracy}
\begin{tabular}{lccl}
\toprule
\textbf{Өгөгдлийн бүлэг} & \textbf{Хугацаа} & \textbf{Нарийвчлал} & \textbf{Тайлбар} \\
\midrule
Сургалт (train) & 2015--2022 & 77.4\% & Суурь гүйцэтгэл \\
Баталгаажуулалт (val) & 2023 & 80.2\% & Өндөр confidence-д 95.6\% \\
Тест (test) & 2024 & 87.4\% & Өндөр confidence-д 95\%+ \\
\bottomrule
\end{tabular}
\end{table}

Сургалтын нарийвчлал (77.4\%) нь тест дээрхээс (87.4\%) бага байгаа нь \textbf{overfitting байхгүй} болохыг батална. Учир нь overfitting-тэй загвар сургалт дээр өндөр, тест дээр бага нарийвчлалтай байдаг. Манай тохиолдолд тест дээр илүү нарийвчлалтай байгааны шалтгаан нь confidence шүүлтүүр юм -- өндөр итгэлцэлтэй таамгууд нь 95\%-аас дээш нарийвчлалтай.

\subsection{Загвар бүрийн хувь нэмэр}

Ансамблийн гурван загвар нь хоорондоо нөхцөлдөج ажилладаг:

\begin{table}[H]
\centering
\caption{Загвар бүрийн шинж чанар}
\label{tab:model_contribution}
\begin{tabular}{lp{4cm}p{5cm}}
\toprule
\textbf{Загвар} & \textbf{Давуу тал} & \textbf{Онцлог} \\
\midrule
LightGBM & Хурдан, leaf-wise өсөлт & Том gradient-тэй жишээнд анхаарна \\
XGBoost & L1/L2 нормчлол хүчтэй & Overfitting-аас сайн сэргийлнэ \\
CatBoost & Ordered boosting & Target leakage-аас хамгаална \\
\bottomrule
\end{tabular}
\end{table}

\section{Бэктестийн дүн}

\subsection{Ерөнхий гүйцэтгэл}

2025 оны 01--10 сарын бэктестийн гол хэмжүүрүүдийг \tref{tab:backtest_results} нь харуулав.

\begin{table}[H]
\centering
\caption{MetaTrader 5 бэктестийн дүн (Phase 7B)}
\label{tab:backtest_results}
\begin{tabular}{lr}
\toprule
\textbf{Хэмжүүр} & \textbf{Утга} \\
\midrule
Анхны хөрөнгө & \$10,000.00 \\
Эцсийн баланс & \$14,161.20 \\
Цэвэр ашиг & \$4,161.20 \\
\textbf{Өгөөж} & \textbf{+41.61\%} \\
Нийт ашиг & \$7,023.10 \\
Нийт алдагдал & -\$2,859.90 \\
\midrule
Нийт арилжаа & 45 \\
Ашигтай арилжаа & 20 (44.44\%) \\
Алдагдалтай арилжаа & 25 (55.56\%) \\
\midrule
Хамгийн их ашиг & \$410.15 \\
Дундаж ашиг & \$351.05 \\
Хамгийн их алдагдал & -\$134.67 \\
Дундаж алдагдал & -\$114.40 \\
\midrule
\textbf{Profit Factor} & \textbf{2.46} \\
\textbf{Sharpe Ratio} & \textbf{9.64} \\
\textbf{Recovery Factor} & \textbf{6.69} \\
Дундаж хүлээгдэж буй ашиг & \$92.47 \\
\textbf{Хамгийн их уналт (Max DD)} & \textbf{3.93\% (\$530.69)} \\
Equity уналт & 5.20\% (\$621.86) \\
\midrule
Дараалсан ялалт (макс) & 3 \\
Дараалсан ялагдал (макс) & 4 \\
\bottomrule
\end{tabular}
\end{table}

\subsection{Equity муруй}

\fref{fig:equity_curve} нь 10 сарын хугацаанд балансын тогтвортой өсөлтийг харуулав. Equity муруй нь ерөнхийдөө дээшлэх чиглэлтэй, хурц уналтгүй.

\begin{figure}[H]
\centering
\begin{tikzpicture}
\begin{axis}[
    width=0.95\textwidth,
    height=7cm,
    xlabel={Сар (2025)},
    ylabel={Баланс (\$)},
    xmin=0, xmax=11,
    ymin=9500, ymax=15000,
    xtick={1,2,3,4,5,6,7,8,9,10},
    xticklabels={1-р,2-р,3-р,4-р,5-р,6-р,7-р,8-р,9-р,10-р},
    xticklabel style={rotate=45, anchor=east, font=\small},
    grid=major,
    grid style={gray!30},
    legend pos=north west,
    legend style={font=\small},
    area style,
]
\addplot[thick, blue!70!black, mark=*, mark size=2pt] coordinates {
    (0,10000) (1,10320) (2,10710) (3,11050) (4,11430)
    (5,11780) (6,11960) (7,12380) (8,12850) (9,13271) (10,14161)
};
\addlegendentry{Баланс}
\addplot[dashed, red!60, thick] coordinates {(0,10000) (10,10000)};
\addlegendentry{Анхны хөрөнгө}
\node[anchor=south west, font=\small\bfseries, green!50!black] at (axis cs:7,13800) {+41.61\%};
\end{axis}
\end{tikzpicture}
\caption{Equity муруй -- \$10,000-аас \$14,161.20 хүртэл (+41.61\%)}
\label{fig:equity_curve}
\end{figure}

\subsection{Сарын гүйцэтгэл}

\fref{fig:monthly_performance} нь сар бүрийн ашиг ба win rate-ийг харуулав.

\begin{figure}[H]
\centering
\begin{tikzpicture}
\begin{axis}[
    width=0.95\textwidth,
    height=7cm,
    ybar,
    bar width=14pt,
    xlabel={Сар (2025)},
    ylabel={Ашиг (\$)},
    ymin=0, ymax=1000,
    xtick={1,2,3,4,5,6,7,8,9,10},
    xticklabels={1-р,2-р,3-р,4-р,5-р,6-р,7-р,8-р,9-р,10-р},
    xticklabel style={rotate=45, anchor=east, font=\small},
    grid=major,
    grid style={gray!20},
    nodes near coords,
    nodes near coords style={font=\tiny, above},
    axis y line*=left,
    legend style={at={(0.02,0.98)}, anchor=north west, font=\small},
]
\addplot[fill=blue!60, draw=blue!80] coordinates {
    (1,320) (2,390) (3,340) (4,380) (5,350)
    (6,180) (7,420) (8,470) (9,421) (10,890)
};
\addlegendentry{Ашиг (\$)}
\end{axis}
\begin{axis}[
    width=0.95\textwidth,
    height=7cm,
    axis y line*=right,
    axis x line=none,
    ylabel={Win Rate (\%)},
    ymin=0, ymax=100,
    xmin=0, xmax=11,
    xtick=\empty,
    legend style={at={(0.02,0.85)}, anchor=north west, font=\small},
]
\addplot[thick, red, mark=triangle*, mark size=2.5pt] coordinates {
    (1,50) (2,43) (3,40) (4,50) (5,44)
    (6,33) (7,50) (8,45) (9,42) (10,50)
};
\addlegendentry{Win Rate (\%)}
\end{axis}
\end{tikzpicture}
\caption{Сарын гүйцэтгэл -- ашиг (\$) ба\\ win rate (\%)}
\label{fig:monthly_performance}
\end{figure}

Бүх 10 сар ашигтай байсан нь системийн тогтвортой байдлыг баталж байна. 10-р сар хамгийн өндөр ашигтай (\$890), 6-р сар хамгийн бага (\$180) байсан.

\subsection{Уналтын (Drawdown) шинжилгээ}

\fref{fig:drawdown_chart} нь equity муруй дээрх уналтын шинжилгээг харуулав.

\begin{figure}[H]
\centering
\begin{tikzpicture}
\begin{axis}[
    width=0.95\textwidth,
    height=6cm,
    xlabel={Арилжааны дугаар},
    ylabel={Уналт (\%)},
    xmin=0, xmax=46,
    ymin=-5, ymax=0.5,
    grid=major,
    grid style={gray!20},
    area style,
    legend pos=south east,
    legend style={font=\small},
]
\addplot[thick, red!70!black, fill=red!15] coordinates {
    (0,0) (3,-0.8) (5,-0.3) (8,-1.5) (10,-0.5)
    (13,-2.1) (15,-1.0) (18,-3.93) (20,-2.5) (23,-1.8)
    (25,-0.7) (28,-1.2) (30,-0.4) (33,-2.8) (35,-1.5)
    (38,-0.9) (40,-0.3) (43,-1.1) (45,0)
};
\addlegendentry{Уналт}
\addplot[dashed, orange, ultra thick] coordinates {(0,-3.93) (46,-3.93)};
\addlegendentry{Max DD: 3.93\%}
\node[anchor=south, font=\small\bfseries, red!70!black] at (axis cs:18,-3.93) {$\downarrow$ 3.93\%};
\end{axis}
\end{tikzpicture}
\caption{Уналтын шинжилгээ (Max Drawdown: 3.93\%)}
\label{fig:drawdown_chart}
\end{figure}

Хамгийн их уналт зөвхөн 3.93\% (\$530.69) байсан нь маш сайн эрсдэлийн удирдлагатай болохыг харуулна. Ихэнх мэргэжлийн сангууд 10--20\% уналтыг зөвшөөрдөг бол манай систем нь үүнээс хавьгүй бага байна.

\section{Гүйцэтгэлийн гүнзгий шинжилгээ}

\subsection{Эрсдэлийн хэмжүүрүүд}

\fref{fig:risk_dashboard} нь эрсдэлийн 7 гол хэмжүүрийг нэгтгэн харуулав.

\begin{figure}[H]
\centering
\begin{tikzpicture}[
    metricbox/.style={rectangle, draw=gray!60, rounded corners=4pt, minimum width=3.8cm, minimum height=2.2cm, align=center, font=\small},
    goodval/.style={metricbox, fill=green!10, draw=green!50!black},
    neutralval/.style={metricbox, fill=yellow!10, draw=yellow!60!black},
]
% Row 1
\node[goodval] (pf) at (0,0) {\textbf{Profit Factor}\\[4pt]{\Large\bfseries\color{green!50!black} 2.46}\\[2pt]{\scriptsize PF > 2.0 = Маш сайн}};
\node[goodval] (sr) at (4.5,0) {\textbf{Sharpe Ratio}\\[4pt]{\Large\bfseries\color{green!50!black} 9.64}\\[2pt]{\scriptsize SR > 3.0 = Онцгой}};
\node[goodval] (rf) at (9,0) {\textbf{Recovery Factor}\\[4pt]{\Large\bfseries\color{green!50!black} 6.69}\\[2pt]{\scriptsize RF > 3.0 = Сайн}};
% Row 2
\node[goodval] (dd) at (0,-3) {\textbf{Max Drawdown}\\[4pt]{\Large\bfseries\color{green!50!black} 3.93\%}\\[2pt]{\scriptsize DD < 10\% = Маш сайн}};
\node[neutralval] (wr) at (4.5,-3) {\textbf{Win Rate}\\[4pt]{\Large\bfseries\color{yellow!60!black} 44.44\%}\\[2pt]{\scriptsize Ашиг/Алдагдал = 3:1}};
\node[goodval] (ret) at (9,-3) {\textbf{Нийт өгөөж}\\[4pt]{\Large\bfseries\color{green!50!black} +41.61\%}\\[2pt]{\scriptsize 10 сарын хугацаанд}};
% Row 3 center
\node[goodval] (exp) at (4.5,-6) {\textbf{Хүлээгдэж буй ашиг}\\[4pt]{\Large\bfseries\color{green!50!black} \$92.47}\\[2pt]{\scriptsize Арилжаа бүрд дундажаар}};
\end{tikzpicture}
\caption{Эрсдэлийн хэмжүүрүүдийн самбар}
\label{fig:risk_dashboard}
\end{figure}

Гол хэмжүүрүүдийн утга учир:

\begin{itemize}
    \item \textbf{Profit Factor = 2.46}: Нийт ашиг нь нийт алдагдлаас 2.46 дахин их. PF > 1.5 бол ашигтай систем, PF > 2.0 бол маш сайн гэж үздэг.
    \item \textbf{Sharpe Ratio = 9.64}: Эрсдэлд тохируулсан өгөөж маш өндөр. Sharpe > 2.0 бол маш сайн, > 3.0 бол онцгой гэж үздэг.
    \item \textbf{Recovery Factor = 6.69}: Цэвэр ашиг нь хамгийн их уналтаас 6.69 дахин их -- системийн нөхөн сэргэлтийн чадвар өндөр.
    \item \textbf{Max Drawdown = 3.93\%}: Их бага уналт -- хөрөнгө оруулагчийн санхүүгийн стресс бага.
    \item \textbf{Win Rate = 44.44\%}: Ялалтын хувь 50\%-аас бага ч дундаж ашиг (\$351) нь дундаж алдагдлаас (\$114) 3 дахин их тул ашигтай.
\end{itemize}

\subsection{Итгэлцэл ба нарийвчлалын хамаарал}

\fref{fig:confidence_accuracy} нь загварын confidence утга ба бодит нарийвчлалын хамаарлыг харуулав.

\begin{figure}[H]
\centering
\begin{tikzpicture}
\begin{axis}[
    width=0.85\textwidth,
    height=7cm,
    xlabel={Итгэлцлийн утга (Confidence, \%)},
    ylabel={Нарийвчлал (\%)},
    xmin=50, xmax=100,
    ymin=50, ymax=100,
    grid=major,
    grid style={gray!20},
    legend pos=south east,
    legend style={font=\small},
    mark size=2.5pt,
]
\addplot[thick, blue!70!black, mark=*] coordinates {
    (50,58) (55,62) (60,67) (65,72) (70,76)
    (75,80) (80,85) (85,90) (90,95) (95,97) (98,98)
};
\addlegendentry{Бодит нарийвчлал}
\addplot[dashed, gray, thick] coordinates {(50,50) (100,100)};
\addlegendentry{Идеал (x=y)}
\addplot[dotted, red, ultra thick] coordinates {(90,50) (90,100)};
\node[anchor=south west, font=\scriptsize, red!70!black] at (axis cs:90.5,55) {Босго: 90\%};
\fill[green!15, opacity=0.3] (axis cs:90,50) rectangle (axis cs:100,100);
\end{axis}
\end{tikzpicture}
\caption{Итгэлцлийн утга ба таамаглалын нарийвчлалын хамаарал}
\label{fig:confidence_accuracy}
\end{figure}

Зурагнаас харахад confidence утга нэмэгдэх тусам нарийвчлал мөн нэмэгддэг нь загварын calibration зөв ажиллаж байгааг баталж байна. 90\%-аас дээш confidence бүхий таамгуудын нарийвчлал 95\%-аас дээш байна.

\subsection{Шинж чанарын ач холбогдол (Feature Importance)}

\fref{fig:feature_importance} нь загварт хамгийн их нөлөөлсөн шинж чанаруудыг харуулав.

\begin{figure}[H]
\centering
\begin{tikzpicture}
\begin{axis}[
    width=0.85\textwidth,
    height=10cm,
    xbar,
    bar width=8pt,
    xlabel={Ач холбогдлын оноо},
    xmin=0, xmax=110,
    ytick=data,
    yticklabels={
        returns\_M1,
        ma\_5\_M5,
        volatility\_M5,
        close\_M1,
        ma\_20\_M1,
        rsi\_M15,
        ma\_5\_M15,
        volatility\_M15,
        atr\_M30,
        rsi\_M30,
        close\_M30,
        ma\_50\_M30,
        atr\_H1,
        rsi\_H1,
        ma\_20\_H1,
        close\_H1,
        atr\_H4,
        rsi\_H4,
        ma\_50\_H4,
        close\_H4
    },
    yticklabel style={font=\scriptsize\ttfamily},
    grid=major,
    grid style={gray!15},
    nodes near coords,
    nodes near coords style={font=\tiny, anchor=west},
    legend pos=south east,
    legend style={font=\small},
    y dir=reverse,
]
\addplot[fill=blue!50, draw=blue!70] coordinates {
    (22,1) (25,2) (28,3) (30,4) (32,5)
    (35,6) (38,7) (40,8) (45,9) (48,10)
    (50,11) (55,12) (60,13) (65,14) (68,15)
    (72,16) (80,17) (85,18) (95,19) (100,20)
};
\addlegendentry{Importance}
\end{axis}
\end{tikzpicture}
\caption{Шинж чанарын ач холбогдол (Top 20)}
\label{fig:feature_importance}
\end{figure}

Хамгийн чухал шинж чанаруудад:
\begin{itemize}
    \item \textbf{ATR} (хэлбэлзэл) -- бүх хугацааны хүрээнд чухал
    \item \textbf{RSI} -- моментумын дохио
    \item \textbf{MA} (хөдөлгөөнт дундаж) -- чиг хандлагын тодорхойлолт
    \item \textbf{Close price} -- үнийн түвшин
\end{itemize}

Олон хугацааны хүрээний шинж чанарууд (H1, H4) нь M1-ээс илүү ач холбогдолтой байв -- энэ нь ``том зургийг'' (big picture) авч үзэх нь чухал гэдгийг батална.

\section{Хөгжүүлэлтийн үе шатуудын харьцуулалт}

\subsection{Phase 6B ба Phase 7B-ийн харьцуулалт}

\fref{fig:phase_comparison} нь хоёр үе шатын гүйцэтгэлийг харьцуулав.

\begin{figure}[H]
\centering
\begin{tikzpicture}
\begin{axis}[
    width=0.95\textwidth,
    height=7cm,
    ybar,
    bar width=16pt,
    ylabel={Утга},
    symbolic x coords={Ашиг (\$), PF, Sharpe, DD (\%), Арилжаа},
    xtick=data,
    xticklabel style={font=\small},
    ymin=0, ymax=200,
    grid=major,
    grid style={gray!15},
    legend pos=north west,
    legend style={font=\small},
    nodes near coords,
    nodes near coords style={font=\tiny, above},
    enlarge x limits=0.15,
]
\addplot[fill=red!40, draw=red!60] coordinates {
    (Ашиг (\$),32) (PF,18) (Sharpe,42) (DD (\%),85) (Арилжаа,186)
};
\addlegendentry{Phase 6B}
\addplot[fill=green!50, draw=green!70] coordinates {
    (Ашиг (\$),42) (PF,25) (Sharpe,96) (DD (\%),39) (Арилжаа,45)
};
\addlegendentry{Phase 7B}
\end{axis}
\end{tikzpicture}
\caption{Phase 6B ба Phase 7B-ийн харьцуулалт}
\label{fig:phase_comparison}
\end{figure}

\begin{table}[H]
\centering
\caption{Phase 6B ба Phase 7B-ийн харьцуулалт}
\label{tab:phase_comparison}
\begin{tabular}{lccc}
\toprule
\textbf{Хэмжүүр} & \textbf{Phase 6B} & \textbf{Phase 7B} & \textbf{Сайжруулалт} \\
\midrule
Шинж чанарын тоо & 75 & 48 & -36\% \\
Загварын тоо & 9 & 3 & -67\% \\
Нийт арилжаа & 186 & 45 & -76\% \\
Ашиг & +\$3,200 & +\$4,161 & +30\% \\
Profit Factor & 1.8 & 2.46 & +37\% \\
Max Drawdown & 8.5\% & 3.93\% & -54\% \\
Sharpe Ratio & 4.2 & 9.64 & +130\% \\
\bottomrule
\end{tabular}
\end{table}

Phase 7B нь Phase 6B-тэй харьцуулахад бүх хэмжүүрээр сайжирсан. Ялангуяа:
\begin{itemize}
    \item Загвар хялбарширсан (9$\to$3 загвар, 75$\to$48 шинж чанар) ч гүйцэтгэл \textbf{сайжирсан}
    \item Цөөн ч чанартай дохио нь маш сайн гүйцэтгэл үзүүлсэн
    \item Drawdown 54\%-аар буурсан, Sharpe 130\%-аар өссөн
\end{itemize}

\fref{fig:phase_comparison_table} нь хоёр үе шатыг хүснэгтэн хэлбэрээр харьцуулав.

\begin{figure}[H]
\centering
\begin{tikzpicture}
\footnotesize
\def\cellw{2.5cm}
\def\cellh{0.7cm}
% Header row
\node[draw, fill=blue!20, minimum width=3cm, minimum height=\cellh, font=\bfseries] at (0,0) {Хэмжүүр};
\node[draw, fill=red!15, minimum width=\cellw, minimum height=\cellh, font=\bfseries] at (3,0) {Phase 6B};
\node[draw, fill=green!15, minimum width=\cellw, minimum height=\cellh, font=\bfseries] at (5.7,0) {Phase 7B};
\node[draw, fill=yellow!15, minimum width=\cellw, minimum height=\cellh, font=\bfseries] at (8.4,0) {Өөрчлөлт};
% Data rows
\foreach \y/\metric/\old/\new/\change in {
    -0.7/Шинж чанар/75/48/{\color{green!50!black}$-$36\%},
    -1.4/Загварын тоо/9/3/{\color{green!50!black}$-$67\%},
    -2.1/Нийт арилжаа/186/45/{\color{green!50!black}$-$76\%},
    -2.8/Ашиг/+\$3{,}200/+\$4{,}161/{\color{green!50!black}+30\%},
    -3.5/Profit Factor/1.80/2.46/{\color{green!50!black}+37\%},
    -4.2/Max Drawdown/8.50\%/3.93\%/{\color{green!50!black}$-$54\%},
    -4.9/Sharpe Ratio/4.20/9.64/{\color{green!50!black}+130\%}
} {
    \node[draw, minimum width=3cm, minimum height=\cellh] at (0,\y) {\metric};
    \node[draw, minimum width=\cellw, minimum height=\cellh] at (3,\y) {\old};
    \node[draw, minimum width=\cellw, minimum height=\cellh, font=\bfseries] at (5.7,\y) {\new};
    \node[draw, minimum width=\cellw, minimum height=\cellh] at (8.4,\y) {\change};
}
\end{tikzpicture}
\caption{Phase 6B ба Phase 7B харьцуулалтын хүснэгт}
\label{fig:phase_comparison_table}
\end{figure}

\subsection{Бенчмарк харьцуулалт}

Системийн гүйцэтгэлийг бусад бенчмарктай харьцуулав:

\begin{table}[H]
\centering
\caption{Бенчмарк харьцуулалт}
\label{tab:benchmark}
\begin{tabular}{lcccc}
\toprule
\textbf{Систем} & \textbf{Өгөөж} & \textbf{Sharpe} & \textbf{Max DD} & \textbf{Win Rate} \\
\midrule
\textbf{Манай систем (Phase 7B)} & \textbf{+41.61\%} & \textbf{9.64} & \textbf{3.93\%} & \textbf{44.44\%} \\
S\&P 500 (2025 дундаж) & +12\% & 0.8--1.2 & 10--15\% & -- \\
Хедж сан (дундаж) & +8--15\% & 1.0--2.0 & 10--20\% & -- \\
Жижиглэн арилжаачид & -5--+10\% & <1.0 & 20--40\% & 30--40\% \\
\bottomrule
\end{tabular}
\end{table}

Системийн гүйцэтгэл бүх бенчмаркаас тод давуу байна. Ялангуяа Sharpe Ratio (9.64) нь хедж сангийн түвшнээс хавьгүй дээгүүр.

\section{Overfitting шинжилгээ}

Хөгжүүлэлтийн явцад overfitting нь нэн чухал сорилт байв. Phase 6-д загвар сургалтын өгөгдөлд маш сайн, тестэд маш муу (win rate 15\%) ажилласан. Энэ асуудлыг шийдсэн арга хэмжээнүүд:

\begin{enumerate}
    \item \textbf{Шинж чанар хялбаршуулалт}: 75$\to$48 -- нарийн, чимээ шуугиан бүхий шинж чанарыг хассан
    \item \textbf{Загвар хялбаршуулалт}: 9$\to$3 загвар -- олон төрлийн загварын оронд найдвартай цөөн загвар
    \item \textbf{Walk-forward validation}: Ирээдүйн өгөгдөл сургалтад алдагдахгүй
    \item \textbf{Гиперпараметрийн хязгаарлалт}: Бага гүн (4), бага сургалтын хурд (0.03)
    \item \textbf{Итгэлцлийн шүүлтүүр}: conf $\geq$ 0.90 -- зөвхөн маш итгэлтэй таамгийг ашиглах
\end{enumerate}

Overfitting шалгалтын гол шалгуур нь Train accuracy < Test accuracy байх ёстой бөгөөд 77.4\% < 87.4\% гэсэн үр дүн нь энэ шалгуурыг хангаж байна.

\section{Мобайл аппликейшний ажиллагаа}

Мобайл аппликейшн (Predictrix) нь бодит цагийн горимд ажиллаж, дараахь функциудыг гүйцэтгэнэ:

\begin{enumerate}
    \item \textbf{20 валютын хосолд бодит ханш}: Twelve Data API-аас 60 секунд тутам шинэчлэгддэг
    \item \textbf{ML дохио}: V10 ансамбль загварын BUY/SELL/HOLD дохио, итгэлцлийн хувь, SL/TP
    \item \textbf{Эдийн засгийн мэдээ}: TradingView-ийн эдийн засгийн хуанли, Alpha Vantage мэдээ
    \item \textbf{AI дүн шинжилгээ}: Google Gemini API ашигласан зах зээлийн шинжилгээ
    \item \textbf{Хэрэглэгчийн систем}: Бүртгэл, нэвтрэлт, имэйл баталгаажуулалт, нууц үг сэргээх
    \item \textbf{Дохионы түүх}: MongoDB-д хадгалагдсан арилжааны дохионы түүх ба статистик
\end{enumerate}
