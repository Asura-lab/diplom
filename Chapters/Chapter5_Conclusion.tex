% Дүгнэлт -- 5-р бүлэгийн үргэлжлэл (Судалгааны үр дүн, дүгнэлт)
\section{Дүгнэлт}
\label{sec:conclusion}

\subsection{Судалгааны гол үр дүн}

Энэхүү судалгааны ажлаар машин сургалтын ансамбль аргыг ашиглан Forex зах зээлийн арилжааны дохио үүсгэх бүрэн систем амжилттай хөгжүүлсэн. Гол үр дүнгүүд:

\begin{enumerate}
    \item \textbf{ML ансамбль загвар}: LightGBM, XGBoost, CatBoost гэсэн гурван GBDT загварын ансамбль нь 6 хугацааны хүрээний 48 шинж чанарыг ашиглан EUR/USD-ийн чиг хандлагыг амжилттай таамаглаж чадсан. Walk-forward validation-аар баталгаажуулсан тест дээрх нарийвчлал 87.4\%, өндөр итгэлцэлтэй дохионы нарийвчлал 95\%+ байсан.
    
    \item \textbf{Бэктестийн гүйцэтгэл}: 2025 оны 10 сарын бодит зах зээлийн бэктестэд:
    \begin{itemize}
        \item Өгөөж: +41.61\% (S\&P 500 дунджаас 3.5 дахин их)
        \item Profit Factor: 2.46 (мэргэжлийн системийн түвшин)
        \item Sharpe Ratio: 9.64 (хедж сангийн түвшнээс хавьгүй дээгүүр)
        \item Max Drawdown: 3.93\% (маш бага эрсдэлтэй)
    \end{itemize}
    
    \item \textbf{Мобайл аппликейшн}: React Native дээр суурилсан бүрэн функциональ ``Predictrix'' аппликейшнийг хөгжүүлж, бодит цагийн ханш, ML дохио, эдийн засгийн мэдээ, AI шинжилгээг нэг дор хүргэсэн.
    
    \item \textbf{Overfitting шийдвэрлэлт}: 7 давталтат хөгжүүлэлтийн үе шатаар загварыг хялбаршуулж (75$\to$48 шинж чанар, 9$\to$3 загвар), walk-forward validation-аар баталгаажуулж, overfitting-ийн асуудлыг бүрэн шийдсэн.
\end{enumerate}

\subsection{Судалгааны шинэлэг тал}

\begin{itemize}
    \item \textbf{Олон хугацааны хүрээний шинж чанар}: 6 хугацааны интервалаас (M1--H4) шинж чанар тооцоолсноор ``том зурагийг'' авч үзэх чадвартай загвар бүтээсэн
    \item \textbf{Calibrated confidence}: Logistic Regression calibrator-аар загварын магадлалын утгыг найдвартай итгэлцлийн хэмжүүр болгосон
    \item \textbf{Чанар $>$ тоо хэмжээ}: 359,639 таамгаас зөвхөн 1,065 (0.3\%) чанартай дохио шүүж илрүүлдэг стратеги
    \item \textbf{End-to-end систем}: Загварын сургалтаас эхлээд мобайл аппликейшн хүртэл бүрэн, нэгдмэл системийг хэрэгжүүлсэн
\end{itemize}

\subsection{Хязгаарлалт}

Судалгааны ажлын зарим хязгаарлалтуудыг тодорхойлох нь чухал:

\begin{enumerate}
    \item \textbf{Ганц валютын хослол}: Зөвхөн EUR/USD дээр сургагдсан -- бусад хослолд шууд ашиглах боломжгүй
    \item \textbf{Бэктест ба бодит арилжааны зөрүү}: Бэктест нь бодит арилжааны бүх нөхцлийг (жишээ нь төлбөрийн чадварын хягаарлалт, шуугиан) бүрэн дүрсэлж чаддаггүй
    \item \textbf{Зах зээлийн горимын өөрчлөлт}: Зах зээлийн бүтцэд суурь өөрчлөлт гарвал загварын гүйцэтгэл буурч болно
    \item \textbf{Арилжааны зардал}: Спрэд, слиппэж зэрэг зардлыг тооцсон ч комисс, своп зэргийг бүрэн тооцоогүй
    \item \textbf{Бодит цагийн хүндрэл}: API rate limit, сүлжээний саатал зэрэг техникийн асуудлууд
\end{enumerate}

\subsection{Цаашдын чиглэл}

Судалгааг цаашид дараахь чиглэлээр хөгжүүлэх боломжтой:

\begin{enumerate}
    \item \textbf{Олон валютын хослолд өргөтгөх}: GBP/USD, USD/JPY зэрэг бусад хослолд загварыг сургаж, портфолио стратеги бүтээх
    \item \textbf{Trailing Stop}: Нээлттэй позицын ашгийг хамгаалах динамик SL -- Phase 4-т туршсан ч цаашид сайжруулах
    \item \textbf{Гүн сургалтын загвар}: LSTM, Transformer зэрэг цуваа өгөгдлийн загваруудыг ансамбль-д нэмэх
    \item \textbf{Бататгалтат сургалт (Reinforcement Learning)}: Портфолиогийн менежмент, позицын хэмжээ тохируулалтад хэрэглэх
    \item \textbf{Мэдээний шинжилгээ}: NLP ашиглан эдийн засгийн мэдээг автоматаар шинжилж, загварт оруулах
    \item \textbf{Cloud deployment}: AWS/GCP дээр Backend-ийг байршуулж, App Store/Google Play дээр апп нийтлэх
\end{enumerate}

\subsection{Эцсийн дүгнэлт}

Машин сургалтын ансамбль загвар нь Forex зах зээл дээр ашигтай арилжааны дохио үүсгэх чадвартай болохыг энэхүү судалгаа бодитоор батлав. +41.61\% өгөөж, 9.64 Sharpe Ratio, 3.93\% Max Drawdown зэрэг хэмжүүрүүд нь мэргэжлийн хөрөнгө оруулалтын сангийн түвшний гүйцэтгэл юм. Гэхдээ санхүүгийн зах зээр дэх аливаа загвар бүрэн төгс биш бөгөөд эрсдэлийн менежмент, тогтмол дахин сургалт, шинэ нөхцөлд дасан зохицох чадвар нь тасралтгүй сайжруулалт шаарддаг.

Систем нь загвар сургалтаас эхлээд эцсийн хэрэглэгч хүртэлх бүрэн process-ийг амжилттай хэрэгжүүлсэн бөгөөд энэ нь машин сургалт, back-end хөгжүүлэлт, мобайл аппликейшний хөгжүүлэлт зэрэг олон салбарыг хамарсан цогц инженерийн ажил юм.
