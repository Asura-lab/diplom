\chapter{Удиртгал}
\label{ch:introduction}

\raggedright

Энэхүү судалгааны ажил нь машин сургалтын ансамбль аргуудыг ашиглан Forex валютын зах зээл дээр арилжааны дохио үүсгэх систем хөгжүүлэх, мөн хэрэглэгчдэд хүртээмжтэй гар утасны аппликейшн бүтээх зорилготой юм. Судалгааны ажил нь онол (машин сургалт, санхүүгийн дүн шинжилгээ), практик хэрэгжүүлэлт (загварын сургалт, бэктест), болон технологийн нэгтгэл (мобайл аппликейшн, backend систем)-ийг хамарсан цогц ажил юм.

Судалгааны гол бүрэлдэхүүн хэсгүүд нь:
\begin{itemize}
    \item \textbf{ML загварын ансамбль}: LightGBM, XGBoost, CatBoost -- гурван GBDT загварыг нэгтгэсэн систем
    \item \textbf{Walk-forward validation}: Overfitting-аас сэргийлэх цаг хугацаанд суурилсан баталгаажуулалт
    \item \textbf{Олон хугацааны дүн шинжилгээ}: M1--H4 хүртэлх 6 интервалын 48 техник шинж чанар
    \item \textbf{Бэктест}: MetaTrader 5 Strategy Tester дээрх бодит зах зээлийн нөхцөлд шалгалт
    \item \textbf{``Predictrix'' аппликейшн}: React Native дээр хөгжүүлсэн бодит цагийн мэдээлэл бүхий гар утасны програм
\end{itemize}

\section{Ажлын бүтэц ба агуулга}

Судалгааны ажил нь дараах бүлгүүдээс бүрдэнэ:

\textbf{2-р бүлэг -- Онолын үндэслэл, холбогдох судалгаанууд}

Энэ бүлэгт валютын зах зээлийн онцлог, техник дүн шинжилгээний үндсэн ойлголтууд (индикаторууд, график загварууд), машин сургалтын аргууд (ялангуяа Gradient Boosted Decision Trees), мөн холбогдох судалгаануудын тойм танилцуулагдана. Онолын суурийг бий болгож, энэхүү судалгааг бусад судалгаануудтай харьцуулан байршуулна.

\textbf{3-р бүлэг -- Арга зүй}

Судалгааны үндсэн техник хэрэгжүүлэлтийг дэлгэрүүлсэн бүлэг. Өгөгдлийн цуглуулалт, цэвэрлэлт, шинж чанарын инженерчлэл (48 техник индикатор, олон хугацааны интервалын мэдээлэл), гурван GBDT загварын сургалт, walk-forward validation аргачлал, ансамблын шийдвэр гаргах механизм, мөн ``Predictrix'' аппликейшний архитектур, backend API (FastAPI), мобайл интерфейсийн (React Native) хөгжүүлэлтийг багтаасан.

\textbf{4-р бүлэг -- Үр дүн}

Загварын гүйцэтгэл (нарийвчлал, сэргээлт, F1-score), бэктестийн дүн (өгөөж, Sharpe Ratio, максимум уналт, Profit Factor), хөгжүүлэлтийн 7 үе шатуудын (Phase 1--7B) харьцуулалт, эрсдэлийн дүн шинжилгээ, системийн хязгаарлалтуудыг танилцуулна. Бодит цагийн гүйцэтгэл, худал дохионы шинжилгээ, зах зээлийн янз бүрийн нөхцөлд систем хэрхэн ажилласан талаар дэлгэрүүлнэ.

\textbf{5-р бүлэг -- Дүгнэлт ба цаашдын судалгаа}

Судалгааны гол үр дүнг товчлон дүгнэж, олж авсан мэдлэг, практик ач холбогдол, системийн давуу болон сул талуудыг үнэлнэ. Цаашдын судалгааны боломжит чиглэлүүд (нэмэлт валютын хослол, гүн сургалтын аргууд, бусад хугацааны интервал, эрсдэлийн удирдлага сайжруулах гэх мэт)-ийг санал болгоно.
