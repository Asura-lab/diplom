\chapter{УДИРТГАЛ}

\section{Судалгааны үндэслэл, ач холбогдол}

Санхүүгийн зах зээл нь дэлхийн эдийн засгийн хамгийн динамик бөгөөд нарийн төвөгтэй салбаруудын нэг юм. Өдөр тутам дэлхийн валютын зах зээл дээр 7.5 их наяд долларын арилжаа хийгддэг бөгөөд энэ нь аливаа хувьцааны зах зээлээс хавьгүй их хэмжээ юм \cite{bis2022}. Forex (Foreign Exchange) зах зээл нь 24 цагийн турш ажилладаг, хамгийн хөрвөх чадвартай санхүүгийн зах зээл бөгөөд олон улсын худалдаа, хөрөнгө оруулалтын үндэс суурь болдог.

Уламжлалт арилжааны арга нь хүний шинжилгээ, туршлага, зах зээлийн мэдрэмжид суурилдаг боловч хүний хязгаарлагдмал чадвар, сэтгэл хөдлөлийн нөлөөлөл зэргээс болж олон арилжаачид алдагдал хүлээдэг. Судалгаанаас үзэхэд арилжаачдын 70-80\% нь санхүүгийн зах зээл дээр алдагдалтай ажилладаг \cite{barber2014}. Энэ нь дараах шалтгаануудтай холбоотой:

\begin{itemize}
    \item \textbf{Хэт их хэмжээний мэдээлэл:} Зах зээлийн үнэ, эдийн засгийн мэдээ, геополитик үйл явдлууд зэрэг маш олон хүчин зүйлийг нэгэн зэрэг боловсруулах шаардлагатай
    \item \textbf{Сэтгэл хөдлөлийн нөлөөлөл:} Айдас, шунал зэрэг сэтгэл хөдлөл нь оновчтой шийдвэр гаргахад саад болдог
    \item \textbf{Хугацааны хязгаарлалт:} Зах зээл 24 цагийн турш ажилладаг учир хүн байнга хяналт тавих боломжгүй
    \item \textbf{Зах зээлийн таних чадварын хязгаарлалт:} Том хэмжээний түүхэн өгөгдлөөс зах зээлийн хөдөлгөөнийг зөв таамаглах хэцүү 
\end{itemize}

Сүүлийн жилүүдэд Machine Learning нь санхүүгийн салбарт өргөнөөр нэвтэрч байна. Эдгээр технологиуд нь дээрх асуудлуудыг шийдвэрлэх боломжийг олгож байна:

\begin{enumerate}
    \item Том хэмжээний өгөгдлийг хурдан боловсруулах
    \item Сэтгэл хөдлөлийн нөлөөлөлгүй объектив шийдвэр гаргах
    \item 24/7 тасралтгүй ажиллах
    \item Нүдэнд харагдахгүй хөдөлгөөнийг илрүүлэх
\end{enumerate}

Монгол улсад FinTech салбар хурдацтай хөгжиж байгаа боловч валютын арилжааны автоматжуулалтын чиглэлээр судалгаа, хөгжүүлэлт хомс байна. Энэхүү судалгааны ажил нь Machine Learning-ийн орчин үеийн аргуудыг ашиглан EUR/USD валютын хослолын ханшийн чиг хандлагыг таамаглах арилжааны дохио үүсгэх систем хөгжүүлэхэд оршино.

\section{Судалгааны зорилго, зорилт}

\subsection{Зорилго}

Энэхүү судалгааны ажлын гол зорилго нь Machine Learning аргуудыг нэгтгэн ашиглан Forex зах зээл дээрх EUR/USD валютын хослолын худалдан авах боломжийг таамаглаж, mobile application-аар дохио илгээх систем хөгжүүлэхэд оршино.

\subsection{Зорилтууд}

Дээрх зорилгод хүрэхийн тулд дараах зорилтуудыг дэвшүүлж байна:

\begin{enumerate}
    \item \textbf{Data Collection:} EUR/USD валютын хосын бодит цагийн үнийн өгөгдлийг татаж авах
    
    \item \textbf{Feature Engineering:} 32 Technical Indicator (RSI, MACD, Bollinger Bands, ATR, ADX гэх мэт) тооцоолох
    
    \item \textbf{ML Model Development:} Hybrid Ensemble систем - XGBoost×3, LightGBM×2, CatBoost×2 гэсэн долоон загварын нэгдэл бүтээх
    
    \item \textbf{Agreement Bonus System:} Загваруудын зөвшилцөлд суурилсан нэмэлт оноо систем хэрэгжүүлэх
    
    \item \textbf{Model Training \& Optimization:} Hyperparameter тохируулга, Cross-Validation, Class Imbalance шийдвэрлэх
    
    \item \textbf{Evaluation:} Accuracy, Profit Factor, Total Pips зэрэг хэмжүүрүүд ашиглан загварын гүйцэтгэлийг үнэлэх
    
    \item \textbf{Backend Development:} Flask + Waitress WSGI server-ээр бодит цагийн дохио үүсгэх API үүсгэх
    
    \item \textbf{Гар утасны аппликейшн хөгжүүлэх:} React Native + Expo ашиглан арилжааны дохиог хүлээн авах гар утасны апп бүтээх
    
    \item \textbf{Entry/SL/TP гаралт:} ATR-д суурилан Entry Price, Stop Loss, Take Profit байршуулах ханшуудыг тооцоолох
\end{enumerate}

\section{Судалгааны хамрах хүрээ}

Энэхүү судалгааны ажил нь дараах хүрээнд хязгаарлагдана:

\begin{itemize}
    \item \textbf{Валютын хос:} Зөвхөн EUR/USD (Евро/АНУ-ын доллар) валютын хослолыг судална
    \item \textbf{Цаг хугацааны хүрээ:} 1 минутын интервалтай өгөгдөл, 500 bar-ын түүхэн өгөгдөл
    \item \textbf{Машин сургалтын арга:} Supervised Learning буюу BUY-only Classification
    \item \textbf{Загварын архитектур:} Hybrid Ensemble - XGBoost×3 + LightGBM×2 + CatBoost×2
    \item \textbf{Платформ:} Python, Flask, React Native, Expo
    \item \textbf{Өгөгдлийн эх үүсвэр:} Twelve Data API
\end{itemize}

\section{Судалгааны шинэлэг тал}

Энэхүү судалгааны ажил нь дараах шинэлэг талуудтай:

\begin{enumerate}
    \item \textbf{Hybrid Ensemble арга:} 7 ялгаатай алгоритмын (XGBoost×3, LightGBM×2, CatBoost×2) нэгтгэлийг ашиглан дан загварын сул талыг нөхөж, таамаглалын найдвартай байдлыг нэмэгдүүлсэн
    
    \item \textbf{Agreement Bonus System:} Загваруудын зөвшилцөлд суурилсан шинэлэг оноо систем (+7\%, +4\%, +2\%) хэрэгжүүлсэн
    
    \item \textbf{32 Technical Indicator:} Trend, Momentum, Volatility, Time Features-ийг цогц ашигласан
    
    \item \textbf{BUY-only стратеги:} Зөвхөн худалдан авах боломжийг тодорхойлох нь шийдвэр гаргалтыг хялбарчилж, эрсдэлийг бууруулдаг
    
    \item \textbf{Entry/SL/TP гаралт:} ATR-д суурилсан тодорхой Entry Price, Stop Loss, Take Profit утгууд гаргадаг
    
    \item \textbf{Цогц систем:} Backend API + машин сургалтын загвар + гар утасны аппликейшн бүхий цогц систем
    
    \item \textbf{Бодит цагийн интеграц:} Бодит цагийн өгөгдлөөр сигнал үүсгэдэг
    
\end{enumerate}