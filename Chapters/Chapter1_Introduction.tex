\chapter{УДИРТГАЛ}

\section{Судалгааны үндэслэл, ач холбогдол}

Санхүүгийн зах зээл нь дэлхийн эдийн засгийн хамгийн динамик бөгөөд нарийн төвөгтэй салбаруудын нэг юм. Өдөр тутам дэлхийн валютын зах зээл дээр 7.5 их наяд долларын арилжаа хийгддэг бөгөөд энэ нь аливаа хувьцааны зах зээлээс хавьгүй их хэмжээ юм \cite{bis2022}. Forex (Foreign Exchange) зах зээл нь 24 цагийн турш ажилладаг, хамгийн хөрвөх чадвартай санхүүгийн зах зээл бөгөөд олон улсын худалдаа, хөрөнгө оруулалтын үндэс суурь болдог.

Уламжлалт арилжааны арга нь хүний шинжилгээ, туршлага, зах зээлийн мэдрэмжид суурилдаг боловч хүний хязгаарлагдмал чадвар, сэтгэл хөдлөлийн нөлөөлөл зэргээс болж олон арилжаачид алдагдал хүлээдэг. Судалгаанаас үзэхэд арилжаачдын 70-80\% нь санхүүгийн зах зээл дээр алдагдалтай ажилладаг \cite{barber2014}. Энэ нь дараах шалтгаануудтай холбоотой:

\begin{itemize}
    \item \textbf{Хэт их хэмжээний мэдээлэл:} Зах зээлийн үнэ, эдийн засгийн мэдээ, геополитик үйл явдлууд зэрэг маш олон хүчин зүйлийг нэгэн зэрэг боловсруулах шаардлагатай
    \item \textbf{Сэтгэл хөдлөлийн нөлөөлөл:} Айдас, шунал зэрэг сэтгэл хөдлөл нь оновчтой шийдвэр гаргахад саад болдог
    \item \textbf{Хугацааны хязгаарлалт:} Зах зээл 24 цагийн турш ажилладаг учир хүн байнга хяналт тавих боломжгүй
    \item \textbf{Зах зээлийн таних чадварын хязгаарлалт:} Том хэмжээний түүхэн өгөгдлөөс зах зээлийн хөдөлгөөнийг зөв таамаглах хэцүү 
\end{itemize}

Сүүлийн жилүүдэд машин сургалт (Machine Learning) болон гүн сургалтын (Deep Learning) технологиуд санхүүгийн салбарт өргөнөөр нэвтэрч байна. Эдгээр технологиуд нь дээрх асуудлуудыг шийдвэрлэх боломжийг олгож байна:

\begin{enumerate}
    \item Том хэмжээний өгөгдлийг хурдан боловсруулах
    \item Сэтгэл хөдлөлийн нөлөөлөлгүй объектив шийдвэр гаргах
    \item 24/7 тасралтгүй ажиллах
    \item Нүдэнд харагдахгүй хөдөлгөөнийг илрүүлэх
\end{enumerate}

Монгол Улсад санхүүгийн технологийн (FinTech) салбар хурдацтай хөгжиж байгаа боловч валютын арилжааны автоматжуулалтын чиглэлээр судалгаа, хөгжүүлэлт хомс байна. Энэхүү судалгааны ажил нь уг цоорхойг нөхөхөд чиглэгдсэн бөгөөд машин сургалтын орчин үеийн аргуудыг ашиглан EUR/USD валютын хослолын ханшийн чиг хандлагыг таамаглах арилжааны бот хөгжүүлэхэд оршино.

\section{Судалгааны зорилго, зорилт}

\subsection{Зорилго}

Энэхүү судалгааны ажлын гол зорилго нь машин сургалтын гүн сургалтын аргуудыг ашиглан Forex зах зээл дээрх EUR/USD валютын хосын үнийн чиг хандлагыг таамаглаж, автоматаар арилжаа хийх бот систем хөгжүүлэхэд оршино.

\subsection{Зорилтууд}

Дээрх зорилгод хүрэхийн тулд дараах зорилтуудыг дэвшүүлж байна:

\begin{enumerate}
    \item \textbf{Өгөгдөл цуглуулах:} EUR/USD валютын хосын түүхэн үнийн өгөгдлийг цуглуулж, чанарын хяналт хийх
    
    \item \textbf{Шинж чанар инженерчлэл:} Техникийн индикаторууд (RSI, MACD, Bollinger Bands гэх мэт), цаг хугацааны шинж чанарууд, статистик шинж чанаруудыг тооцоолох
    
    \item \textbf{Машин сургалтын модель хөгжүүлэх:} CNN + BiLSTM + Attention архитектур бүхий гүн сургалтын модель бүтээх
    
    \item \textbf{Моделийг сургах, оновчлох:} Hyperparameter тохируулга, Early stopping, Learning rate scheduling зэрэг техникүүдийг ашиглах
    
    \item \textbf{Үнэлгээ хийх:} Regression болон classification metrics ашиглан моделийн гүйцэтгэлийг үнэлэх
    
    \item \textbf{Гар утасны аппликейшн хөгжүүлэх:} React Native ашиглан арилжааны дохиог хүлээн авах мобайл апп бүтээх
    
    \item \textbf{Бодит арилжааны туршилт:} MetaTrader 5 платформтой холбогдож арилжааны стратегийг турших
\end{enumerate}

\section{Судалгааны хамрах хүрээ}

Энэхүү судалгааны ажил нь дараах хүрээнд хязгаарлагдана:

\begin{itemize}
    \item \textbf{Валютын хос:} Зөвхөн EUR/USD (Евро/АНУ-ын доллар) валютын хосыг судална
    \item \textbf{Цаг хугацааны хүрээ:} 1 минутын интервалтай өгөгдөл, 10 минутын таамаглалын хугацаа
    \item \textbf{Машин сургалтын арга:} Supervised learning буюу хяналттай сургалтын арга
    \item \textbf{Моделийн архитектур:} CNN + BiLSTM + Multi-Head Attention гибрид архитектур
    \item \textbf{Платформ:} Python, PyTorch, React Native
\end{itemize}

\section{Судалгааны шинэлэг тал}

Энэхүү судалгааны ажил нь дараах шинэлэг талуудтай:

\begin{enumerate}
    \item \textbf{Гибрид архитектур:} CNN, BiLSTM, Multi-Head Attention механизмуудыг нэгтгэсэн орчин үеийн архитектур
    
    \item \textbf{Олон зорилтот сургалт:} Үнийн таамаглал (regression), чиглэлийн таамаглал (classification), том хөдөлгөөний таамаглал (±30 pip) гэсэн гурван зорилтыг нэгэн зэрэг шийдвэрлэх
    
    \item \textbf{Сайжруулсан сургалтын техник:} Focal Loss, Stochastic Weight Averaging (SWA), Mixup augmentation зэрэг орчин үеийн техникүүд
    
    \item \textbf{Бүрэн систем:} Backend API + Машин сургалтын модель + Мобайл аппликейшн бүхий бүрэн систем
    
    \item \textbf{Монгол хэл дээрх судалгаа:} Forex арилжааны машин сургалтын чиглэлээр Монгол хэл дээрх анхны дэлгэрэнгүй судалгааны нэг
\end{enumerate}

\section{Судалгааны бүтэц}

Энэхүү дипломын ажил нь дараах бүтэцтэй:

\begin{itemize}
    \item \textbf{1-р бүлэг - Удиртгал:} Судалгааны үндэслэл, зорилго, зорилт, хамрах хүрээ
    
    \item \textbf{2-р бүлэг - Онолын судалгаа:} Машин сургалт, гүн сургалтын онолын үндэс, холбогдох судалгаанууд
    
    \item \textbf{3-р бүлэг - Судалгааны арга зүй:} Өгөгдөл, шинж чанар инженерчлэл, моделийн архитектур, сургалтын процесс
    
    \item \textbf{4-р бүлэг - Судалгааны үр дүн:} Моделийн гүйцэтгэл, үнэлгээ, арилжааны туршилт
    
    \item \textbf{5-р бүлэг - Дүгнэлт:} Үр дүнгийн хэлэлцүүлэг, цаашдын судалгааны чиглэл
\end{itemize}
