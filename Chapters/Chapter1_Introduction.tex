\chapter{УДИРТГАЛ}

\section{Судалгааны үндэслэл, ач холбогдол}

Санхүүгийн зах зээл нь дэлхийн эдийн засгийн хамгийн динамик бөгөөд нарийн төвөгтэй салбаруудын нэг юм. Өдөр тутам дэлхийн валютын зах зээл дээр 7.5 их наяд долларын арилжаа хийгддэг бөгөөд энэ нь аливаа хувьцааны зах зээлээс хавьгүй их хэмжээ юм \cite{bis2022}. Forex (Foreign Exchange) зах зээл нь 24 цагийн турш ажилладаг, хамгийн хөрвөх чадвартай санхүүгийн зах зээл бөгөөд олон улсын худалдаа, хөрөнгө оруулалтын үндэс суурь болдог.

Уламжлалт арилжааны арга нь хүний шинжилгээ, туршлага, зах зээлийн мэдрэмжид суурилдаг боловч хүний хязгаарлагдмал чадвар, сэтгэл хөдлөлийн нөлөөлөл зэргээс болж олон арилжаачид алдагдал хүлээдэг. Судалгаанаас үзэхэд арилжаачдын 70-80\% нь санхүүгийн зах зээл дээр алдагдалтай ажилладаг \cite{barber2014}. Энэ нь дараах шалтгаануудтай холбоотой:

\begin{itemize}
    \item \textbf{Хэт их хэмжээний мэдээлэл:} Зах зээлийн үнэ, эдийн засгийн мэдээ, геополитик үйл явдлууд зэрэг маш олон хүчин зүйлийг нэгэн зэрэг боловсруулах шаардлагатай
    \item \textbf{Сэтгэл хөдлөлийн нөлөөлөл:} Айдас, шунал зэрэг сэтгэл хөдлөл нь оновчтой шийдвэр гаргахад саад болдог
    \item \textbf{Хугацааны хязгаарлалт:} Зах зээл 24 цагийн турш ажилладаг учир хүн байнга хяналт тавих боломжгүй
    \item \textbf{Зах зээлийн таних чадварын хязгаарлалт:} Том хэмжээний түүхэн өгөгдлөөс зах зээлийн хөдөлгөөнийг зөв таамаглах хэцүү 
\end{itemize}

Сүүлийн жилүүдэд машин сургалт (Machine Learning) нь санхүүгийн салбарт өргөнөөр нэвтэрч байна. Эдгээр технологиуд нь дээрх асуудлуудыг шийдвэрлэх боломжийг олгож байна:

\begin{enumerate}
    \item Том хэмжээний өгөгдлийг хурдан боловсруулах
    \item Сэтгэл хөдлөлийн нөлөөлөлгүй объектив шийдвэр гаргах
    \item 24/7 тасралтгүй ажиллах
    \item Нүдэнд харагдахгүй хөдөлгөөнийг илрүүлэх
\end{enumerate}

Монгол Улсад санхүүгийн технологийн (FinTech) салбар хурдацтай хөгжиж байгаа боловч валютын арилжааны автоматжуулалтын чиглэлээр судалгаа, хөгжүүлэлт хомс байна. Энэхүү судалгааны ажил нь уг цоорхойг нөхөхөд чиглэгдсэн бөгөөд машин сургалтын орчин үеийн аргуудыг ашиглан EUR/USD валютын хослолын ханшийн чиг хандлагыг таамаглах арилжааны дохио үүсгэх систем хөгжүүлэхэд оршино.

\section{Судалгааны зорилго, зорилт}

\subsection{Зорилго}

Энэхүү судалгааны ажлын гол зорилго нь машин сургалтын ensemble аргуудыг ашиглан Forex зах зээл дээрх EUR/USD валютын хосын худалдан авах боломжийг таамаглаж, мобайл аппликейшнээр дохио илгээх систем хөгжүүлэхэд оршино.

\subsection{Зорилтууд}

Дээрх зорилгод хүрэхийн тулд дараах зорилтуудыг дэвшүүлж байна:

\begin{enumerate}
    \item \textbf{Өгөгдөл цуглуулах:} EUR/USD валютын хосын бодит цагийн үнийн өгөгдлийг API-ээр татаж авах
    
    \item \textbf{Шинж чанар инженерчлэл:} 32 техникийн индикатор (RSI, MACD, Bollinger Bands, ATR, ADX гэх мэт) тооцоолох
    
    \item \textbf{Машин сургалтын модель хөгжүүлэх:} V10 Ensemble систем - XGBoost×3, LightGBM×2, CatBoost×2 гэсэн долоон загварын нэгдэл бүтээх
    
    \item \textbf{Agreement Bonus System:} Загваруудын зөвшилцөлд суурилсан нэмэлт оноо систем хэрэгжүүлэх
    
    \item \textbf{Моделийг сургах, оновчлох:} Hyperparameter тохируулга, cross-validation, class imbalance шийдвэрлэх
    
    \item \textbf{Үнэлгээ хийх:} Accuracy, Profit Factor, Total Pips зэрэг metrics ашиглан моделийн гүйцэтгэлийг үнэлэх
    
    \item \textbf{Backend API хөгжүүлэх:} Flask + Waitress WSGI серверээр бодит цагийн дохио үүсгэх API үүсгэх
    
    \item \textbf{Мобайл аппликейшн хөгжүүлэх:} React Native + Expo ашиглан арилжааны дохиог хүлээн авах мобайл апп бүтээх
    
    \item \textbf{Entry/SL/TP гаралт:} ATR-д суурилсан entry price, stop loss, take profit тооцоолох
\end{enumerate}

\section{Судалгааны хамрах хүрээ}

Энэхүү судалгааны ажил нь дараах хүрээнд хязгаарлагдана:

\begin{itemize}
    \item \textbf{Валютын хос:} Зөвхөн EUR/USD (Евро/АНУ-ын доллар) валютын хосыг судална
    \item \textbf{Цаг хугацааны хүрээ:} 1 минутын интервалтай өгөгдөл, 500 bar-ын түүхэн өгөгдөл
    \item \textbf{Машин сургалтын арга:} Supervised learning буюу хяналттай сургалтын арга (BUY-only classification)
    \item \textbf{Моделийн архитектур:} V10 Ensemble - XGBoost×3 + LightGBM×2 + CatBoost×2
    \item \textbf{Платформ:} Python, Flask, React Native, Expo
    \item \textbf{Өгөгдлийн эх үүсвэр:} Twelve Data API
\end{itemize}

\section{Судалгааны шинэлэг тал}

Энэхүү судалгааны ажил нь дараах шинэлэг талуудтай:

\begin{enumerate}
    \item \textbf{V10 Ensemble арга:} 7 ялгаатай алгоритмын (XGBoost×3, LightGBM×2, CatBoost×2) нэгтгэл нь дан загвараас маш их давуу 96.9\% accuracy өгдөг
    
    \item \textbf{Agreement Bonus System:} Загваруудын зөвшилцөлд суурилсан шинэлэг оноо систем (+7\%, +4\%, +2\%) хэрэгжүүлсэн
    
    \item \textbf{32 техникийн индикатор:} Trend, Momentum, Volatility, Time features-ийг цогц ашигласан
    
    \item \textbf{BUY-only стратеги:} Зөвхөн худалдан авах боломжийг тодорхойлох нь шийдвэр гаргалтыг хялбарчилж, эрсдэлийг бууруулдаг
    
    \item \textbf{Entry/SL/TP гаралт:} ATR-д суурилсан тодорхой entry price, stop loss, take profit утгууд гаргадаг
    
    \item \textbf{Бүрэн систем:} Backend API + Машин сургалтын модел + Мобайл аппликейшн бүхий бүрэн систем
    
    \item \textbf{Бодит цагийн интеграц:} Бодит цагийн өгөгдлөөр сигнал үүсгэдэг
    
\end{enumerate}