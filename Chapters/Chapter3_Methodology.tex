\chapter{Судалгааны арга зүй}
\label{ch:methodology}

\section{Системийн ерөнхий архитектур}

Хөгжүүлсэн систем нь гурван үндсэн хэсгээс бүрдэнэ: (1) ML загварын сургалт ба дохио үүсгэх хэсэг, (2) Backend API сервер, (3) Мобайл аппликейшн. \fref{fig:system_architecture} нь системийн ерөнхий бүтцийг харуулав.

\begin{figure}[H]
\centering
\begin{tikzpicture}[
    node distance=1.2cm and 2.5cm,
    box/.style={rectangle, draw, rounded corners, minimum width=3cm, minimum height=1cm, align=center, font=\small},
    data/.style={box, fill=blue!10},
    model/.style={box, fill=green!10},
    server/.style={box, fill=orange!10},
    app/.style={box, fill=purple!10},
    arrow/.style={-{Stealth[length=3mm]}, thick}
]

% Data layer
\node[data] (rawdata) {Түүхэн өгөгдөл\\(2015--2024)\\M1--H4 интервал};
\node[data, right=of rawdata] (features) {48 шинж чанар\\6 хугацааны хүрээ};
\node[model, right=of features] (ensemble) {GBDT Ансамбль\\LightGBM + XGBoost\\+ CatBoost};

% Signal layer
\node[model, below=of ensemble] (calibrator) {Logistic Regression\\Calibrator};
\node[model, left=of calibrator] (signals) {Дохио үүсгэгч\\(conf $\geq$ 0.90)};
\node[data, left=of signals] (mt5) {MT5 Strategy\\Tester бэктест};

% Server layer
\node[server, below=2cm of signals] (flask) {Flask Backend\\REST API сервер};
\node[server, left=of flask] (mongodb) {MongoDB\\өгөгдлийн сан};
\node[server, right=of flask] (twelvedata) {Twelve Data\\API (бодит цагийн)};

% App layer
\node[app, below=of flask] (mobile) {React Native\\Мобайл апп};
\node[app, left=of mobile] (auth) {Хэрэглэгчийн\\баталгаажуулалт};
\node[app, right=of mobile] (display) {Ханш, дохио,\\мэдээ харуулах};

% Arrows
\draw[arrow] (rawdata) -- (features);
\draw[arrow] (features) -- (ensemble);
\draw[arrow] (ensemble) -- (calibrator);
\draw[arrow] (calibrator) -- (signals);
\draw[arrow] (signals) -- (mt5);
\draw[arrow] (signals) -- (flask);
\draw[arrow] (twelvedata) -- (flask);
\draw[arrow] (mongodb) -- (flask);
\draw[arrow] (flask) -- (mobile);
\draw[arrow] (mobile) -- (auth);
\draw[arrow] (mobile) -- (display);

\end{tikzpicture}
\caption{Системийн ерөнхий архитектур}
\label{fig:system_architecture}
\end{figure}

\section{Өгөгдлийн бэлтгэл}

\subsection{Түүхэн өгөгдөл}

MetaTrader 5-аас EUR/USD валютын хослолын 2015--2024 оны OHLCV (Open, High, Low, Close, Volume) өгөгдлийг 6 хугацааны интервалаар татан авсан. \tref{tab:data_summary} нь өгөгдлийн хэмжээг харуулав.

\begin{table}[H]
\centering
\caption{Өгөгдлийн хэмжээний хураангуй}
\label{tab:data_summary}
\begin{tabular}{lrrl}
\toprule
\textbf{Интервал} & \textbf{Бааруудын тоо} & \textbf{Хугацаа} & \textbf{Үүрэг} \\
\midrule
M1 (1 мин) & $\sim$3,700,000 & 2015--2024 & Үндсэн давтамж \\
M5 (5 мин) & $\sim$740,000 & 2015--2024 & Нэмэлт шинж чанар \\
M15 (15 мин) & $\sim$247,000 & 2015--2024 & Нэмэлт шинж чанар \\
M30 (30 мин) & $\sim$123,000 & 2015--2024 & Нэмэлт шинж чанар \\
H1 (1 цаг) & $\sim$62,000 & 2015--2024 & Нэмэлт шинж чанар \\
H4 (4 цаг) & $\sim$15,500 & 2015--2024 & Нэмэлт шинж чанар \\
\bottomrule
\end{tabular}
\end{table}

\subsection{Шинж чанарын инженерчлэл (Feature Engineering)}

Интервал бүрээс 8 техник шинж чанар тооцоолж, нийт 48 шинж чанар бүхий өгөгдлийн бүтэц (feature matrix) үүсгэсэн. \tref{tab:features} нь шинж чанаруудыг жагсаав.

\begin{table}[H]
\centering
\caption{Шинж чанарын жагсаалт (интервал бүрд)}
\label{tab:features}
\begin{tabular}{llp{7cm}}
\toprule
\textbf{№} & \textbf{Шинж чанар} & \textbf{Тайлбар}  \\
\midrule
1 & \texttt{close} & Хаалтын үнэ \\
2 & \texttt{rsi\_14} & 14 цонхтой RSI \\
3 & \texttt{atr\_14} & 14 цонхтой ATR (хэлбэлзэл) \\
4 & \texttt{ma\_5} & 5 цонхтой SMA \\
5 & \texttt{ma\_20} & 20 цонхтой SMA \\
6 & \texttt{ma\_50} & 50 цонхтой SMA \\
7 & \texttt{volatility} & 20 цонхтой стандарт хазайлт \\
8 & \texttt{returns} & Үнийн өөрчлөлтийн хувь \\
\bottomrule
\end{tabular}
\end{table}

Шинж чанаруудыг \texttt{\_M1}, \texttt{\_M5}, \texttt{\_M15}, \texttt{\_M30}, \texttt{\_H1}, \texttt{\_H4} гэсэн дагаваруудаар ялгаж, нийт $8 \times 6 = 48$ шинж чанар үүсгэсэн. Жишээлбэл, \texttt{rsi\_M1} нь 1 минутын RSI, \texttt{atr\_H4} нь 4 цагийн ATR-ийг илэрхийлнэ.

\subsection{Шошго (Label) үүсгэх}

Сургалтын шошгыг ирээдүйн 240 минутын (4 цаг) үнийн хөдөлгөөнд үндэслэн гурван ангилалд хуваасан:

\begin{equation}
\text{label} = \begin{cases}
    \text{BUY (1)} & \text{хэрэв } \Delta_{\text{up}} \geq 30\text{ пипс ба } \Delta_{\text{up}} > 1.5 \cdot \Delta_{\text{down}} \\
    \text{SELL (-1)} & \text{хэрэв } \Delta_{\text{down}} \geq 30\text{ пипс ба } \Delta_{\text{down}} > 1.5 \cdot \Delta_{\text{up}} \\
    \text{HOLD (0)} & \text{бусад тохиолдолд}
\end{cases}
\end{equation}

Үүнд:
\begin{itemize}
    \item $\Delta_{\text{up}}$ -- ирээдүйн 240 минутын хамгийн дээд үнэ ба одоогийн хаалтын үнийн зөрүү (пипсээр)
    \item $\Delta_{\text{down}}$ -- одоогийн хаалтын үнэ ба ирээдүйн хамгийн доод үнийн зөрүү (пипсээр)
    \item 30 пипс -- хамгийн бага шаардлагатай хөдөлгөөн (шуугианаас ялгах)
    \item 1.5 дахин давамгайлал -- чиг хандлагын тодорхой байдлыг шаардах
\end{itemize}

\subsection{Өгөгдлийн хуваалт -- Walk-Forward Validation}

Цаг хугацааны дарааллыг хадгалсан Walk-Forward Validation аргачлалыг ашиглан өгөгдлийг дараахь байдлаар хуваасан:

\begin{table}[H]
\centering
\caption{Walk-Forward Validation -- өгөгдлийн хуваалт}
\label{tab:data_split}
\begin{tabular}{llrl}
\toprule
\textbf{Бүлэг} & \textbf{Хугацаа} & \textbf{Баарын тоо} & \textbf{Зорилго} \\
\midrule
Сургалт & 2015--2022 & $\sim$2,972,000 & Загвар сургах (80\%) \\
Баталгаажуулалт & 2023 & $\sim$371,000 & Early stopping, calibration (10\%) \\
Тест & 2024 & $\sim$371,000 & Эцсийн үнэлгээ (10\%) \\
\bottomrule
\end{tabular}
\end{table}

Энэ хуваалт нь look-ahead bias-аас бүрэн сэргийлж, загварын бодит гүйцэтгэлийг зөв үнэлэх боломжийг олгоно.

\section{Загварын бүтэц ба сургалт}

\subsection{Ансамбль загварын бүтэц}

Системийн загвар нь гурван GBDT загвараас бүрдэх ансамбль юм. \tref{tab:model_params} нь загвар бүрийн гол параметрүүдийг харуулав.

\begin{table}[H]
\centering
\caption{Загвар бүрийн гол гиперпараметрүүд}
\label{tab:model_params}
\begin{tabular}{llll}
\toprule
\textbf{Параметр} & \textbf{LightGBM} & \textbf{XGBoost} & \textbf{CatBoost} \\
\midrule
Модны тоо & 496 & $\sim$400 & 499 \\
Хамгийн их гүн & 4 & 4 & 4 \\
Сургалтын хурд & 0.03 & 0.03 & 0.03 \\
L1 нормчлол & Тийм & Тийм & -- \\
L2 нормчлол & Тийм & Тийм & Тийм \\
Early stopping & 50 давталт & 50 давталт & 50 давталт \\
Модны өсөлтийн арга & Leaf-wise & Level-wise & Symmetric \\
\bottomrule
\end{tabular}
\end{table}

Загварын бүтцийг \fref{fig:ensemble_arch} нь харуулав.

\begin{figure}[H]
\centering
\resizebox{\textwidth}{!}{
\begin{tikzpicture}[
    node distance=1cm and 1.5cm,
    box/.style={rectangle, draw, rounded corners, minimum width=2.8cm, minimum height=0.9cm, align=center, font=\small},
    input/.style={box, fill=blue!15},
    model/.style={box, fill=green!15},
    output/.style={box, fill=red!15},
    arrow/.style={-{Stealth[length=2.5mm]}, thick}
]

% Input
\node[input] (X) {48 шинж чанар};

% Models
\node[model, above right=1cm and 2cm of X] (lgb) {LightGBM\\496 мод};
\node[model, right=2cm of X] (xgb) {XGBoost\\$\sim$400 мод};
\node[model, below right=1cm and 2cm of X] (cat) {CatBoost\\499 мод};

% Probabilities
\node[box, fill=yellow!15, right=2cm of xgb] (avg) {Магадлалын\\дундаж};

% Calibrator
\node[box, fill=orange!15, right=1.5cm of avg] (cal) {LogReg\\Calibrator};

% Output
\node[output, right=1.5cm of cal] (signal) {Дохио\\BUY/SELL/HOLD};

% Arrows
\draw[arrow] (X) -- (lgb);
\draw[arrow] (X) -- (xgb);
\draw[arrow] (X) -- (cat);
\draw[arrow] (lgb) -- (avg);
\draw[arrow] (xgb) -- (avg);
\draw[arrow] (cat) -- (avg);
\draw[arrow] (avg) -- (cal);
\draw[arrow] (cal) -- (signal);

\end{tikzpicture}
}
\caption{Ансамбль загварын архитектур}
\label{fig:ensemble_arch}
\end{figure}

\subsection{Overfitting-аас сэргийлэх арга хэмжээ}

Санхүүгийн загварт overfitting нь хамгийн чухал сорилт юм. Дараахь арга хэмжээнүүдийг авсан:

\begin{enumerate}
    \item \textbf{Модны гүнийг хязгаарлах}: \texttt{max\_depth=4} -- энгийн мод нь ерөнхийлөн суралцах чадвартай
    \item \textbf{Сургалтын хурд бага}: \texttt{learning\_rate=0.03} -- удаан ч тогтвортой суралцах
    \item \textbf{Early stopping}: Баталгаажуулалтын алдаа 50 давталт дотор сайжрахгүй бол зогсоох
    \item \textbf{L1/L2 нормчлол}: Загвар хэт нарийн тааруулахаас сэргийлэх
    \item \textbf{Walk-forward validation}: Цаг хугацааны дарааллыг хадгалах
    \item \textbf{Шинж чанарын хялбаржуулалт}: Анхны 75 шинж чанараас 48 болгож бууруулсан (Phase 6B)
\end{enumerate}

\subsection{Сургалтын алгоритм}

Загварын сургалтын үндсэн алгоритмыг \fref{alg:training} нь харуулав.

\begin{algorithm}[H]
\caption{Ансамбль загварын сургалт}
\label{alg:training}
\begin{algorithmic}[1]
\REQUIRE Өгөгдлийн бүтэц $D = \{(x_i, y_i)\}_{i=1}^{N}$, хугацааны хуваалт
\ENSURE Сургагдсан ансамбль загвар $\mathcal{M}$
\STATE $D_{\text{train}}, D_{\text{cal}}, D_{\text{val}}, D_{\text{test}} \gets \text{TimeSplit}(D)$
\STATE Шинж чанар тооцоолох: $X \gets \text{ComputeFeatures}(D)$ (48 шинж чанар)
\FOR{загвар $k \in \{\text{LightGBM, XGBoost, CatBoost}\}$}
    \STATE $M_k \gets \text{Train}(X_{\text{train}}, y_{\text{train}})$ early stopping ${X_{\text{val}}}$-тэй
\ENDFOR
\STATE Магадлал: $P_k \gets M_k.\text{predict\_proba}(X_{\text{cal}})$ загвар бүрд
\STATE Calibrator: $\text{LR} \gets \text{LogisticRegression.fit}(\bar{P}, y_{\text{cal}})$
\STATE Тест дээрх гүйцэтгэл үнэлэх: $\text{accuracy}(M, X_{\text{test}}, y_{\text{test}})$
\STATE \textbf{return} $\mathcal{M} = \{M_1, M_2, M_3, \text{LR}\}$
\end{algorithmic}
\end{algorithm}

\section{Дохио үүсгэх систем}

\subsection{Дохионы шүүлтүүр}

Ансамбль загварын 2025 оны өгөгдөл дээрх анхны 359,639 таамгаас дохионы шүүлтүүрийг хэрэглэж 1,065 чанартай дохио үүсгэсэн. Шүүлтүүрийн нөхцөлүүд:

\begin{itemize}
    \item \textbf{Итгэлцлийн босго}: Calibrated confidence $\geq$ 0.90 (90\%)
    \item \textbf{ATR шүүлтүүр}: ATR $\geq$ 4.0 пипс (зах зээлд хангалттай хэлбэлзэл байх)
\end{itemize}

Шүүлтүүрийн үр дүнд:
\begin{equation}
\frac{1{,}065}{359{,}639} \times 100\% = 0.296\%
\end{equation}

Бүх таамгийн зөвхөн 0.3\% нь шаардлагыг хангасан -- энэ нь ``чанар $>$ тоо хэмжээ'' зарчмыг баримталсан болохыг харуулна.

\subsection{Stop Loss ба Take Profit тооцоолол}

SL/TP-г ATR дээр суурилан динамикаар тооцоолсон:

\begin{equation}
    SL = \max(\text{ATR}_{14} \times 5.0,\ 15\text{ пипс})
\end{equation}
\begin{equation}
    TP = \max(SL \times 3.0,\ 45\text{ пипс})
\end{equation}

\begin{itemize}
    \item SL-ийн ATR үржвэр: 5.0 -- зах зээлийн хэлбэлзэлд тохирсон
    \item TP/SL харьцаа: 3:1 -- эрсдэл-өгөөжийн харьцаа
    \item Хамгийн бага SL: 15 пипс -- хэт бага SL-аас сэргийлэх
    \item Хамгийн бага TP: 45 пипс -- утга учиртай ашгийг баталгаажуулах
\end{itemize}

\subsection{Эрсдэлийн менежмент}

Арилжаа бүрд балансын 1\%-ийг эрсдэлд оруулах зарчим баримталсан. Лотын тооцоолол:

\begin{equation}
    \text{Lot} = \frac{\text{Balance} \times 0.01}{\text{SL}_{\text{пипс}} \times \text{Pip Value}}
\end{equation}

МaxPositions = 1 гэсэн хязгаарлалт тавьж, нэг удаад зөвхөн нэг нээлттэй позиц байлгасан.

\section{MetaTrader 5 бэктест}

\subsection{Expert Advisor (EA) бүтэц}

MQL5 хэл дээр хөгжүүлсэн Expert Advisor нь CSV файлаас дохиог уншиж, бодит зах зээлийн нөхцөлд (спрэд, слиппэж) шалгадаг. EA-ийн гол параметрүүд:

\begin{table}[H]
\centering
\caption{EA-ийн тохиргооны параметрүүд}
\label{tab:ea_params}
\begin{tabular}{lll}
\toprule
\textbf{Параметр} & \textbf{Утга} & \textbf{Тайлбар} \\
\midrule
\texttt{RiskPerTrade} & 1.0\% & Арилжаа бүрийн эрсдэл \\
\texttt{MaxPositions} & 1 & Нэг удаад нэг позиц \\
\texttt{MinConfidence} & 0.90 & Итгэлцлийн доод хязгаар \\
\texttt{SlippagePoints} & 10 & Зөвшөөрөгдөх слиппэж \\
\texttt{MagicNumber} & 60609688 & EA-ийн дугаар \\
\bottomrule
\end{tabular}
\end{table}

\subsection{Бэктестийн нөхцөл}

\begin{itemize}
    \item \textbf{Платформ}: MetaTrader 5 Strategy Tester
    \item \textbf{Горим}: Every tick (бүх тикийн өгөгдөлтэй)
    \item \textbf{Хугацаа}: 2025.01.01 -- 2025.10.31
    \item \textbf{Анхны хөрөнгө}: \$10,000
    \item \textbf{Хэрэгсэл}: EUR/USD (бодит спрэд)
\end{itemize}

\section{Backend API хөгжүүлэлт}

\subsection{Технологийн стек}

Backend серверийг Python хэл дээр Flask фреймворк ашиглан хөгжүүлсэн. \tref{tab:backend_stack} нь хэрэглэсэн технологиудыг харуулав.

\begin{table}[H]
\centering
\caption{Backend технологийн стек}
\label{tab:backend_stack}
\begin{tabular}{llp{6cm}}
\toprule
\textbf{Технологи} & \textbf{Хувилбар} & \textbf{Үүрэг} \\
\midrule
Python & 3.10+ & Серверийн хэл \\
Flask & 3.0+ & REST API фреймворк \\
MongoDB & 7.0+ & Хэрэглэгч, дохионы мэдээллийн сан \\
JWT & -- & Хэрэглэгчийн баталгаажуулалт \\
Twelve Data API & -- & Бодит цагийн Forex ханш (20 хослол) \\
Google Gemini & -- & AI зах зээлийн дүн шинжилгээ \\
Waitress & -- & WSGI сервер (4 thread) \\
Flask-Mail & -- & Имэйл баталгаажуулалт \\
\bottomrule
\end{tabular}
\end{table}

\subsection{API Endpoint-ууд}

Backend API нь дараахь гол endpoint-уудтай:

\begin{table}[H]
\centering
\caption{Гол API endpoint-ууд}
\label{tab:api_endpoints}
\small
\begin{tabular}{lll}
\toprule
\textbf{Endpoint} & \textbf{Арга} & \textbf{Тайлбар} \\
\midrule
\texttt{/auth/register} & POST & Хэрэглэгч бүртгэх \\
\texttt{/auth/verify-email} & POST & Имэйл баталгаажуулах \\
\texttt{/auth/login} & POST & Нэвтрэх \\
\texttt{/rates/live} & GET & 20 хослолын бодит ханш \\
\texttt{/signal/v2} & GET & V10 ML дохио авах \\
\texttt{/signal/save} & POST & Дохио хадгалах \\
\texttt{/signals/history} & GET & Дохионы түүх \\
\texttt{/api/news} & GET & Мэдээ мэдээлэл \\
\texttt{/api/market-analysis} & GET & AI зах зээлийн дүн шинжилгээ \\
\texttt{/health} & GET & Серверийн төлөв \\
\bottomrule
\end{tabular}
\end{table}

\subsection{Бодит цагийн мэдээлэл}

Twelve Data API-аас EUR/USD-ийн бодит ханш авч, кэш механизмтай (2 минутын TTL) ажилладаг. 20 валютын хослолд зориулсан ханшийг cross-rate тооцоолсноор нэг API дуудлагаар бүгдийг нийлүүлдэг. Rate limiting (1 хүсэлт/минут, үнэгүй төлөвлөгөө)-ийг non-blocking хандлагаар шийдсэн -- rate limit-д орвол кэш-дэх сүүлийн хадгалсан өгөгдлийг шууд буцаана.

\subsection{Дохио үүсгэх урсгал (Signal Generation Flow)}

Мобайл аппликейшнаас \texttt{/signal/v2} endpoint-руу хүсэлт ирэхэд:
\begin{enumerate}
    \item Twelve Data API-аас сүүлийн 500 бааран (1 минутын интервал) авна
    \item Загварын шинж чанарыг тооцоолно (48 шинж чанар)
    \item V10 ансамбль загвараар магадлал тооцоолно
    \item Confidence босго шалгана ($\geq$ 85\%)
    \item BUY/SELL/HOLD дохиог динамик SL/TP-тэй хамт буцаана
\end{enumerate}

\section{Мобайл аппликейшн хөгжүүлэлт}

\subsection{Технологийн стек}

\begin{table}[H]
\centering
\caption{Мобайл аппликейшний технологийн стек}
\label{tab:mobile_stack}
\begin{tabular}{llp{6cm}}
\toprule
\textbf{Технологи} & \textbf{Хувилбар} & \textbf{Үүрэг} \\
\midrule
React Native & 0.74.5 & Cross-platform мобайл фреймворк \\
TypeScript & 5.9+ & Type-safe хөгжүүлэлт \\
Expo & 51.0 & Хөгжүүлэлтийн хэрэгсэл \\
React Navigation & 6.x & Навигаци (Stack + Tab) \\
React Query & 5.x & Серверийн өгөгдөл менежмент \\
Axios & 1.6+ & HTTP client \\
AsyncStorage & -- & Локал хадгалалт (JWT token) \\
\bottomrule
\end{tabular}
\end{table}

\subsection{Аппликейшний бүтэц}

Аппликейшн нь дараахь дэлгэцүүдтэй:

\begin{itemize}
    \item \textbf{Нэвтрэх дэлгэц} (LoginScreen) -- Имэйл + нууц үгээр нэвтрэх
    \item \textbf{Бүртгэлийн дэлгэц} (SignUpScreen) -- Шинэ хэрэглэгч бүртгэх
    \item \textbf{Баталгаажуулалт} (EmailVerificationScreen) -- 6 оронтой код
    \item \textbf{Нүүр дэлгэц} (HomeScreen) -- 20 валютын хослолын бодит ханш, өсөлт/бууралт
    \item \textbf{Дохио дэлгэц} (SignalScreen) -- ML дохио, итгэлцлийн хувь, SL/TP
    \item \textbf{Мэдээний дэлгэц} (NewsScreen) -- Эдийн засгийн мэдээ, AI шинжилгээ
    \item \textbf{Профайл дэлгэц} (ProfileScreen) -- Хэрэглэгчийн мэдээлэл, тохиргоо
\end{itemize}

Аппликейшний навигацийн бүтэц:
\begin{itemize}
    \item \textbf{Stack Navigator}: Auth дэлгэцүүд (Login, SignUp, Verify, Forgot Password, Main)
    \item \textbf{Bottom Tab Navigator}: Market (Нүүр), News (Мэдээ), Profile (Профайл)
\end{itemize}

\subsection{Бодит цагийн өгөгдлийн менежмент}

React Query ашиглан серверийн өгөгдлийг бодит цагийн горимд удирддаг:
\begin{itemize}
    \item Ханшийн дата: 60 секунд тутам автоматаар шинэчлэгднэ
    \item API төлөв: 30 секунд тутам шалгана
    \item Pull-to-refresh: Хэрэглэгч гараар шинэчлэх боломжтой
\end{itemize}

\section{Хөгжүүлэлтийн үе шатууд}

Систем нь 7 давталтат (iterative) үе шатаар хөгжсөн:

\begin{table}[H]
\centering
\caption{Хөгжүүлэлтийн үе шатууд}
\label{tab:dev_phases}
\small
\begin{tabular}{clp{8cm}}
\toprule
\textbf{Үе шат} & \textbf{Нэр} & \textbf{Гол өөрчлөлт} \\
\midrule
Phase 1 & ATR шүүлтүүр & ATR босго 3$\to$4 пипс (86.8\% дохио буурсан) \\
Phase 2 & Шинж чанар нэмэх & +15 нэмэлт шинж чанар \\
Phase 3 & Загварын олон янз байдал & 3$\to$9 загвар (сүүлд 3 болгож хялбаршуулсан) \\
Phase 4 & Trailing stop & Хойшлогдсон SL стратеги \\
Phase 5 & Drawdown засвар & MaxPositions=1, Risk=1\% \\
Phase 6B & Overfitting засвар & 75$\to$48 шинж чанар, walk-forward, шуурхай загвар \\
Phase 7B & Чанартай дохио & conf$\geq$0.90, ATR$\geq$4.0, TP:SL=3:1 \\
\bottomrule
\end{tabular}
\end{table}

\section{Бүлгийн дүгнэлт}

Арга зүйн хүрээнд олон хугацааны хүрээний 48 шинж чанарыг тооцоолж, 3 GBDT загварын ансамблийг walk-forward validation-аар сургаж, чанарын шүүлтүүрээр дохио үүсгэж, MT5-д бэктест хийж, Flask backend болон React Native мобайл аппликейшнаар хэрэглэгчдэд хүргэх бүрэн системийг бүтээсэн. Дараагийн бүлэгт энэ системийн бодит үр дүнг танилцуулна.
