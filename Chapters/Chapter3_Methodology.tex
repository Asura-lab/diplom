\chapter{СУДАЛГААНЫ АРГА ЗҮЙ}

\section{Системийн ерөнхий тойм}

Энэхүү хэсэгт системийн бүтэц, ажиллагааны урсгал, хэрэглэгчийн харилцан үйлдлийг диаграммуудаар харуулав.

\subsection{Системийн Flow Diagram}

Системийн ажиллагааны үндсэн урсгалыг доорх диаграммаар харуулав:

\begin{figure}[H]
\centering
\begin{tikzpicture}[
    scale=0.95, transform shape,
    node distance=1cm,
    startstop/.style={rectangle, rounded corners, minimum width=2.5cm, minimum height=0.7cm, draw=black, fill=red!20, align=center, font=\small},
    process/.style={rectangle, minimum width=2.5cm, minimum height=0.7cm, draw=black, fill=blue!20, align=center, font=\small},
    decision/.style={diamond, minimum width=2cm, minimum height=0.9cm, draw=black, fill=yellow!20, align=center, aspect=2, font=\small},
    io/.style={trapezium, trapezium left angle=70, trapezium right angle=110, minimum width=2.2cm, minimum height=0.6cm, draw=black, fill=green!20, align=center, font=\small},
    arrow/.style={thick,->,>=stealth}
]
    % Nodes
    \node (start) [startstop] {Эхлэх};
    \node (fetch) [io, below=of start] {Twelve Data API-\\аас өгөгдөл татах};
    \node (validate) [decision, below=of fetch] {Өгөгдөл\\хүчинтэй?};
    \node (cache) [process, right=1.5cm of validate] {Cache-аас\\авах};
    \node (features) [process, below=of validate] {70 техникийн\\индикатор тооцоох};
    \node (scale) [process, below=of features] {StandardScaler\\хэрэглэх};
    \node (predict) [process, below=of scale] {Ensemble загвар\\таамаглал хийх};
    \node (conf) [decision, below=of predict] {Итгэлцүүр$\geq$\\Confidence threshold?};
    \node (buy) [io, below left=0.8cm and 1.2cm of conf] {BUY дохио\\үүсгэх};
    \node (hold) [io, below right=0.8cm and 1.2cm of conf] {HOLD\\буцаах};
    \node (sltp) [process, below=of buy] {SL/TP\\тооцоох};
    \node (send) [io, below=1.2cm of conf] {API хариу\\илгээх};
    \node (end) [startstop, below=of send] {Дуусах};
    
    % Arrows
    \draw [arrow] (start) -- (fetch);
    \draw [arrow] (fetch) -- (validate);
    \draw [arrow] (validate) -- node[anchor=south, font=\scriptsize] {Үгүй} (cache);
    \draw [arrow] (cache) |- (features);
    \draw [arrow] (validate) -- node[anchor=east, font=\scriptsize] {Тийм} (features);
    \draw [arrow] (features) -- (scale);
    \draw [arrow] (scale) -- (predict);
    \draw [arrow] (predict) -- (conf);
    \draw [arrow] (conf) -- node[anchor=south east, font=\scriptsize] {Тийм} (buy);
    \draw [arrow] (conf) -- node[anchor=south west, font=\scriptsize] {Үгүй} (hold);
    \draw [arrow] (buy) -- (sltp);
    \draw [arrow] (sltp) |- (send);
    \draw [arrow] (hold) |- (send);
    \draw [arrow] (send) -- (end);
\end{tikzpicture}
\caption{Дохио үүсгэх системийн Flow Diagram}
\label{fig:flow_diagram}
\end{figure}

\clearpage
\subsection{Хэрэглээний тохиолдлын диаграмм}

Системийн хэрэглэгчийн харилцан үйлдлийг Use Case диаграммаар харуулав:

\begin{figure}[H]
\centering
\begin{tikzpicture}[
    scale=1.0, transform shape,
    usecase/.style={ellipse, draw=black, thick, minimum width=3.5cm, minimum height=1.2cm, align=center, font=\small},
    system/.style={rectangle, draw=black, dashed, thick, minimum width=9cm, minimum height=13cm},
    arrow/.style={-, thick}
]
    % System boundary
    \node[system, label={[font=\large\bfseries]above:FOREX Signal System}] (sys) at (5,-4) {};
    
    % Actor - Trader (left)
    \draw[thick] (-1.5,0) circle (0.3);
    \draw[thick] (-1.5,-0.3) -- (-1.5,-1);
    \draw[thick] (-1.5,-1) -- (-1.9,-1.8);
    \draw[thick] (-1.5,-1) -- (-1.1,-1.8);
    \draw[thick] (-1.9,-0.5) -- (-1.1,-0.5);
    \node[font=\small] at (-1.5,-2.2) {Арилжаачин};
    
    % Actor - Admin (bottom left)
    \draw[thick] (-1.5,-8) circle (0.3);
    \draw[thick] (-1.5,-8.3) -- (-1.5,-9);
    \draw[thick] (-1.5,-9) -- (-1.9,-9.8);
    \draw[thick] (-1.5,-9) -- (-1.1,-9.8);
    \draw[thick] (-1.9,-8.5) -- (-1.1,-8.5);
    \node[font=\small] at (-1.5,-10.2) {Админ};
    
    % Use cases - centered in system (swapped uc4 and uc5)
    \node[usecase] (uc1) at (5,1) {Дохио хүлээн авах};
    \node[usecase] (uc2) at (5,-1) {Таамаглал харах};
    \node[usecase] (uc3) at (5,-3) {Түүх харах};
    \node[usecase] (uc4) at (5,-5) {Мэдэгдэл авах};
    \node[usecase] (uc5) at (5,-7) {Тохиргоо хийх};
    \node[usecase] (uc6) at (5,-9) {Загвар сургах};
    
    % Connections from Trader to use cases
    \draw[arrow] (-1,-0.5) -- (uc1.west);
    \draw[arrow] (-1,-0.7) -- (uc2.west);
    \draw[arrow] (-1,-0.9) -- (uc3.west);
    \draw[arrow] (-1,-1.1) -- (uc4.west);
    \draw[arrow] (-1,-1.3) -- (uc5.west);
    
    % Connections from Admin to use cases
    \draw[arrow] (-1,-8.5) -- (uc5.west);
    \draw[arrow] (-1,-8.7) -- (uc6.west);
\end{tikzpicture}
\caption{Хэрэглээний тохиолдлын диаграмм}
\label{fig:usecase_diagram}
\end{figure}

\subsection{Activity Diagram - Дохио авах үйлдэл}

Хэрэглэгч дохио авах үйлдлийн Activity диаграмм:

\begin{figure}[H]
\centering
\begin{tikzpicture}[
    scale=0.85, transform shape,
    node distance=0.8cm,
    initial/.style={circle, draw=black, fill=black, minimum size=0.4cm},
    final/.style={circle, draw=black, thick, fill=white, minimum size=0.5cm, path picture={\fill[black] (path picture bounding box.center) circle (0.15cm);}},
    action/.style={rectangle, rounded corners=8pt, draw=black, thick, minimum width=2.4cm, minimum height=0.8cm, align=center, font=\small},
    decision/.style={diamond, draw=black, thick, fill=yellow!30, minimum width=1.4cm, minimum height=0.8cm, aspect=1.8, align=center, font=\small},
    mergepoint/.style={circle, draw=black, fill=black, minimum size=0.08cm, inner sep=0pt},
    arrow/.style={->, >=stealth, thick},
    line/.style={-, thick}
]
    % Swimlanes - Backend API wider (7cm)
    \node[rectangle, draw=black, thick, minimum width=3.5cm, minimum height=13cm, fill=blue!8] (s1) at (0,0) {};
    \node[rectangle, draw=black, thick, minimum width=7cm, minimum height=13cm, fill=green!8] (s2) at (5.5,0) {};
    \node[rectangle, draw=black, thick, minimum width=3.5cm, minimum height=13cm, fill=orange!8] (s3) at (11,0) {};
    
    % Swimlane labels - moved higher
    \node[font=\small\bfseries] at (0,6.8) {Mobile App};
    \node[font=\small\bfseries] at (5.5,6.8) {Backend API};
    \node[font=\small\bfseries] at (11,6.8) {ML Model};
    
    % Mobile App actions
    \node[initial] (start) at (0,5.3) {};
    \node[action, fill=blue!15] (open) at (0,4.2) {Апп нээх};
    \node[action, fill=blue!15] (request) at (0,2.5) {Дохио хүсэх};
    \node[action, fill=blue!15] (display) at (0,-4.5) {Дохио\\харуулах};
    \node[final] (end) at (0,-5.8) {};
    
    % Backend actions
    \node[action, fill=green!15] (fetch) at (4,2.5) {Өгөгдөл\\татах};
    \node[decision] (check) at (4,0.5) {Cache?};
    \node[action, fill=green!15] (api) at (7.2,0.5) {API дуудах};
    \node[mergepoint] (merge) at (5.5,-1.2) {};
    \node[action, fill=green!15] (process) at (5.5,-2.5) {Өгөгдөл\\боловсруулах};
    \node[action, fill=green!15] (response) at (5.5,-4.5) {Хариу\\илгээх};
    
    % ML actions
    \node[action, fill=orange!15] (features) at (11,-2.5) {Онцлог\\тооцоох};
    \node[action, fill=orange!15] (predict) at (11,-4.5) {Таамаглал\\хийх};
    
    % Arrows - Mobile App
    \draw[arrow] (start) -- (open);
    \draw[arrow] (open) -- (request);
    
    % Arrow - Mobile to Backend
    \draw[arrow] (request.east) -- (fetch.west);
    
    % Arrow - fetch to check
    \draw[arrow] (fetch) -- (check);
    
    % Arrow - Cache? No -> API (with visible label)
    \draw[arrow] (check.east) -- node[above=2pt, font=\scriptsize, fill=white, inner sep=1pt] {Үгүй} (api.west);
    
    % Arrow - API -> merge point (down then left - no arrow head)
    \draw[line] (api.south) -- (7.2,-1.2) -- (merge);
    
    % Arrow - Cache? Yes -> merge point (down then right - no arrow head)
    \draw[line] (check.south) -- node[left, font=\scriptsize] {Тийм} (4,-1.2) -- (merge);
    
    % Arrow - merge point to process (SINGLE arrow)
    \draw[arrow] (merge) -- (process.north);
    
    % Arrow - process to features
    \draw[arrow] (process.east) -- (features.west);
    
    % Arrow - features to predict
    \draw[arrow] (features) -- (predict);
    
    % Arrow - predict to response (left then down - 90 degree)
    \draw[arrow] (predict.west) -- (9.5,-4.5) -- (response.east);
    
    % Arrow - response to display
    \draw[arrow] (response.west) -- (display.east);
    
    % Arrow - display to end
    \draw[arrow] (display) -- (end);
\end{tikzpicture}
\caption{Дохио авах Activity Diagram}
\label{fig:activity_diagram}
\end{figure}

\subsection{EUR/USD Үнийн динамик диаграмм}

2025 оны 3-4-р сарын EUR/USD валютын хосын бодит үнийн хөдөлгөөн ба техникийн индикаторуудыг харуулав. Энэ хугацаанд BUY дохионууд дундажаар 1.9\% ашиг өгсөн:

\begin{figure}[H]
\centering
\begin{tikzpicture}
\begin{axis}[
    width=14cm,
    height=7cm,
    xlabel={Өдөр (2025.03.02 - 2025.04.30)},
    ylabel={Үнэ (EUR/USD)},
    xmin=0, xmax=53,
    ymin=1.02, ymax=1.16,
    legend pos=north west,
    ymajorgrids=true,
    grid style=dashed,
    title={EUR/USD 2025 оны 3-4-р сарын бодит үнийн хөдөлгөөн}
]

% Close Price - Real data from EUR_USD_test.csv (March-April 2025)
\addplot[color=black, thick, mark=none, smooth] coordinates {
    (1,1.04119) (2,1.04851) (3,1.06230) (4,1.07941) (5,1.07871) (6,1.08292) (7,1.08600) (8,1.08390) (9,1.09136)
    (10,1.08852) (11,1.08548) (12,1.08765) (13,1.08792) (14,1.09180) (15,1.09385) (16,1.09115) (17,1.08532) (18,1.08121) (19,1.08358) (20,1.08022) (21,1.07888) (22,1.07409) (23,1.08006) (24,1.08237) (25,1.08234) (26,1.08173) (27,1.07940) (28,1.09041) (29,1.10450) (30,1.09573) (31,1.09832) (32,1.09146) (33,1.09773) (34,1.09507) (35,1.12576) (36,1.13554) (37,1.13420) (38,1.13360) (39,1.12932) (40,1.13953) (41,1.13715) (42,1.13901) (43,1.14488) (44,1.15137) (45,1.13504) (46,1.13292) (47,1.13725) (48,1.13628) (49,1.13448) (50,1.14075) (51,1.13902) (52,1.13219)
};
\addlegendentry{Close Price}

% SMA 7
\addplot[color=blue, thick, dashed, mark=none, smooth] coordinates {
    (1,1.04119) (2,1.04485) (3,1.05067) (4,1.05785) (5,1.06202) (6,1.06551) (7,1.06843) (8,1.07454) (9,1.08066)
    (10,1.08440) (11,1.08527) (12,1.08655) (13,1.08726) (14,1.08809) (15,1.08951) (16,1.08948) (17,1.08902) (18,1.08841) (19,1.08783) (20,1.08673) (21,1.08489) (22,1.08206) (23,1.08048) (24,1.08006) (25,1.08022) (26,1.07996) (27,1.07984) (28,1.08149) (29,1.08583) (30,1.08807) (31,1.09035) (32,1.09165) (33,1.09394) (34,1.09617) (35,1.10122) (36,1.10566) (37,1.11115) (38,1.11619) (39,1.12160) (40,1.12757) (41,1.13359) (42,1.13548) (43,1.13681) (44,1.13927) (45,1.13947) (46,1.13999) (47,1.13966) (48,1.13954) (49,1.13889) (50,1.13830) (51,1.13653) (52,1.13613)
};
\addlegendentry{SMA 7}

% SMA 14
\addplot[color=red, thick, dotted, mark=none, smooth] coordinates {
    (1,1.04119) (2,1.04485) (3,1.05067) (4,1.05785) (5,1.06202) (6,1.06551) (7,1.06843) (8,1.07037) (9,1.07270)
    (10,1.07428) (11,1.07530) (12,1.07633) (13,1.07722) (14,1.07826) (15,1.08202) (16,1.08507) (17,1.08671) (18,1.08684) (19,1.08719) (20,1.08700) (21,1.08649) (22,1.08579) (23,1.08498) (24,1.08454) (25,1.08432) (26,1.08389) (27,1.08329) (28,1.08319) (29,1.08395) (30,1.08427) (31,1.08520) (32,1.08594) (33,1.08695) (34,1.08801) (35,1.09136) (36,1.09574) (37,1.09961) (38,1.10327) (39,1.10663) (40,1.11075) (41,1.11488) (42,1.11835) (43,1.12124) (44,1.12521) (45,1.12783) (46,1.13079) (47,1.13362) (48,1.13656) (49,1.13718) (50,1.13756) (51,1.13790) (52,1.13780)
};
\addlegendentry{SMA 14}

% BUY signal markers - High quality signals
\addplot[only marks, mark=triangle*, mark size=4pt, color=green!70!black] coordinates {
    (2,1.04851) (24,1.08237) (28,1.09041) (33,1.09773) (35,1.12576) (50,1.14075)
};
\addlegendentry{BUY Signal}

% Bollinger Band upper
\addplot[color=gray, thin, mark=none, smooth] coordinates {
    (1,1.04719) (2,1.05520) (3,1.07210) (4,1.09151) (5,1.09663) (6,1.10085) (7,1.10422) (8,1.10232) (9,1.09893)
    (10,1.09363) (11,1.09338) (12,1.09231) (13,1.09209) (14,1.09382) (15,1.09533) (16,1.09525) (17,1.09560) (18,1.09701) (19,1.09719) (20,1.09771) (21,1.09623) (22,1.09281) (23,1.08765) (24,1.08617) (25,1.08653) (26,1.08575) (27,1.08564) (28,1.09123) (29,1.10381) (30,1.10659) (31,1.10951) (32,1.10947) (33,1.10982) (34,1.10560) (35,1.12427) (36,1.14055) (37,1.15057) (38,1.15695) (39,1.15670) (40,1.15757) (41,1.14295) (42,1.14254) (43,1.14684) (44,1.15373) (45,1.15359) (46,1.15256) (47,1.15241) (48,1.15241) (49,1.15233) (50,1.15085) (51,1.14195) (52,1.14243)
};
\addlegendentry{BB Upper}

% Bollinger Band lower
\addplot[color=gray, thin, mark=none, smooth] coordinates {
    (1,1.03519) (2,1.03450) (3,1.02923) (4,1.02420) (5,1.02742) (6,1.03016) (7,1.03265) (8,1.04676) (9,1.06238)
    (10,1.07518) (11,1.07716) (12,1.08078) (13,1.08243) (14,1.08236) (15,1.08370) (16,1.08371) (17,1.08245) (18,1.07982) (19,1.07848) (20,1.07576) (21,1.07355) (22,1.07132) (23,1.07331) (24,1.07394) (25,1.07391) (26,1.07416) (27,1.07404) (28,1.07175) (29,1.06785) (30,1.06954) (31,1.07118) (32,1.07383) (33,1.07806) (34,1.08675) (35,1.07818) (36,1.07077) (37,1.07174) (38,1.07544) (39,1.08651) (40,1.09758) (41,1.12422) (42,1.12842) (43,1.12679) (44,1.12481) (45,1.12535) (46,1.12741) (47,1.12691) (48,1.12666) (49,1.12545) (50,1.12575) (51,1.13112) (52,1.12983)
};
\addlegendentry{BB Lower}

\end{axis}
\end{tikzpicture}
\caption{EUR/USD 2025 оны 3-4-р сарын бодит үнийн хөдөлгөөн ба техникийн индикаторууд}
\label{fig:price_chart}
\end{figure}

\subsection{Системийн архитектурын диаграмм}

Системийн бүрэлдэхүүн хэсгүүдийн харилцан холболтыг давхаргат архитектураар харуулав:

\begin{figure}[H]
\centering
\begin{tikzpicture}[scale=0.9, transform shape,
    box/.style={rectangle, draw=#1!70, line width=1pt, fill=#1!12, rounded corners=3pt, minimum height=1cm, align=center, font=\sffamily\small, drop shadow={shadow xshift=1pt, shadow yshift=-1pt, opacity=0.2}},
    layer/.style={rectangle, draw=#1!40, fill=#1!6, rounded corners=5pt, line width=0.8pt},
    db/.style={cylinder, draw=orange!70, fill=orange!12, shape border rotate=90, aspect=0.3, minimum height=0.9cm, minimum width=1.4cm, font=\sffamily\footnotesize, drop shadow={shadow xshift=1pt, shadow yshift=-1pt, opacity=0.2}},
    myarrow/.style={-{Stealth[length=2.5mm]}, line width=1pt, #1},
    mydarrow/.style={{Stealth[length=2.5mm]}-{Stealth[length=2.5mm]}, line width=1pt, #1}
]
    % === LAYERS ===
    \node[layer=violet, minimum width=3cm, minimum height=4.5cm] (L1) at (0,0) {};
    \node[layer=blue, minimum width=3cm, minimum height=4.5cm] (L2) at (4,0) {};
    \node[layer=green!70!black, minimum width=7.2cm, minimum height=2cm] (L3) at (10,1.5) {};
    \node[layer=orange, minimum width=7.2cm, minimum height=2cm] (L4) at (10,-1.5) {};
    
    % === LAYER LABELS ===
    \node[font=\sffamily\small\bfseries, violet!60!black] at (0,2) {Presentation};
    \node[font=\sffamily\small\bfseries, blue!60!black] at (4,2) {Application};
    \node[font=\sffamily\small\bfseries, green!50!black] at (10,2.3) {ML Layer};
    \node[font=\sffamily\small\bfseries, orange!60!black] at (10,-0.7) {Data Layer};
    
    % === COMPONENTS ===
    \node[box=violet, minimum width=2.4cm, minimum height=1.3cm] (app) at (0,0) {Mobile App\\\scriptsize Expo + React Native};
    \node[box=blue, minimum width=2.4cm, minimum height=1.3cm] (api) at (4,0) {Flask API\\\scriptsize Waitress WSGI};
    
    % ML Models - тэнцүү зайтай (7.2cm layer, 3 model x 1.9cm = 5.7cm, зай = 0.75cm x 2)
    \node[box=green!70!black, minimum width=1.9cm] (xgb) at (7.55,1.5) {\footnotesize XGBoost};
    \node[box=green!70!black, minimum width=1.9cm] (lgb) at (10,1.5) {\footnotesize LightGBM};
    \node[box=green!70!black, minimum width=2.1cm] (rf) at (12.45,1.5) {\footnotesize RandomForest};
    
    % Data
    \node[db] (mongo) at (8,-1.5) {MongoDB};
    \node[box=cyan, minimum width=2cm] (twelve) at (12,-1.5) {\footnotesize Twelve Data};
    
    % === ARROWS ===
    \draw[mydarrow=violet!70] (app.east) -- (api.west);
    \node[font=\sffamily\scriptsize, fill=white, inner sep=1pt, text=violet!70] at (2,0.15) {REST};
    \node[font=\sffamily\scriptsize, fill=white, inner sep=1pt, text=violet!70] at (2,-0.15) {API};
    
    \draw[myarrow=blue!70] (api.east) -- ++(0.8,0) |- (L3.west) node[pos=0.25, above, font=\sffamily\scriptsize, fill=white, inner sep=1pt] {Predict};
    
    \draw[mydarrow=orange!70] (api.east) -- ++(0.8,0) |- (mongo.west) node[pos=0.25, below, font=\sffamily\scriptsize, fill=white, inner sep=1pt] {Users/Signals};
    
    \draw[myarrow=cyan!70] (twelve.west) -- (mongo.east) node[midway, above, font=\sffamily\scriptsize] {Prices};
    
    % === ENSEMBLE LABEL ===
    \node[font=\sffamily\scriptsize\itshape, green!40!black] at (10,0.7) {Weighted Ensemble Voting};

\end{tikzpicture}
\caption{Системийн архитектурын диаграмм}
\label{fig:system_architecture}
\end{figure}

\subsection{Sequence Diagram - API дуудлага}

API дуудлагын дарааллын диаграмм:

\begin{figure}[H]
\centering
\begin{tikzpicture}[scale=0.9, transform shape,
    actor/.style={rectangle, draw=black, fill=yellow!20, minimum width=2.2cm, minimum height=0.7cm, font=\sffamily\small\bfseries},
    msg/.style={-{Stealth[length=2.5mm]}, thick},
    ret/.style={{Stealth[length=2.5mm]}-, thick},
    lbl/.style={font=\sffamily\scriptsize, fill=white, inner sep=1pt}
]
    % === ACTORS ===
    \node[actor] (user) at (0,0) {Хэрэглэгч};
    \node[actor] (app) at (3.5,0) {Аппликейшн};
    \node[actor] (api) at (7,0) {Backend API};
    \node[actor] (ml) at (10.5,0) {ML Model};
    \node[actor] (data) at (14,0) {Twelve Data};
    
    % === LIFELINES ===
    \draw[thick] (0,-0.5) -- (0,-9.5);
    \draw[thick] (3.5,-0.5) -- (3.5,-9.5);
    \draw[thick] (7,-0.5) -- (7,-9.5);
    \draw[thick] (10.5,-0.5) -- (10.5,-9.5);
    \draw[thick] (14,-0.5) -- (14,-9.5);
    
    % === ACTIVATION BOXES ===
    \fill[yellow!30] (3.35,-1.2) rectangle (3.65,-8.3);
    \fill[green!20] (6.85,-2.2) rectangle (7.15,-7.3);
    \fill[orange!20] (10.35,-5.2) rectangle (10.65,-6.3);
    \fill[cyan!20] (13.85,-3.2) rectangle (14.15,-4.3);
    
    % === MESSAGES ===
    % 1. Апп нээх
    \draw[msg] (0,-1.5) -- (3.35,-1.5);
    \node[lbl] at (1.75,-1.2) {1. Апп нээх};
    
    % 2. Дохио хүсэх
    \draw[msg] (3.65,-2.5) -- (6.85,-2.5);
    \node[lbl] at (5.25,-2.2) {2. Дохио хүсэх};
    
    % 3. Өгөгдөл татах
    \draw[msg] (7.15,-3.5) -- (13.85,-3.5);
    \node[lbl] at (10.5,-3.2) {3. Өгөгдөл татах};
    
    % 4. OHLCV өгөгдөл
    \draw[ret] (7.15,-4.3) -- (13.85,-4.3);
    \node[lbl] at (10.5,-4) {4. OHLCV өгөгдөл};
    
    % 5. Таамаглал хүсэх
    \draw[msg] (7.15,-5.5) -- (10.35,-5.5);
    \node[lbl] at (8.75,-5.2) {5. Таамаглал хүсэх};
    
    % 6. Таамаглал
    \draw[ret] (7.15,-6.3) -- (10.35,-6.3);
    \node[lbl] at (8.75,-6) {6. Таамаглал};
    
    % 7. Дохио илгээх
    \draw[ret] (3.65,-7.3) -- (6.85,-7.3);
    \node[lbl] at (5.25,-7) {7. Дохио илгээх};
    
    % 8. Push notification
    \draw[ret] (0,-8.3) -- (3.35,-8.3);
    \node[lbl] at (1.75,-8) {8. Push notification};

\end{tikzpicture}
\caption{Sequence диаграм - Дохио хүлээн авах}
\label{fig:sequence_diagram}
\end{figure}

\section{Өгөгдлийн тодорхойлолт}

\subsection{Өгөгдлийн эх сурвалж}

Энэхүү судалгаанд EUR/USD валютын хосын 1 минутын интервалтай түүхэн өгөгдлийг ашигласан. Өгөгдлийг Twelve Data API үйлчилгээнээс татан авсан бөгөөд дараах талбаруудыг агуулна:

\begin{itemize}
    \item \textbf{timestamp} - Цаг хугацааны тэмдэг (UTC)
    \item \textbf{open} - Нээлтийн үнэ
    \item \textbf{high} - Хамгийн өндөр үнэ
    \item \textbf{low} - Хамгийн бага үнэ
    \item \textbf{close} - Хаалтын үнэ
    \item \textbf{volume} - Арилжааны хэмжээ
\end{itemize}

\subsection{Өгөгдлийн хугацааны хүрээ}

\begin{table}[H]
\centering
\caption{Өгөгдлийн хугацааны хүрээ}
\begin{tabular}{|l|l|l|r|}
\hline
\textbf{Төрөл} & \textbf{Эхлэх огноо} & \textbf{Дуусах огноо} & \textbf{Мөрийн тоо} \\
\hline
Сургалтын өгөгдөл & 2019-12-31 & 2024-12-30 & 1,859,492 \\
\hline
Тестийн өгөгдөл & 2024-12-31 & 2025-10-17 & 296,778 \\
\hline
\end{tabular}
\label{tab:data_range}
\end{table}

\subsection{Өгөгдлийн чанарын хяналт}

Өгөгдлийн чанарыг дараах алхмуудаар шалгаж, засварласан:

\begin{enumerate}
    \item \textbf{Давхардсан бичлэг шалгах:} Timestamp давхардсан бичлэгүүдийг устгах
    \item \textbf{Дутуу утга нөхөх:} Forward fill аргаар дутуу үнийн утгуудыг нөхөх
    \item \textbf{Аномали илрүүлэх:} Хэвийн бус үнийн өөрчлөлтүүдийг шалгах
\end{enumerate}

\section{Шинж чанар инженерчлэл}

Энэхүү судалгаанд нийт 70 техникийн индикаторыг шинж чанар болгон ашигласан. Эдгээр индикаторуудыг дараах ангиллаар бүлэглэж болно:

\subsection{Чиг хандлагын индикаторууд (Trend Indicators)}

\subsubsection{Хөдөлгөөнт дундаж (Moving Averages)}

Энгийн хөдөлгөөнт дундаж (SMA) ба экспоненциал хөдөлгөөнт дундаж (EMA)-ийг 6 өөр хугацаанд (5, 10, 20, 50, 100, 200) тооцоолсон.

\subsubsection{MA Crossover дохионууд}

Хөдөлгөөнт дундажийн огтлолцлыг BUY дохионы шинж чанар болгон ашигласан:

\begin{itemize}
    \item \textbf{sma\_5\_20\_cross} - SMA(5) > SMA(20) бол 1
    \item \textbf{sma\_20\_50\_cross} - SMA(20) > SMA(50) бол 1
    \item \textbf{ema\_10\_50\_cross} - EMA(10) > EMA(50) бол 1
    \item \textbf{golden\_cross} - SMA(50) > SMA(200) бол 1 (Golden Cross)
\end{itemize}

\subsubsection{Үнэ ба MA-ийн харьцаа}

\begin{equation}
\text{price\_vs\_sma20} = \frac{C - SMA_{20}}{SMA_{20}} \times 100
\end{equation}

\subsection{Моментум индикаторууд (Momentum Indicators)}

\subsubsection{RSI (Relative Strength Index)}

RSI-ийг 7, 14, 21 хугацаанд тооцоолсон:

\begin{equation}
RSI = 100 - \frac{100}{1 + RS}
\end{equation}

Энд $RS = \frac{\text{Average Gain}}{\text{Average Loss}}$

RSI-ийн бүсчлэл:
\begin{itemize}
    \item \textbf{rsi\_oversold} - RSI(14) < 30 бол 1 (хэт зарагдсан)
    \item \textbf{rsi\_bullish} - 50 < RSI(14) < 70 бол 1 (өсөлтийн бүс)
\end{itemize}

\subsubsection{MACD (Moving Average Convergence Divergence)}

\begin{align}
MACD &= EMA_{12} - EMA_{26} \\
Signal &= EMA_9(MACD) \\
Histogram &= MACD - Signal
\end{align}

MACD дохионууд:
\begin{itemize}
    \item \textbf{macd\_cross} - MACD > Signal бол 1
    \item \textbf{macd\_bullish} - MACD > Signal болон Histogram > 0 бол 1
\end{itemize}

\subsubsection{Stochastic Oscillator}

\begin{equation}
\%K = \frac{C - L_{14}}{H_{14} - L_{14}} \times 100
\end{equation}

Энд $L_{14}$ ба $H_{14}$ нь 14 хугацааны хамгийн бага ба өндөр үнэ.

\subsubsection{ROC (Rate of Change)}

\begin{equation}
ROC_n = \frac{C_t - C_{t-n}}{C_{t-n}} \times 100
\end{equation}

\subsection{Хэлбэлзлийн индикаторууд (Volatility Indicators)}

\subsubsection{ATR (Average True Range)}

ATR нь Stop Loss, Take Profit тооцоолоход чухал үүрэгтэй:

\begin{equation}
TR = max(H - L, |H - C_{prev}|, |L - C_{prev}|)
\end{equation}

\begin{equation}
ATR_{14} = SMA_{14}(TR)
\end{equation}

ATR-ийг pip болгон хөрвүүлэх:
\begin{equation}
ATR_{pips} = ATR_{14} \times 10000
\end{equation}

\subsubsection{Bollinger Bands}

\begin{align}
BB_{middle} &= SMA_{20} \\
BB_{upper} &= BB_{middle} + 2 \times \sigma_{20} \\
BB_{lower} &= BB_{middle} - 2 \times \sigma_{20}
\end{align}

Bollinger Bands шинж чанарууд:
\begin{itemize}
    \item \textbf{bb\_width} - Зурвасын өргөн (хувиар)
    \item \textbf{bb\_position} - Үнэ зурвасын хаана байгаа (0-1)
    \item \textbf{bb\_squeeze} - Зурваснаас нягтралт үүсч байвал 1
\end{itemize}

\subsection{Лааны загварууд (Candle Patterns)}

Лааны биеийн хэмжээ, сүүдрийн урт зэргийг тооцоолж, bullish engulfing, hammer зэрэг загваруудыг илрүүлсэн.

\subsection{Дэмжлэг/Эсэргүүцэл (Support/Resistance)}

Pivot Point системийг ашигласан:

\begin{align}
Pivot &= \frac{H_{prev} + L_{prev} + C_{prev}}{3} \\
R1 &= 2 \times Pivot - L_{prev} \\
S1 &= 2 \times Pivot - H_{prev} \\
R2 &= Pivot + (H_{prev} - L_{prev}) \\
S2 &= Pivot - (H_{prev} - L_{prev})
\end{align}

\subsection{Чиг хандлагын хүч (Trend Strength)}

Гурван хугацааны (богино, дунд, урт) чиг хандлагыг нэгтгэн trend\_alignment үзүүлэлтийг тооцоолсон. Бүх гурван хугацаа өсөлтийн чиглэлтэй бол strong\_uptrend = 1 гэж тодорхойлсон.

\subsection{BUY Score - Нэгдсэн дохионы үнэлгээ}

BUY дохионы хүчийг MACD, RSI, MA crossover, Golden Cross, чиг хандлага зэрэг индикаторуудын нийлбэр оноогоор тодорхойлсон.

\section{Зорилтот хувьсагч (Target Variable)}

\subsection{BUY-Only Classification}

Энэхүү судалгаанд BUY дохиог таамаглахад анхаарал хандуулсан. Зорилтот хувьсагчийг дараах байдлаар тодорхойлсон:

\begin{itemize}
    \item \textbf{BUY (1):} Take Profit (TP) нь Stop Loss (SL)-ээс өмнө хүрсэн
    \item \textbf{NOT\_BUY (0):} SL эхлээд хүрсэн эсвэл аль нь ч хүрээгүй
\end{itemize}

\subsection{Параметрүүд}

\begin{table}[H]
\centering
\caption{Зорилтот хувьсагчийн параметрүүд}
\begin{tabular}{|l|l|l|}
\hline
\textbf{Параметр} & \textbf{Утга} & \textbf{Тайлбар} \\
\hline
Forward periods & 60 bars & 1 цагийн дотор \\
\hline
Take Profit & 20 pips & Ашгийн зорилт \\
\hline
Stop Loss & 10 pips & Алдагдлын хязгаар \\
\hline
Risk:Reward & 1:2 & Эрсдэл/Ашгийн харьцаа \\
\hline
\end{tabular}
\label{tab:target_params}
\end{table}

\section{Өгөгдлийн хуваалт ба Scaling}

\subsection{Хуваалт}

Санхүүгийн өгөгдөлд хугацааны дарааллыг хадгалах шаардлагатай тул temporal split ашигласан:

\begin{itemize}
    \item \textbf{Train set:} 2019-12-31 - 2024-12-30 (1,859,492 мөр)
    \item \textbf{Test set:} 2024-12-31 - 2025-10-17 (296,778 мөр)
\end{itemize}

\subsection{StandardScaler}

Бүх шинж чанаруудыг StandardScaler ашиглан масштабчилсан:

\begin{equation}
x_{scaled} = \frac{x - \mu}{\sigma}
\end{equation}

\section{Моделийн архитектур}

\subsection{Ensemble арга}

Энэхүү судалгаанд гурван машин сургалтын моделийн ensemble ашигласан:

\begin{enumerate}
    \item \textbf{XGBoost} - Gradient Boosting Decision Trees
    \item \textbf{LightGBM} - Light Gradient Boosting Machine
    \item \textbf{Random Forest} - Санамсаргүй ойн ангилал
\end{enumerate}

\begin{figure}[H]
\centering
\begin{tikzpicture}[
    block/.style={rectangle, draw, rounded corners, minimum width=2.5cm, minimum height=0.8cm, fill=blue!20},
    arrow/.style={-Stealth, thick},
    node distance=1cm
]
    % Input
    \node[block, fill=gray!20, minimum width=4cm] (input) {70 Technical Features};
    
    % Models
    \node[block, below left=1.5cm and 0.5cm of input, fill=green!20] (xgb) {XGBoost};
    \node[block, below=1.5cm of input, fill=yellow!20] (lgb) {LightGBM};
    \node[block, below right=1.5cm and 0.5cm of input, fill=orange!20] (rf) {Random Forest};
    
    % Ensemble
    \node[block, below=3cm of input, fill=red!20, minimum width=4cm] (ensemble) {Average Ensemble};
    
    % Output
    \node[block, below=1cm of ensemble, fill=purple!20, minimum width=3cm] (output) {BUY Probability};
    
    % Arrows
    \draw[arrow] (input) -- (xgb);
    \draw[arrow] (input) -- (lgb);
    \draw[arrow] (input) -- (rf);
    \draw[arrow] (xgb) -- (ensemble);
    \draw[arrow] (lgb) -- (ensemble);
    \draw[arrow] (rf) -- (ensemble);
    \draw[arrow] (ensemble) -- (output);
\end{tikzpicture}
\caption{Ensemble моделийн архитектур}
\label{fig:ensemble_arch}
\end{figure}

\subsection{XGBoost}

XGBoost (eXtreme Gradient Boosting) нь gradient boosting framework дээр суурилсан, маш үр дүнтэй ангиллын алгоритм юм. n\_estimators=300, max\_depth=6, learning\_rate=0.05 параметрүүдтэй тохируулсан.

\subsection{LightGBM}

LightGBM нь Microsoft-ийн боловсруулсан, хурдан бөгөөд санах ойн хэрэглээ бага gradient boosting framework юм. XGBoost-тэй ижил параметрүүдээр тохируулсан.

\subsection{Random Forest}

Random Forest нь олон шийдвэрийн модны (decision tree) ensemble бөгөөд overfitting-д тэсвэртэй байдаг. n\_estimators=200, max\_depth=10, class\_weight='balanced' параметрүүдтэй.

\subsection{Ensemble Averaging}

Гурван моделийн магадлалыг дундажлаж эцсийн таамаглалыг гаргана:

\begin{equation}
P_{BUY} = \frac{P_{XGBoost} + P_{LightGBM} + P_{RandomForest}}{3}
\end{equation}

\subsection{Hyperparameter}

\begin{table}[H]
\centering
\caption{Моделийн Hyperparameter тохиргоо}
\begin{tabular}{|l|l|l|l|}
\hline
\textbf{Параметр} & \textbf{XGBoost} & \textbf{LightGBM} & \textbf{Random Forest} \\
\hline
n\_estimators & 300 & 300 & 200 \\
\hline
max\_depth & 6 & 6 & 10 \\
\hline
learning\_rate & 0.05 & 0.05 & - \\
\hline
subsample & 0.8 & 0.8 & - \\
\hline
colsample\_bytree & 0.8 & 0.8 & - \\
\hline
\end{tabular}
\label{tab:hyperparams}
\end{table}

\section{Confidence Threshold}

Моделийн итгэлцүүрийн босго нь дохиог үүсгэх эсэхийг тодорхойлно:

\begin{itemize}
    \item $P_{BUY} \geq 80\%$: BUY дохио үүсгэнэ
    \item $P_{BUY} < 80\%$: HOLD (хүлээх)
\end{itemize}

80\% босго нь моделийн нарийвчлалыг 80\%+ түвшинд хадгалахад тусална.

\section{Динамик SL/TP тооцоолол}

Stop Loss ба Take Profit-ийг ATR дээр суурилан динамикаар тооцоолно:

\begin{align}
SL_{pips} &= ATR_{pips} \times 1.5 \\
TP_{pips} &= ATR_{pips} \times 2.5
\end{align}

Хамгийн бага утгууд:
\begin{itemize}
    \item $SL_{min} = 8$ pips
    \item $TP_{min} = 12$ pips
\end{itemize}

\section{Backend API}

\subsection{Flask + Waitress}

Backend нь Flask framework ба Waitress WSGI server ашиглан хөгжүүлэгдсэн. /signal/v2 endpoint нь Twelve Data API-аас өгөгдөл авч, ML загваруудаар таамаглал хийж, JSON хэлбэрээр дохио буцаана.

\subsection{Twelve Data API}

Бодит цагийн болон түүхэн өгөгдлийг Twelve Data API-аас авдаг:

\begin{itemize}
    \item \textbf{Live rate:} \texttt{/price} endpoint
    \item \textbf{Historical data:} \texttt{/time\_series} endpoint
    \item \textbf{Cache TTL:} Live - 2 минут, Historical - 5 минут
    \item \textbf{Rate limit:} 1 request/minute (free tier)
\end{itemize}

\section{Мобайл аппликейшн}

\subsection{React Native + Expo}

Мобайл аппликейшнийг React Native framework ба Expo toolchain ашиглан хөгжүүлсэн.

\subsection{Үндсэн дэлгэцүүд}

\begin{enumerate}
    \item \textbf{HomeScreen} - Валютын хосуудын жагсаалт, бодит цагийн ханш
    \item \textbf{SignalScreen} - AI дохио, итгэлцүүр, Entry/SL/TP
    \item \textbf{ProfileScreen} - Хэрэглэгчийн мэдээлэл
    \item \textbf{SettingsScreen} - Тохиргоо (theme, notification)
\end{enumerate}

\subsection{API Integration}

Мобайл апп axios сан ашиглан Backend API-тай холбогдож дохио авдаг.

\section{MongoDB Database}

\subsection{Collections}

\begin{itemize}
    \item \textbf{users} - Хэрэглэгчийн мэдээлэл, нууц үг (bcrypt hash)
    \item \textbf{signals} - Хадгалагдсан дохионууд
    \item \textbf{verification\_codes} - Имэйл баталгаажуулалтын код (TTL: 10 мин)
\end{itemize}

\subsection{Authentication}

JWT (JSON Web Token) ашиглан хэрэглэгчийг баталгаажуулна:

\begin{itemize}
    \item \textbf{Token хугацаа}: 7 хоног
    \item \textbf{Algorithm}: HS256
    \item \textbf{Payload}: user\_id, email, exp (expiration)
    \item \textbf{Нууцлал}: SECRET\_KEY ашиглан encode хийнэ
\end{itemize}
