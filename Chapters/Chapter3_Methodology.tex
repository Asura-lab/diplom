\chapter{СУДАЛГААНЫ АРГА ЗҮЙ}

\section{Өгөгдлийн тодорхойлолт}

\subsection{Өгөгдлийн эх сурвалж}

Энэхүү судалгаанд EUR/USD валютын хосын 1 минутын интервалтай түүхэн өгөгдлийг ашигласан. Өгөгдлийг MetaTrader 5 (MT5) платформоос татан авсан бөгөөд дараах талбаруудыг агуулна:

\begin{itemize}
    \item \textbf{timestamp} - Цаг хугацааны тэмдэг (UTC)
    \item \textbf{open} - Нээлтийн үнэ
    \item \textbf{high} - Хамгийн өндөр үнэ
    \item \textbf{low} - Хамгийн бага үнэ
    \item \textbf{close} - Хаалтын үнэ
    \item \textbf{volume} - Арилжааны хэмжээ
\end{itemize}

\subsection{Өгөгдлийн хугацааны хүрээ}

\begin{table}[H]
\centering
\caption{Өгөгдлийн хугацааны хүрээ}
\begin{tabular}{|l|l|l|l|}
\hline
\textbf{Төрөл} & \textbf{Эхлэх огноо} & \textbf{Дуусах огноо} & \textbf{Мөрийн тоо} \\
\hline
Сургалтын өгөгдөл & 2023-01-01 & 2024-09-30 & ~600,000 \\
\hline
Тестийн өгөгдөл & 2024-10-01 & 2024-11-15 & ~65,000 \\
\hline
\end{tabular}
\label{tab:data_range}
\end{table}

\subsection{Өгөгдлийн чанарын хяналт}

Өгөгдлийн чанарыг дараах алхмуудаар шалгаж, засварласан:

\begin{enumerate}
    \item \textbf{Давхардсан бичлэг шалгах:} Timestamp давхардсан бичлэгүүдийг устгах
    \item \textbf{Дутуу утга нөхөх:} Forward fill аргаар дутуу үнийн утгуудыг нөхөх
    \item \textbf{Минутын grid бүрдүүлэх:} Бүх минут бүрийн өгөгдөл байгаа эсэхийг шалгах
    \item \textbf{Аномали илрүүлэх:} Хэвийн бус үнийн өөрчлөлтүүдийг шалгах
\end{enumerate}

\section{Шинж чанар инженерчлэл}

\subsection{Үнийн шинж чанарууд}

Үндсэн үнийн өгөгдлөөс дараах шинж чанаруудыг тооцоолсон:

\begin{lstlisting}[language=Python, caption=Price Features]
# Mid price
mid = (high + low) / 2
typical_price = (high + low + close) / 3

# Returns
return_1m = close.pct_change()
log_return = np.log(close).diff()

# Candle features
candle_body = close - open
candle_range = high - low
body_to_range = abs(candle_body) / (candle_range + 1e-9)
\end{lstlisting}

\subsection{Техникийн индикаторууд}

Дараах техникийн индикаторуудыг тооцоолсон:

\begin{table}[H]
\centering
\caption{Техникийн индикаторууд}
\begin{tabular}{|l|l|l|}
\hline
\textbf{Индикатор} & \textbf{Параметр} & \textbf{Тайлбар} \\
\hline
EMA & 9, 21, 55, 200 & Экспоненциал хөдөлгөөнт дундаж \\
\hline
RSI & 14, 21 & Харьцангуй хүчний индекс \\
\hline
MACD & 12, 26, 9 & Хөдөлгөөнт дундажийн нийлэл/салалт \\
\hline
Bollinger Bands & 20, 2 & Хэлбэлзлийн зурвас \\
\hline
ATR & 14, 30 & Дундаж жинхэнэ хүрээ \\
\hline
ADX & 14 & Чиг хандлагын хүч \\
\hline
OBV & - & On-Balance Volume \\
\hline
MFI & 14 & Money Flow Index \\
\hline
VWAP & 30 & Volume Weighted Average Price \\
\hline
\end{tabular}
\label{tab:indicators}
\end{table}

\subsection{Rolling статистик}

Өөр өөр цонхны хэмжээтэй rolling статистикуудыг тооцоолсон:

\begin{lstlisting}[language=Python, caption=Rolling Statistics]
ROLLING_WINDOWS = {
    30: "30m",    # 30 minutes
    120: "2h",    # 2 hours
    240: "4h",    # 4 hours
    720: "12h",   # 12 hours
    1440: "24h"   # 24 hours
}

for window, label in ROLLING_WINDOWS.items():
    roll_close_mean = close.rolling(window).mean()
    roll_close_std = close.rolling(window).std()
    roll_return_mean = return_1m.rolling(window).mean()
    roll_return_std = return_1m.rolling(window).std()
    roll_return_skew = return_1m.rolling(window).skew()
    roll_return_kurt = return_1m.rolling(window).kurt()
\end{lstlisting}

\subsection{Цаг хугацааны шинж чанарууд}

Арилжааны сесс, долоо хоногийн өдөр зэрэг цаг хугацааны шинж чанаруудыг үүсгэсэн:

\begin{itemize}
    \item \textbf{Цикл шинж чанар:} $\sin$ ба $\cos$ хувиргалтаар цагийн цикл мэдээлэл
    \item \textbf{Арилжааны сесс:} Азийн (00:00-07:00 UTC), Лондоны (07:00-16:00 UTC), Нью-Йоркийн (12:00-21:00 UTC)
    \item \textbf{Давхцсан сесс:} Лондон-Нью-Йорк давхцал (12:00-16:00 UTC)
    \item \textbf{Макро цаг:} Эдийн засгийн мэдээ гарах цагууд
\end{itemize}

\begin{lstlisting}[language=Python, caption=Time Features]
# Cyclic encoding
sin_hour = np.sin(2 * np.pi * hour / 24)
cos_hour = np.cos(2 * np.pi * hour / 24)

# Trading sessions
is_asia = (hour >= 0) & (hour < 7)
is_london = (hour >= 7) & (hour < 16)
is_ny = (hour >= 12) & (hour < 21)
is_overlap = is_london & is_ny
\end{lstlisting}

\subsection{Зорилтот хувьсагчууд}

Моделийн зорилтот хувьсагчууд:

\begin{enumerate}
    \item \textbf{target\_pips:} 10 минутын дараах үнийн өөрчлөлт (pips-ээр)
    \begin{equation}
    \text{target\_pips} = \frac{C_{t+10} - C_t}{0.0001}
    \end{equation}
    
    \item \textbf{target\_direction:} Үнэ өсөх эсэх (binary)
    \begin{equation}
    \text{target\_direction} = \mathbf{1}[C_{t+10} > C_t]
    \end{equation}
    
    \item \textbf{target\_up30:} 30+ pip өсөх эсэх
    \begin{equation}
    \text{target\_up30} = \mathbf{1}[\text{target\_pips} \geq 30]
    \end{equation}
    
    \item \textbf{target\_down30:} 30+ pip буурах эсэх
    \begin{equation}
    \text{target\_down30} = \mathbf{1}[\text{target\_pips} \leq -30]
    \end{equation}
\end{enumerate}

\section{Өгөгдлийн хуваалт}

\subsection{Хугацаан дээр суурилсан хуваалт}

Санхүүгийн өгөгдөлд хугацааны дарааллыг хадгалах шаардлагатай тул random split биш, харин temporal split ашигласан:

\begin{figure}[H]
\centering
\begin{tikzpicture}[
    block/.style={rectangle, draw, minimum width=2cm, minimum height=0.8cm, fill=blue!20},
    arrow/.style={-Stealth, thick}
]
    \node[block, fill=green!30, minimum width=5cm] (train) at (0,0) {Train};
    \node[block, fill=yellow!30, minimum width=2cm] (val) at (4,0) {Val};
    \node[block, fill=orange!30, minimum width=1.5cm] (test) at (6,0) {Test};
    \node[block, fill=red!30, minimum width=1cm] (oos) at (7.75,0) {OOS};
    
    \node[below=0.3cm of train] {\small ~6 сар};
    \node[below=0.3cm of val] {\small 60 өдөр};
    \node[below=0.3cm of test] {\small 21 өдөр};
    \node[below=0.3cm of oos] {\small 10 өдөр};
\end{tikzpicture}
\caption{Өгөгдлийн хуваалт}
\label{fig:data_split}
\end{figure}

\subsection{Scaling}

RobustScaler ашиглан outlier-т тэсвэртэй scaling хийсэн:

\begin{equation}
x_{scaled} = \frac{x - \text{median}(x)}{IQR(x)}
\end{equation}

Энд $IQR = Q_{75} - Q_{25}$ (5-95 percentile ашигласан).

\section{Моделийн архитектур}

\subsection{Ерөнхий бүтэц}

Энэхүү судалгаанд CNN + BiLSTM + Multi-Head Attention гибрид архитектур хэрэглэсэн. Энэ архитектур нь:

\begin{enumerate}
    \item \textbf{CNN:} Богино хугацааны локал загвар илрүүлэх
    \item \textbf{BiLSTM:} Урт хугацааны хамаарал загварчлах
    \item \textbf{Multi-Head Attention:} Чухал цаг хугацааны цэгүүдэд анхаарал хандуулах
\end{enumerate}

\begin{figure}[H]
\centering
\begin{tikzpicture}[
    block/.style={rectangle, draw, rounded corners, minimum width=3cm, minimum height=0.8cm, fill=blue!20},
    arrow/.style={-Stealth, thick},
    node distance=0.8cm
]
    % Input
    \node[block, fill=gray!20] (input) {Input (240 × 120)};
    
    % CNN blocks
    \node[block, below=of input] (cnn1) {ResConv Block 1 (192)};
    \node[block, below=of cnn1] (se1) {SE Block};
    \node[block, below=of se1] (cnn2) {ResConv Block 2 (192)};
    \node[block, below=of cnn2] (se2) {SE Block};
    \node[block, below=of se2] (cnn3) {ResConv Block 3 (192)};
    
    % LSTM
    \node[block, below=of cnn3, fill=green!20] (lstm) {BiLSTM (3 layers, 384)};
    
    % Attention
    \node[block, below=of lstm, fill=yellow!20] (attn) {Multi-Head Attention (8 heads)};
    \node[block, below=of attn, fill=yellow!20] (pool) {Attention Pooling};
    
    % Heads
    \node[block, below left=0.8cm and -0.5cm of pool, fill=red!20, minimum width=2cm] (reg) {Regression Head};
    \node[block, below=0.8cm of pool, fill=orange!20, minimum width=2cm] (dir) {Direction Head};
    \node[block, below right=0.8cm and -0.5cm of pool, fill=purple!20, minimum width=2cm] (big) {Big Move Head};
    
    % Arrows
    \draw[arrow] (input) -- (cnn1);
    \draw[arrow] (cnn1) -- (se1);
    \draw[arrow] (se1) -- (cnn2);
    \draw[arrow] (cnn2) -- (se2);
    \draw[arrow] (se2) -- (cnn3);
    \draw[arrow] (cnn3) -- (lstm);
    \draw[arrow] (lstm) -- (attn);
    \draw[arrow] (attn) -- (pool);
    \draw[arrow] (pool) -- (reg);
    \draw[arrow] (pool) -- (dir);
    \draw[arrow] (pool) -- (big);
\end{tikzpicture}
\caption{Моделийн архитектур}
\label{fig:model_arch}
\end{figure}

\subsection{Residual Convolutional Block}

Gradient flow сайжруулахын тулд skip connection бүхий residual block ашигласан:

\begin{lstlisting}[language=Python, caption=Residual Conv Block]
class ResidualConvBlock(nn.Module):
    def __init__(self, in_ch, out_ch, kernel=3, dropout=0.2):
        super().__init__()
        self.conv1 = nn.Conv1d(in_ch, out_ch, kernel, padding=kernel//2)
        self.bn1 = nn.BatchNorm1d(out_ch)
        self.conv2 = nn.Conv1d(out_ch, out_ch, kernel, padding=kernel//2)
        self.bn2 = nn.BatchNorm1d(out_ch)
        self.skip = nn.Conv1d(in_ch, out_ch, 1) if in_ch != out_ch else nn.Identity()
        
    def forward(self, x):
        residual = self.skip(x)
        out = F.gelu(self.bn1(self.conv1(x)))
        out = self.bn2(self.conv2(out))
        return F.gelu(out + residual)
\end{lstlisting}

\subsection{Squeeze-and-Excitation Block}

Channel-wise attention хэрэгжүүлэхэд SE block ашигласан:

\begin{equation}
\text{SE}(x) = x \cdot \sigma(W_2 \cdot \text{ReLU}(W_1 \cdot \text{GAP}(x)))
\end{equation}

\subsection{Multi-Head Self Attention}

8 толгой бүхий self-attention механизм:

\begin{lstlisting}[language=Python, caption=Multi-Head Attention]
class MultiHeadAttention(nn.Module):
    def __init__(self, dim, num_heads=8, dropout=0.1):
        super().__init__()
        self.num_heads = num_heads
        self.head_dim = dim // num_heads
        self.scale = self.head_dim ** -0.5
        self.qkv = nn.Linear(dim, dim * 3)
        self.proj = nn.Linear(dim, dim)
        
    def forward(self, x):
        B, T, C = x.shape
        qkv = self.qkv(x).reshape(B, T, 3, self.num_heads, self.head_dim)
        q, k, v = qkv.permute(2, 0, 3, 1, 4)
        attn = (q @ k.transpose(-2, -1)) * self.scale
        attn = attn.softmax(dim=-1)
        out = (attn @ v).transpose(1, 2).reshape(B, T, C)
        return self.proj(out)
\end{lstlisting}

\section{Сургалтын тохиргоо}

\subsection{Loss функц}

Олон зорилтот сургалтын нийлмэл loss:

\begin{equation}
\mathcal{L}_{total} = \mathcal{L}_{reg} + 0.5 \cdot \mathcal{L}_{dir} + 0.5 \cdot \mathcal{L}_{big}
\end{equation}

\subsubsection{Regression Loss}

Huber loss (outlier-т тэсвэртэй):
\begin{equation}
\mathcal{L}_{reg} = \begin{cases}
\frac{1}{2}(y - \hat{y})^2 & \text{if } |y - \hat{y}| \leq \delta \\
\delta|y - \hat{y}| - \frac{1}{2}\delta^2 & \text{otherwise}
\end{cases}
\end{equation}

\subsubsection{Classification Loss}

Focal Loss (class imbalance шийдвэрлэх):
\begin{equation}
\mathcal{L}_{focal} = -\alpha (1 - p_t)^\gamma \log(p_t)
\end{equation}
Энд $\gamma = 2$, $\alpha = 0.25$.

\subsection{Optimizer ба Scheduler}

\begin{itemize}
    \item \textbf{Optimizer:} AdamW ($\beta_1=0.9$, $\beta_2=0.999$, weight\_decay=$10^{-3}$)
    \item \textbf{Learning Rate:} $3 \times 10^{-4}$
    \item \textbf{Scheduler:} Cosine Annealing with Warm Restarts ($T_0=10$, $T_{mult}=2$)
    \item \textbf{SWA:} Epoch 40-өөс эхлэн Stochastic Weight Averaging
\end{itemize}

\subsection{Regularization техникүүд}

\begin{enumerate}
    \item \textbf{Dropout:} 0.25 (CNN), 0.2 (LSTM, Attention)
    \item \textbf{Label Smoothing:} Direction (0.08), Big Move (0.05)
    \item \textbf{Gradient Clipping:} max\_norm = 1.0
    \item \textbf{Weight Decay:} $10^{-3}$
\end{enumerate}

\subsection{Data Augmentation}

\begin{itemize}
    \item \textbf{Gaussian Noise:} $\sigma = 0.015$
    \item \textbf{Mixup:} $\alpha = 0.2$ (эхний 70\% epoch-д)
    \item \textbf{Time Masking:} 10\% timestep-ийг mask хийх
    \item \textbf{Feature Masking:} 5\% feature-ийг mask хийх
\end{itemize}

\subsection{Hyperparameter}

\begin{table}[H]
\centering
\caption{Hyperparameter тохиргоо}
\begin{tabular}{|l|l|l|}
\hline
\textbf{Параметр} & \textbf{Утга} & \textbf{Тайлбар} \\
\hline
Sequence Length & 240 & 4 цагийн түүхэн өгөгдөл \\
\hline
Batch Size & 256 & Mini-batch хэмжээ \\
\hline
Epochs & 100 & Хамгийн их epoch \\
\hline
Early Stopping & 12 & Patience \\
\hline
Conv Channels & 192 & CNN channel тоо \\
\hline
LSTM Hidden & 384 & LSTM далд хэмжээс \\
\hline
Attention Heads & 8 & Multi-head тоо \\
\hline
\end{tabular}
\label{tab:hyperparams}
\end{table}

\section{Мобайл аппликейшн}

\subsection{React Native архитектур}

Арилжааны дохиог хэрэглэгчдэд хүргэхэд React Native ашиглан мобайл апп хөгжүүлсэн:

\begin{itemize}
    \item \textbf{Frontend:} React Native + Expo
    \item \textbf{State Management:} React Context API
    \item \textbf{API холболт:} Axios
    \item \textbf{Push Notification:} Firebase Cloud Messaging
\end{itemize}

\subsection{Backend API}

FastAPI ашиглан REST API хөгжүүлсэн:

\begin{itemize}
    \item \textbf{/predict:} Шинэ таамаглал авах
    \item \textbf{/signals:} Арилжааны дохионуудын түүх
    \item \textbf{/status:} Системийн төлөв
\end{itemize}
