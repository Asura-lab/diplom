\chapter{ОНОЛЫН ҮНДЭСЛЭЛ}

\section{Машин сургалтын онолын үндэс}

Машин сургалт (Machine Learning) нь өгөгдлөөс автоматаар суралцаж, туршлагаасаа сайжирдаг алгоритмуудыг судалдаг салбар юм. Том Митчелл (1997) машин сургалтыг дараах байдлаар тодорхойлсон \cite{mitchell1997}:

\begin{quote}
``Компьютерийн программ нь T даалгаврын хувьд P гүйцэтгэлийн үзүүлэлтээр хэмжигдэх E туршлагаас суралцсан гэж хэлнэ, хэрэв T даалгаврын P гүйцэтгэл E туршлагын дагуу сайжирсан бол.''
\end{quote}

Энэхүү судалгааны хүрээнд:
\begin{itemize}
    \item \textbf{T (Даалгавар)}: EUR/USD валютын үнийн чиг хандлагыг таамаглах
    \item \textbf{E (Туршлага)}: Түүхэн үнийн өгөгдөл болон техникийн индикаторууд
    \item \textbf{P (Гүйцэтгэл)}: Ангилалын нарийвчлал (accuracy), precision, recall
\end{itemize}

\subsection{Хяналттай сургалт}

Хяналттай сургалт (Supervised Learning) нь шошготой өгөгдлөөс суралцдаг арга юм. Оролтын $X$ болон гаралтын $Y$ хоорондын хамаарлыг $f: X \rightarrow Y$ функцээр дүрсэлнэ. Энэ судалгаанд хоёртын ангилал (binary classification) ашигласан бөгөөд $Y \in \{0, 1\}$ буюу HOLD эсвэл BUY дохио юм.

\section{Ensemble сургалтын аргууд}

Ensemble арга нь олон моделийн таамаглалыг нэгтгэн илүү нарийвчлалтай үр дүнд хүрдэг. Dietterich (2000) ensemble аргын гурван үндсэн давуу талыг тодорхойлсон \cite{dietterich2000}: статистик, тооцооллын болон дүрслэлийн.

\subsection{Random Forest}

Breiman (2001) санал болгосон Random Forest нь bootstrap aggregating (bagging) дээр суурилсан алгоритм юм \cite{breiman2001}. Алгоритм нь $B$ ширхэг шийдвэрийн модыг бүтээж, тэдгээрийн олонхийн саналаар эцсийн таамаглал гаргана:

\begin{equation}
\hat{y} = \text{mode}\{h_1(x), h_2(x), ..., h_B(x)\}
\end{equation}

Энд $h_b(x)$ нь $b$-р модны таамаглал юм.

\subsection{Gradient Boosting}

Gradient Boosting нь алдааг дараалан засах зарчмаар ажилладаг. $m$-р алхам дахь модель нь өмнөх моделийн алдааг (residual) таамаглахад суралцана:

\begin{equation}
F_m(x) = F_{m-1}(x) + \gamma_m h_m(x)
\end{equation}

\subsubsection{XGBoost}

Chen ба Guestrin (2016) XGBoost (eXtreme Gradient Boosting) алгоритмыг санал болгосон \cite{chen2016}. XGBoost нь L1, L2 regularization ашиглан overfitting-аас сэргийлж, параллел тооцоолол хийх боломжтой.

Зорилгын функц:
\begin{equation}
\mathcal{L} = \sum_{i=1}^{n} l(y_i, \hat{y}_i) + \sum_{k=1}^{K} \Omega(f_k)
\end{equation}

Энд $\Omega(f) = \gamma T + \frac{1}{2}\lambda||w||^2$ нь regularization term юм.

\subsubsection{LightGBM}

Ke нар (2017) LightGBM алгоритмыг хөгжүүлсэн \cite{ke2017}. LightGBM нь leaf-wise tree growth стратеги ашигладаг бөгөөд энэ нь level-wise стратегиас илүү хурдан бөгөөд нарийвчлалтай.

\subsection{Voting Ensemble}

Энэ судалгаанд hard voting ensemble ашигласан бөгөөд гурван моделийн олонхийн саналаар эцсийн таамаглал гаргана:

\begin{equation}
\hat{y} = \text{mode}(f_{RF}(x), f_{XGB}(x), f_{LGBM}(x))
\end{equation}

\section{Техникийн шинжилгээ}

Техникийн шинжилгээ нь түүхэн үнэ, хэмжээний өгөгдлөөс ирээдүйн үнийн хөдөлгөөнийг таамаглахад ашиглагддаг. Murphy (1999) техникийн шинжилгээний гурван үндсэн зарчмыг тодорхойлсон \cite{murphy1999}: зах зээл бүх мэдээллийг агуулдаг, үнэ чиг хандлагаар хөдөлдөг, түүх давтагддаг.

\subsection{Trend индикаторууд}

\textbf{Moving Average (MA):} Тодорхой хугацааны дундаж үнийг тооцоолно.
\begin{equation}
SMA_n = \frac{1}{n} \sum_{i=0}^{n-1} P_{t-i}
\end{equation}

\textbf{MACD:} Хоёр EMA-ийн зөрүүгээр momentum-ийг хэмждэг.
\begin{equation}
MACD = EMA_{12} - EMA_{26}
\end{equation}

\subsection{Momentum индикаторууд}

\textbf{RSI (Relative Strength Index):} Wilder (1978) санал болгосон RSI нь 0-100 хооронд хэмжигддэг \cite{wilder1978}.
\begin{equation}
RSI = 100 - \frac{100}{1 + RS}, \quad RS = \frac{\text{Avg Gain}}{\text{Avg Loss}}
\end{equation}

\subsection{Volatility индикаторууд}

\textbf{Bollinger Bands:} Bollinger (2002) хөгжүүлсэн энэ индикатор нь үнийн хэлбэлзлийг хэмждэг \cite{bollinger2002}.
\begin{align}
\text{Upper Band} &= SMA_{20} + 2\sigma \\
\text{Lower Band} &= SMA_{20} - 2\sigma
\end{align}

\section{Холбогдох судалгааны тойм}

Forex таамаглалд машин сургалт ашигласан судалгаанууд сүүлийн жилүүдэд нэмэгдэж байна. Krollner нар (2010) 2010 оноос өмнөх 25 жилийн судалгааг нэгтгэн дүгнэхдээ машин сургалтын аргууд нь уламжлалт статистик аргуудаас илүү үр дүнтэй болохыг тогтоосон \cite{krollner2010}.

Fischer ба Krauss (2018) LSTM сүлжээг S\&P 500 индексийн таамаглалд ашиглаж, уламжлалт аргуудаас давсан үр дүн гаргасан \cite{fischer2018}. Гэсэн хэдий ч энэ судалгаанд ensemble аргыг сонгосон нь тайлбарлах чадвар (interpretability) болон тооцооллын хурдын давуу талтай.

\section{Бүлгийн дүгнэлт}

Энэ бүлэгт судалгааны онолын үндэс болох машин сургалт, ensemble аргууд, техникийн индикаторуудыг авч үзлээ. Дараагийн бүлэгт эдгээр аргуудыг хэрхэн хэрэгжүүлсэн талаар дэлгэрэнгүй тайлбарлана.
