\chapter{ОНОЛЫН ҮНДЭСЛЭЛ}

\section{Машин сургалтын онолын үндэс}

Machine Learning нь өгөгдлөөс автоматаар суралцаж, туршлагаасаа сайжирдаг алгоритмуудыг судалдаг салбар юм. Том Митчелл (1997) машин сургалтыг дараах байдлаар тодорхойлсон \cite{mitchell1997}:

\begin{quote}
``Компьютерийн программ нь T даалгаврын хувьд P гүйцэтгэлийн үзүүлэлтээр хэмжигдэх E туршлагаас суралцсан гэж хэлнэ, хэрэв T даалгаврын P гүйцэтгэл E туршлагын дагуу сайжирсан бол.''
\end{quote}

Энэхүү судалгааны хүрээнд:
\begin{itemize}
    \item \textbf{T (Даалгавар)}: EUR/USD валютын үнийн чиг хандлагыг таамаглах
    \item \textbf{E (Туршлага)}: Түүхэн үнийн өгөгдөл болон Technical Indicators
    \item \textbf{P (Гүйцэтгэл)}: Ангилалын accuracy, recall, precision
\end{itemize}

\subsection{Supervised Learning}

Supervised Learning нь шошготой өгөгдлөөс суралцдаг арга юм. Оролтын $X$ болон гаралтын $Y$ хоорондын хамаарлыг $f: X \rightarrow Y$ функцээр дүрсэлнэ. Энэ судалгаанд binary classification ашигласан бөгөөд $Y \in \{0, 1\}$ буюу HOLD эсвэл BUY дохио юм.

\section{Ensemble Learning}

Ensemble Learning арга нь олон загварын таамаглалыг нэгтгэн илүү нарийвчлалтай үр дүнд хүрдэг. Dietterich (2000) ensemble аргын гурван үндсэн давуу талыг тодорхойлсон \cite{dietterich2000}: статистик, тооцооллын болон дүрслэлийн.

\subsection{Gradient Boosting}

Gradient Boosting нь алдааг дараалан засах зарчмаар ажилладаг. $m$-р алхам дахь загвар нь өмнөх загварын алдааг (residual) таамаглахад суралцана:

\begin{equation}
F_m(x) = F_{m-1}(x) + \gamma_m h_m(x)
\end{equation}

\subsubsection{XGBoost}

Chen ба Guestrin (2016) XGBoost (eXtreme Gradient Boosting) алгоритмыг санал болгосон \cite{chen2016}. XGBoost нь L1, L2 regularization ашиглан overfitting-аас сэргийлж, параллел тооцоолол хийх боломжтой.

Зорилгын функц:
\begin{equation}
\mathcal{L} = \sum_{i=1}^{n} l(y_i, \hat{y}_i) + \sum_{k=1}^{K} \Omega(f_k)
\end{equation}

Энд $\Omega(f) = \gamma T + \frac{1}{2}\lambda||w||^2$ нь regularization term юм.

\subsubsection{LightGBM}

Ke, Meng нар (2017) LightGBM алгоритмыг хөгжүүлсэн \cite{ke2017}. LightGBM нь leaf-wise tree growth стратеги ашигладаг бөгөөд энэ нь level-wise стратегиас илүү хурдан бөгөөд нарийвчлалтай.

\subsubsection{CatBoost}

Yandex-ийн хөгжүүлсэн CatBoost нь categorical feature-тэй сайн ажилладаг Gradient Boosting алгоритм юм. Ordered boosting арга ашиглан target leakage-аас сэргийлдэг бөгөөд hyperparameter тохируулга бага шаарддаг.

\subsection{Hybrid Ensemble арга}

Энэ судалгаанд Hybrid Ensemble гэж нэрлэгдсэн 7 загварыг нэгтгэх арга ашигласан. 3 XGBoost, 2 LightGBM, 2 CatBoost загварыг нэгтгэж, Agreement Bonus System-ээр итгэлцүүрийг нэмэгдүүлсэн:

\begin{equation}
P_{final} = \sum_{i=1}^{7} w_i \cdot P_i(BUY) + \text{Agreement Bonus}
\end{equation}

Agreement Bonus System нь загваруудын зөвшилцлөөс хамааран +7\% (7/7), +4\% (6/7), +2\% (5/7) нэмдэг.

\section{Technical Analysis}

Technical Analysis нь түүхэн үнэ, хэмжээний өгөгдлөөс ирээдүйн үнийн хөдөлгөөнийг таамаглахад ашиглагддаг. Murphy (1999) Technical Analysis-ийн гурван үндсэн зарчмыг тодорхойлсон \cite{murphy1999}: зах зээл бүх мэдээллийг агуулдаг, үнэ чиг хандлагаар хөдөлдөг, түүх давтагддаг.

\subsection{Trend Indicators}

\textbf{Moving Average (MA):} Тодорхой хугацааны дундаж үнийг тооцоолно.
\begin{equation}
SMA_n = \frac{1}{n} \sum_{i=0}^{n-1} P_{t-i}
\end{equation}

\textbf{MACD:} Хоёр EMA-ийн зөрүүгээр momentum-ийг хэмждэг.
\begin{equation}
MACD = EMA_{12} - EMA_{26}
\end{equation}

\subsection{Momentum Indicators}

\textbf{RSI (Relative Strength Index):} Wilder (1978) санал болгосон RSI нь 0-100 хооронд хэмжигддэг \cite{wilder1978}.
\begin{equation}
RSI = 100 - \frac{100}{1 + RS}, \quad RS = \frac{\text{Average Gain}}{\text{Average Loss}}
\end{equation}

\subsection{Volatility Indicators}

\textbf{Bollinger Bands:} Bollinger (2002) хөгжүүлсэн энэ индикатор нь үнийн хэлбэлзлийг хэмждэг \cite{bollinger2002}.
\begin{align}
\text{Upper Band} &= SMA_{20} + 2\sigma \\
\text{Lower Band} &= SMA_{20} - 2\sigma
\end{align}

\section{Холбогдох судалгааны тойм}

Forex таамаглалд машин сургалт ашигласан судалгаанууд сүүлийн жилүүдэд нэмэгдэж байна. Krollner, Vanstone нар (2010) 2010 оноос өмнөх 25 жилийн судалгааг нэгтгэн дүгнэхдээ машин сургалтын аргууд нь уламжлалт статистик аргуудаас илүү үр дүнтэй болохыг тогтоосон \cite{krollner2010}.

Fischer ба Krauss (2018) LSTM сүлжээг S\&P 500 индексийн таамаглалд ашиглаж, уламжлалт аргуудаас давсан үр дүн гаргасан \cite{fischer2018}.
