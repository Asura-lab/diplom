\chapter{Судалгааны сэдвийн онол, өнөөгийн түвшин}
\label{ch:literature}

\section{Forex зах зээлийн таамаглалд машин сургалтыг хэрэглэсэн судалгаанууд}

\subsection{Уламжлалт машин сургалтын аргууд}

Forex зах зээлийн таамаглалд машин сургалтын аргыг хэрэглэсэн эрт үеийн судалгаанууд нь голчлон neural networks болон support vector machines (SVM)-д төвлөрч байсан. Kamruzzaman ба Sarker (2003) нь нейрон сүлжээг ашиглан валютын ханш таамаглахад 81\%-ийн нарийвчлал хүрсэн \cite{kamruzzaman2003forecasting}. Гэвч судалгааны дараах сул талууд байна: (1) зөвхөн нэг валютын хослолд хязгаарлагдсан, (2) overfitting-ийн асуудлыг тодорхой авч үзээгүй, (3) бэктест хийгээгүй. Kumar ба Thenmozhi (2006) нь SVM-ийг ашиглан USD/INR хослолыг судалж 67\%-ийн нарийвчлал авсан боловч модель нь зах зээлийн горимын өөрчлөлтөд мэдрэмтгий байсан \cite{kumar2006forecasting}.

Random Forest болон Gradient Boosting аргууд нь санхүүгийн зах зээлийн таамаглалд илүү тогтвортой үр дүн үзүүлдэг. Ballings нар (2015) нь хувьцааны үнийн чиглэл таамаглахад олон машин сургалтын загваруудыг харьцуулан үнэлж, \textbf{ансамбль аргууд нь дангаар ажилладаг загваруудаас 3--7\% нарийвчлалын ахиц хүрсэн} гэж тогтоосон \cite{ballings2015evaluating}. Гэсэн хэдий ч, Forex зах зээлд энэ аргуудыг бүрэн туршиж үзээгүй нь судалгаан дахь цоорхой хэвээр байна.

\subsection{Гүн сургалтын аргууд}

Сүүлийн жилүүдэд Long Short-Term Memory (LSTM) болон гүн нейрон сүлжээ (DNN) нь санхүүгийн цаг хугацааны цуваа таамаглалд түгээмэл ашиглагдаж эхэлсэн. Fischer ба Krauss (2018) нь LSTM-ийг S\&P 500 хувьцааны зах зээлд хэрэглэж, уламжлалт аргуудаас давуу гүйцэтгэлтэй байсныг харуулсан \cite{fischer2018deep}. Zhang нар (2019) нь LSTM сүлжээг алтны үнэ таамаглахад хэрэглэж 58.7\% нарийвчлал авсан \cite{zhang2019deep}.

Гэсэн хэдий ч, \textbf{гүн сургалтын аргуудын дараах сул талууд} илэрхий байна \cite{krauss2017deep}:
\begin{itemize}
    \item \textbf{Тайлбарлах боломжгүй (black-box)}: Санхүүгийн шийдвэр гаргалтанд шаардлагатай тайлбарлах чадвар дутмаг
    \item \textbf{Их параметр, удаан сургалт}: 100,000+ параметртэй, өгөгдөл их шаарддаг
    \item \textbf{Overfitting-д эмзэг}: Түүхэн өгөгдөлд хэт тохирч, шинэ зах зээлийн нөхцөлд муу ажилладаг
    \item \textbf{Тогтворгүй}: Анхны параметрээс их хамаардаг
\end{itemize}

Krauss нар (2017) нь гүн нейрон сүлжээ болон gradient-boosted trees загваруудыг S\&P 500 дээр харьцуулж, \textbf{GBDT загварууд илүү тогтвортой үр дүн үзүүлсэн} гэж дүгнэсэн \cite{krauss2017deep}. Энэ нь бидний судалгаанд GBDT ансамбль загваруудыг ашиглах үндэслэл болж байна.

\subsection{Gradient Boosted Decision Trees санхүүгийн таамаглалд}

GBDT нь Friedman (2001)-ийн боловсруулсан олон сул суралцагч (decision tree)-ийг дараалалтай нэмж, алдааг (gradient) бууруулах аргачлал юм \cite{friedman2001greedy}. Gu нар (2020) нь 150+ шинж чанартай GBDT загварыг хувьцааны өгөөж таамаглахад хэрэглэж, \textbf{гүн сургалтын загваруудтай харьцуулахад дутуугүй нарийвчлалтай} байсныг тогтоосон \cite{gu2020empirical}. Тэд мөн GBDT загваруудын дараах давуу талуудыг дурдсан:
\begin{itemize}
    \item Табулар өгөгдөлд тохирсон feature interaction-г олж илрүүлдэг
    \item Цөөн параметртэй -- хурдан сургаж болно
    \item Feature importance хялбар тооцоолох -- тайлбарлах боломжтой
    \item Outlier-д мэдрэмтгий бус
\end{itemize}

XGBoost \cite{chen2016xgboost}, LightGBM \cite{ke2017lightgbm}, CatBoost \cite{prokhorenkova2018catboost} нь GBDT-ийн орчин үеийн хэрэгжүүлэлт бөгөөд өөр өөр онцлогтой. XGBoost нь L1/L2 regularization давуу тал, LightGBM нь хурдтай боловч overfitting-д илүү эмзэг, CatBoost нь ordered boosting ашиглан target leakage-аас сэргийлдэг. Гэсэн хэдий ч, \textbf{эдгээр загваруудыг Forex зах зээлд ансамбль хэлбэрээр хэрэглэж, харьцуулсан судалгаа маш хомс} байна. Энэ нь бидний судалгаа дүүргэх гэж буй gap байна.

\subsection{Ансамбль аргуудын санхүүгийн хэрэглээ}

Ансамбль аргын үндсэн санаа нь олон загварын таамгийг нэгтгэснээр нэг загварын алдааг нөхөж чаддагт оршино \cite{dietterich2000ensemble}. Breiman (2001) Random Forest аргаар олон модыг параллель сургаж, average-ээр нэгтгэх замаар variance-г бууруулсан \cite{breiman2001random}. Sezer нар (2020) нь хувьцааны зах зээлийн таамаглалд машин сургалтын аргуудыг тоймлосон судалгаандаа \textbf{ансамбль аргууд нь дангаар ажилладаг загваруудаас тогтмол давуу гүйцэтгэлтэй байдгийг} тэмдэглэсэн \cite{sezer2020financial}.

\textbf{Өмнөх судалгаануудын gap-ууд}:
\begin{itemize}
    \item Ихэнх нь хувьцааны зах зээлд төвлөрсөн -- Forex зах зээлийн онцлог (24/5 арилжаа, макро эдийн засгийн event-ийн нөлөөлөл, ам.доллар индекс) харгалзаагүй
    \item Дан загвар эсвэл нэг төрлийн ансамбль (жишээ нь зөвхөн Random Forest) хэрэглэсэн -- \textbf{өөр өөр алгоритмуудын (XGBoost, LightGBM, CatBoost) ансамбль судлаагүй}
    \item Walk-forward validation хийж, цаг хугацааны leak-аас сэргийлээгүй
    \item Практик хэрэглээ (мобайл аппликейшн, бодит цагийн дохио үүсгэх) хэрэгжүүлээгүй
    \item Олон хугацааны интервалын (multi-timeframe) мэдээлэл хангалттай ашиглаагүй
\end{itemize}

\section{Техник шинжилгээний индикаторуудын судалгаа}

\subsection{Индикаторын үр дүнтэй байдлын харьцуулалт}

Murphy (1999) нь техник шинжилгээний системчилсэн танилцуулга өгсөн боловч эмпирик баталгаажуулалт дутмаг байсан \cite{murphy1999technical}. Fama (1970)-ын Efficient Market Hypothesis (EMH)-аар техник дүн шинжилгээ үр дүнгүй байх ёстой гэсэн боловч, Lo нар (2000) нь нарийвчилсан статистик тестээр \textbf{техник аргууд статистик ач холбогдолтой үр дүн өгч чадах} үндэслэлийг нотолсон \cite{fama1970efficient, lo2000foundations}. Энэ нь EMH нь зах зээлийн бүх төрөлд бүрэн хамаарна гэсэнтэй маргаж байна.

Olson (2004) нь RSI, MACD, Stochastic индикаторуудыг Forex зах зээлд судалж, \textbf{нэг индикатор дангаараа тогтвортой үр дүн өгч чадахгүй} гэж дүгнэсэн \cite{olson2004have}. Тэр өөр хугацааны chart (daily vs hourly) дээр ялгаатай үр дүн гарч, индикаторууд нь зах зээлийн нөхцөлөөс хамаардаг гэж тэмдэглэсэн. Энэ нь \textbf{олон индикаторыг хослуулан, олон хугацааны интервалаас ашиглах} шаардлагатайг харуулж байна -- энэхүү судалгаанд 48 индикатор 6 хугацааны интервалаас (M1, M5, M15, M30, H1, H4) тооцоолсон шалтгаан.

Brock нар (1992) нь moving average crossover стратегиудыг судалж, тодорхой нөхцөлд ашигтай байж болохыг харуулсан боловч \textbf{transaction cost-ыг тооцох үед ашиг ихээхэн багассан} \cite{brock1992simple}. Энэ нь MetaTrader 5 бэктестд spread (EUR/USD дээр ~1.5 pips), slippage (~0.5 pips), swap тооцоолох шаардлагатайг баталж байна. Marshall нар (2008) нь emerging markets дээр техник аргуудын үр дүнтэй ажилладаг төдий засвартай (developed) зах зээл дээр үр дүн муудаж байгааг тогтоосон \cite{marshall2008technical}.

\subsection{Индикаторуудын хослуулалт ба feature engineering}

Park ба Irwin (2007) нь 95 судалгааг тоймлон, техник шинжилгээний аргууд 1995 хүртэл үр дүнтэй байсан боловч сүүлийн жилүүдэд үр дүн буурч байгааг тогтоосон -- энэ нь \textbf{зах зээлийн дасан зохицох чадвар (adaptive efficiency)-ын} илрэл \cite{park2007we}. Энэ нь дангаар ашиглавал үр дүнгүй боловч машин сургалтын загварт feature-ээр хэрэглэхэд илүү ашигтай байж болно гэсэн үг.

Техник индикаторууд нь GBDT загварт feature-ээр ашиглагдахад үр дүнтэй байдаг. Ихэнх судалгаанууд зөвхөн нэг хугацааны интервал (жишээ нь hourly эсвэл daily) ашигладаг боловч, \textbf{multi-timeframe analysis-ийн давуу тал нь харагдаагүй} байна. Энэхүү судалгаанд өөр өөр давтамжийн хэв маягийг илрүүлэхийн тулд M1-ээс H4 хүртэлх 6 интервалаас индикаторууд тооцоолсон нь судалгааны нэмэлт хувь нэмэр болно.

\section{Walk-Forward Validation ба санхүүгийн цаг хугацааны цуваа}

Уламжлалт k-fold cross-validation нь санхүүгийн цуваа өгөгдөлд тохиромжгүй, учир нь цаг хугацааны дарааллыг зөрчиж, look-ahead bias үүсгэдэг. Pardo (2008) нь walk-forward validation (WFV)-г танилцуулж, энэхүү аргачлал нь цаг хугацааны дарааллыг хадгалж, загварыг ирээдүйн өгөгдөл дээр шалгахад илүү бодит болохыг харуулсан \cite{pardo2008evaluation}.

\textbf{Walk-forward validation-ийн давуу талууд}:
\begin{itemize}
    \item Look-ahead bias-аас сэргийлнэ -- ирээдүйн өгөгдөл хэзээ ч сургалтад алдагдахгүй
    \item Зах зээлийн горимын өөрчлөлт (regime change) илрэх боломжтой
    \item Бодит арилжааны нөхцөлд илүү ойр -- тухайн үе бүрт загварыг дахин сургах шаардлагатай
\end{itemize}

Bailey нар (2014) нь санхүүгийн загваруудын overfitting-ийн эрсдэлийг задлан, \textbf{олон параметр туршилт (parameter tuning) хийх нь in-sample үр дүнг сайжруулна гэхдээ out-of-sample дээр муудна} гэж анхааруулсан \cite{bailey2014probability}. Энэ нь WFV ашиглах чухал шалтгаан -- энэхүү судалгаанд 2015--2022 (сургалт), 2023 (баталгаажуулалт), 2024 (тест), 2025 (бэктест) гэж тусгаарлаж, overfitting-аас сэргийлсэн.

\section{Forex арилжааны аппликейшн ба технологийн хэрэгжүүлэлт}

\subsection{Мобайл арилжааны технологи}

Гар утасны арилжааны аппликейшнууд санхүүгийн салбарт хурдацтай нэвтэрч байна. Charles Schwab-ийн 2023 оны судалгаагаар хөрөнгө оруулагчдын 60\%-аас дээш нь гар утсаар арилжаа хийдэг болсон. React Native \cite{reactnative} нь нэг код бааз дээр iOS болон Android аппликейшн хөгжүүлэх боломж олгодог cross-platform фреймворк юм. Eisenman нар (2018) нь React Native-ийн давуу тал нь хурдан хөгжүүлэлт, нэг код бааз боловч гүйцэтгэл нь native apps-аас бага зэрэг унадаг гэж тогтоосон \cite{eisenman2018react}.

\subsection{Backend технологи ба API design}

Backend систем нь FastAPI framework дээр хөгжүүлсэн. FastAPI нь Python-д зориулсан орчин үеийн веб framework бөгөөд async/await дэмждэг, автомат API documentation (Swagger/OpenAPI) үүсгэдэг, type hints ашигладаг \cite{fastapi}. Энэ нь бодит цагийн өгөгдөл (Twelve Data API, Alpha Vantage API) татах, ML загварууд ажиллуулах, мобайл app-д RESTful API-аар мэдээлэл дамжуулах боломжийг олгоно.

\section{Судалгааны gap ба энэхүү ажлын хувь нэмэр}

Дээр дурдсан судалгаануудыг тоймлон үзвэл, дараах \textbf{судалгааны gaps} илэрхий байна:

\begin{enumerate}
    \item \textbf{Ансамбль GBDT загваруудыг Forex зах зээлд хэрэглэсэн судалгаа хомс}: Ихэнх нь дан загвар эсвэл хувьцааны зах зээлд төвлөрсөн
    \item \textbf{Олон хугацааны интервалын (multi-timeframe) шинж чанар ашиглаагүй}: Зөвхөн нэг chart timeframe ашигласан
    \item \textbf{Walk-forward validation дутмаг}: Overfitting шалгалт сул
    \item \textbf{Бодит арилжааны нөхцөл (spread, slippage, swap) тооцоолоогүй}: Бэктест нь түүхэн дүн шинжилгээнд л хязгаарлагдсан
    \item \textbf{Практик хэрэглээ (мобайл app, бодит цагийн дохио) хийгээгүй}: Академик судалгаа л болсон
\end{enumerate}

\textbf{Энэхүү судалгааны шинэлэг хувь нэмэр}:
\begin{itemize}
    \item \textbf{XGBoost, LightGBM, CatBoost гурван загварын ансамбль}: Өөр өөр алгоритмуудын давуу талыг нэгтгэсэн
    \item \textbf{48 техник индикатор 6 хугацааны интервалаас}: M1--H4 олон давтамжийн дүн шинжилгээ
    \item \textbf{Walk-forward validation + бодит MT5 бэктест}: Overfitting сэргийлж, бодит арилжааны нөхцөлд шалгасан
    \item \textbf{``Predictrix'' React Native аппликейшн}: Хэрэглэгчдэд хүртээмжтэй, практик хэрэглээтэй систем
    \item \textbf{End-to-end шийдэл}: Өгөгдлөөс эхлэн ML сургалт, бэктест, мобайл app хүртэлх бүрэн систем
\end{itemize}

\section{Бүлгийн дүгнэлт}

Энэ бүлэгт Forex зах зээлийн таамаглалд машин сургалтыг хэрэглэсэн судалгаануудыг шүүмжлэлтэй авч үзлээ. Уламжлалт ML аргууд болон гүн сургалтын аргууд дээр тулгуурлан судалгаанууд хийгдсэн боловч, \textbf{GBDT ансамбль аргууд} нь санхүүгийн табулар өгөгдөлд хамгийн тохиромжтой гэдгийг олон судалгаанууд харуулсан. Гэсэн хэдий ч, Forex зах зээлд, ялангуяа олон хугацааны интервалын мэдээлэл ашиглан, walk-forward validation хийх, бодит арилжааны нөхцөлд шалгах талаар судалгаа хомс байна.

Техник индикаторууд нь дангаараа үр дүнгүй боловч ML загварын feature-ээр ашиглахад илүү үр дүнтэй байдаг. Олон хугацааны интервалын дүн шинжилгээ, walk-forward validation, бодит бэктест зэрэг нь энэхүү судалгааны гол багана болж байна. Дараагийн бүлэгт судалгааны арга зүйг дэлгэрэнгүй танилцуулна.
