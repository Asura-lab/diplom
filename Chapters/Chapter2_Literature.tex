\chapter{ОНОЛЫН СУДАЛГАА}

\section{Машин сургалтын үндэс}

\subsection{Машин сургалтын тодорхойлолт}

Машин сургалт (Machine Learning) нь хиймэл оюун ухааны (Artificial Intelligence) дэд салбар бөгөөд компьютерийн системд тодорхой програмчлалгүйгээр өгөгдлөөс суралцах чадварыг олгодог арга зүй юм \cite{mitchell1997}. Tom Mitchell-ийн тодорхойлолтоор: ``Компьютерийн програм нь T даалгаврын хувьд P гүйцэтгэлийн хэмжүүрээр хэмжигдэх үед E туршлагаас суралцаж байна гэж хэлнэ, хэрэв E туршлага нэмэгдэхэд T даалгаврын P гүйцэтгэл сайжирвал.''

\subsection{Машин сургалтын төрлүүд}

Машин сургалтыг гурван үндсэн төрөлд хуваадаг:

\subsubsection{Хяналттай сургалт (Supervised Learning)}

Хяналттай сургалт нь хамгийн түгээмэл хэрэглэгддэг машин сургалтын арга бөгөөд оролт-гаралтын хос өгөгдөл дээр суралцдаг. Энэ аргад:

\begin{itemize}
    \item Сургалтын өгөгдөл нь $(x_i, y_i)$ хос хэлбэртэй байдаг
    \item $x_i$ - оролтын өгөгдөл (features)
    \item $y_i$ - зорилтот утга (label/target)
    \item Модель нь $f: X \rightarrow Y$ функцийг суралцдаг
\end{itemize}

Хяналттай сургалт нь дараах хэрэглээнд өргөн хэрэглэгддэг:
\begin{enumerate}
    \item \textbf{Regression (Регресс):} Тасралтгүй утга таамаглах (жишээ нь: үнийн таамаглал)
    \item \textbf{Classification (Ангилал):} Дискрет ангилалд хуваах (жишээ нь: өсөх/буурах таамаглал)
\end{enumerate}

\subsubsection{Хяналтгүй сургалт (Unsupervised Learning)}

Хяналтгүй сургалт нь зөвхөн оролтын өгөгдөл дээр суралцдаг бөгөөд зорилтот утга байхгүй. Гол хэрэглээ нь:

\begin{itemize}
    \item Кластерчлал (Clustering)
    \item Хэмжээс бууруулалт (Dimensionality Reduction)
    \item Аномали илрүүлэлт (Anomaly Detection)
\end{itemize}

\subsubsection{Бататгах сургалт (Reinforcement Learning)}

Бататгах сургалт нь агент орчинтой харилцан үйлдэл хийж, шагнал/шийтгэлийн дагуу оновчтой бодлого суралцдаг арга юм. Энэ арга нь арилжааны стратеги сургахад өргөн хэрэглэгддэг.

\section{Гүн сургалт (Deep Learning)}

\subsection{Мэдрэлийн сүлжээний үндэс}

Хиймэл мэдрэлийн сүлжээ (Artificial Neural Network) нь хүний тархины мэдрэлийн эсүүдийн бүтцээс санаа авсан тооцоолох загвар юм. Нэг нейрон нь дараах томъёогоор илэрхийлэгдэнэ:

\begin{equation}
y = \sigma\left(\sum_{i=1}^{n} w_i x_i + b\right)
\end{equation}

Энд:
\begin{itemize}
    \item $x_i$ - оролтын утгууд
    \item $w_i$ - жингүүд
    \item $b$ - bias
    \item $\sigma$ - идэвхжүүлэх функц (activation function)
\end{itemize}

\subsection{Конволюшн мэдрэлийн сүлжээ (CNN)}

CNN нь дүрс боловсруулалтад өргөн хэрэглэгддэг боловч цуваа өгөгдөл дээр локал загвар илрүүлэхэд маш үр дүнтэй. 1D CNN нь санхүүгийн цуваа өгөгдөлд богино хугацааны загваруудыг илрүүлэхэд хэрэглэгддэг.

Конволюшн үйлдэл:
\begin{equation}
(x * k)[n] = \sum_{m=-M}^{M} x[n-m] \cdot k[m]
\end{equation}

Энд $x$ нь оролтын цуваа, $k$ нь kernel буюу шүүлтүүр юм.

\subsection{Рекуррент мэдрэлийн сүлжээ (RNN) ба LSTM}

RNN нь цуваа өгөгдөл боловсруулахад зориулагдсан бөгөөд өмнөх алхмын мэдээллийг дараагийн алхамд дамжуулдаг:

\begin{equation}
h_t = \tanh(W_{hh} h_{t-1} + W_{xh} x_t + b_h)
\end{equation}

Гэвч энгийн RNN нь ``vanishing gradient'' асуудалтай учир урт хугацааны хамаарлыг сурахад хүндрэлтэй. LSTM (Long Short-Term Memory) нь энэ асуудлыг шийдвэрлэхээр зохиогдсон:

\begin{figure}[H]
\centering
\begin{tikzpicture}[
    gate/.style={rectangle, draw, minimum width=1cm, minimum height=0.6cm, fill=blue!20},
    op/.style={circle, draw, minimum size=0.5cm, fill=yellow!30},
    arrow/.style={-Stealth, thick}
]
    % Gates
    \node[gate] (forget) at (0,0) {$f_t$};
    \node[gate] (input) at (2,0) {$i_t$};
    \node[gate] (cell) at (4,0) {$\tilde{c}_t$};
    \node[gate] (output) at (6,0) {$o_t$};
    
    % Labels
    \node[below=0.3cm of forget] {\small Forget Gate};
    \node[below=0.3cm of input] {\small Input Gate};
    \node[below=0.3cm of cell] {\small Cell State};
    \node[below=0.3cm of output] {\small Output Gate};
    
    % Connections
    \draw[arrow] (-1.5,0) -- (forget);
    \draw[arrow] (forget) -- (input);
    \draw[arrow] (input) -- (cell);
    \draw[arrow] (cell) -- (output);
    \draw[arrow] (output) -- (7.5,0);
\end{tikzpicture}
\caption{LSTM нэгжийн бүтэц}
\label{fig:lstm}
\end{figure}

LSTM-ийн томъёонууд:
\begin{align}
f_t &= \sigma(W_f \cdot [h_{t-1}, x_t] + b_f) \quad \text{(Forget gate)} \\
i_t &= \sigma(W_i \cdot [h_{t-1}, x_t] + b_i) \quad \text{(Input gate)} \\
\tilde{c}_t &= \tanh(W_c \cdot [h_{t-1}, x_t] + b_c) \quad \text{(Candidate)} \\
c_t &= f_t \odot c_{t-1} + i_t \odot \tilde{c}_t \quad \text{(Cell state)} \\
o_t &= \sigma(W_o \cdot [h_{t-1}, x_t] + b_o) \quad \text{(Output gate)} \\
h_t &= o_t \odot \tanh(c_t) \quad \text{(Hidden state)}
\end{align}

\subsection{Bidirectional LSTM (BiLSTM)}

BiLSTM нь хоёр чиглэлд (урагш, хойшоо) цуваа өгөгдлийг боловсруулдаг бөгөөд илүү баялаг контекст мэдээлэл олж авдаг:

\begin{equation}
h_t = [\overrightarrow{h_t}; \overleftarrow{h_t}]
\end{equation}

\subsection{Attention механизм}

Attention механизм нь моделд оролтын цувааны аль хэсэгт илүү анхаарал хандуулахыг сурах боломж олгодог:

\begin{equation}
\text{Attention}(Q, K, V) = \text{softmax}\left(\frac{QK^T}{\sqrt{d_k}}\right)V
\end{equation}

Multi-Head Attention нь олон толгойг ашиглан өөр өөр representation subspace-ийн мэдээллийг авдаг:

\begin{equation}
\text{MultiHead}(Q, K, V) = \text{Concat}(\text{head}_1, ..., \text{head}_h)W^O
\end{equation}

\section{Санхүүгийн зах зээлийн техникийн шинжилгээ}

\subsection{Техникийн индикаторууд}

Техникийн шинжилгээ нь үнийн түүхэн өгөгдөл дээр суурилан ирээдүйн үнийн хөдөлгөөнийг таамаглах арга юм.

\subsubsection{Хөдөлгөөнт дундаж (Moving Average)}

Экспоненциал хөдөлгөөнт дундаж (EMA):
\begin{equation}
EMA_t = \alpha \cdot P_t + (1 - \alpha) \cdot EMA_{t-1}
\end{equation}
Энд $\alpha = \frac{2}{n+1}$, $n$ нь үе.

\subsubsection{RSI (Relative Strength Index)}

RSI нь momentum индикатор бөгөөд 0-100 хооронд хэлбэлздэг:
\begin{equation}
RSI = 100 - \frac{100}{1 + RS}
\end{equation}
Энд $RS = \frac{\text{Дундаж өсөлт}}{\text{Дундаж бууралт}}$

\subsubsection{MACD (Moving Average Convergence Divergence)}

\begin{align}
MACD &= EMA_{12} - EMA_{26} \\
Signal &= EMA_9(MACD) \\
Histogram &= MACD - Signal
\end{align}

\subsubsection{Bollinger Bands}

\begin{align}
Middle &= SMA_{20} \\
Upper &= Middle + 2\sigma \\
Lower &= Middle - 2\sigma
\end{align}

\subsubsection{ATR (Average True Range)}

Хэлбэлзлийн хэмжүүр:
\begin{equation}
TR = \max(H_t - L_t, |H_t - C_{t-1}|, |L_t - C_{t-1}|)
\end{equation}
\begin{equation}
ATR = EMA_{14}(TR)
\end{equation}

\section{Ижил төстэй судалгааны тойм}

\subsection{Уламжлалт машин сургалтын аргууд}

Fischer \& Krauss (2018) \cite{fischer2018} нь LSTM сүлжээг S\&P 500 индексийн чиглэлийн таамаглалд хэрэглэж, уламжлалт аргуудаас давуу гүйцэтгэл үзүүлснийг харуулсан. Random Forest болон Gradient Boosting аргууд 54-56\% нарийвчлалтай байхад LSTM 58-60\% нарийвчлалтай байв.

\subsection{Гүн сургалтын аргууд}

Sezer et al. (2020) \cite{sezer2020} нь CNN ашиглан санхүүгийн цуваа өгөгдлийг 2D дүрс болгон хувиргаж, техникийн индикаторуудын загварыг илрүүлсэн. Энэ арга нь FOREX зах зээл дээр 55-57\% нарийвчлалтай байв.

\subsection{Гибрид аргууд}

Kim \& Won (2018) \cite{kim2018} нь CNN + LSTM гибрид архитектурыг санхүүгийн таамаглалд ашигласан. CNN нь локал загвар илрүүлж, LSTM нь цаг хугацааны хамаарлыг загварчилсан. Энэ хослол нь тус тусдаа ашиглахаас илүү үр дүнтэй байв.

\subsection{Attention-д суурилсан аргууд}

Ding et al. (2020) \cite{ding2020} нь Transformer архитектурыг хувьцааны үнийн таамаглалд ашиглаж, LSTM-ээс давуу гүйцэтгэл үзүүлснийг нотолсон. Self-attention механизм нь урт хугацааны хамаарлыг илүү сайн загварчилдаг.

\subsection{Судалгааны цоорхой}

Одоогийн судалгаануудад дараах цоорхой байна:
\begin{enumerate}
    \item Ихэнх судалгаа нь хувьцааны зах зээлд төвлөрсөн, FOREX зах зээлийн судалгаа харьцангуй цөөн
    \item Multi-task learning буюу олон зорилтот сургалтын хэрэглээ хомс
    \item Regression болон classification-ийг нэгтгэсэн судалгаа цөөн
    \item Орчин үеийн техникүүд (Focal Loss, SWA) санхүүгийн таамаглалд бага хэрэглэгдсэн
\end{enumerate}

\begin{table}[H]
\centering
\caption{Холбогдох судалгаануудын харьцуулалт}
\begin{tabular}{|l|c|c|c|c|}
\hline
\textbf{Судалгаа} & \textbf{Арга} & \textbf{Зах зээл} & \textbf{Нарийвчлал} & \textbf{Он} \\
\hline
Fischer \& Krauss & LSTM & S\&P 500 & 58-60\% & 2018 \\
\hline
Sezer et al. & CNN & FOREX & 55-57\% & 2020 \\
\hline
Kim \& Won & CNN+LSTM & KOSPI & 56-58\% & 2018 \\
\hline
Ding et al. & Transformer & China A-share & 57-59\% & 2020 \\
\hline
\textbf{Энэ судалгаа} & \textbf{CNN+BiLSTM+Attn} & \textbf{FOREX} & \textbf{TBD} & \textbf{2025} \\
\hline
\end{tabular}
\label{tab:related_work}
\end{table}
